\chapter{The Passage of Recreational Marijuana Legalization in the United States, 1990--2016}



%\begin{abstract}
%\begin{singlespace}
%What drives the diffusion of drug policy in the United States? This study examines the dramatic increase in the number of states legalizing marijuana from 1990 to 2016. %I argue that, when politicians are immovable on an issue, social movements can impact policy change by instead targeting voters, by gaining standing on and through a strategic reframing of the issue. Results from negative binomial regression analyses of state-level discourse in local articles, compared to statements promulgated by marijuana movement organizations in local articles, demonstrate that the marijuana movement significantly influenced the shift towards positive national discourse about marijuana. 
%\end{singlespace}
%\end{abstract}
%\newpage

%\bibpunct[:]{(}{)}{;}{a}{}{,} % new punctuation for these, replace parentheses before and after citealt, and replace citealt with citep



%--------------------------------------------------------------------------------------------------------------------------------------
\section{Introduction}

What accounts for the rate of passage of marijuana legalization in the United States? In the absence of federal legislation, states have become sites for policy change. Yet in 2019, more than a decade after President Barack Obama's Deputy Attorney General, David W. Ogden, released a memorandum recommending that the Department of Justice only prosecute individuals or business that did not comply with state medical marijuana laws \citep{ogden_2009}, most states have not yet legalized. While U.S. Senators and Representatives have advocated for, sponsored, and even support bills for Federal policy change, 

I seek to understand why some states embrace legalization and others do not. I explore the effects of public opinion, political environments, and the marijuana discourse on the rate of passage of legalization. The literature on U.S. policy change has highlighted how political contexts and public opinion have influenced policy \citep{burstein_and_linton_2002,amenta_et_al_2001,elliott_and_amenta_2019}. Although most studies have emphasized the role of public opinion and aspects of the political environment  \citep{burstein_and_linton_2002}, it ignores others. For example, the focus on political context centers, almost exclusively, on the makeup of larger legislative bodies (e.g. Congress or State Assemblies), which overlooks policy change that occurs by alternative means (e.g. public citizen initiative). What is more, these works often ignore the exogenous effects of nearby locales and the impact of discourse about political issues. Because of this, scholars may lack broader understanding of the causal factors associated with policy change and policy diffusion across the United States. 

I ask how public opinion, aspects of the political environment -- other than the presence of allies in government -- and discourse affect the rate of passage of marijuana legislation across the United Staes over time. In so doing, I account for the roles of marijuana public opinion, political competition in elections, a state's history with voting on the marijuana issue, the impact of legalization in nearby locales, as well as the effects of positive marijuana discourse on the passage of legalization across the States. I argue that political context and discursive contexts matter for increasing the rate at which states legalize. In particular, I argue that competitive elections stimulate interest and voter turnout, which increases representation of interests previously excluded from the political system. As such, under competitive elections, marijuana legalization initiatives have better chances of success. Moreover, amenable discourse about political issues creates a context where voters not only are more willing to discuss marijuana, but are more willing to talk about and perceive marijuana in positive ways, creating openings for legalization. 



I use original data to test these hypotheses about policy change. I draw on panel data from the United States between 1990 and 2016 to investigate the factors that influence the rate of passage of marijuana legalization in the U.S. By 2016, eight states did so for recreational use. %Although many states passed recreational legalization through the ballot initiative process during the Obama presidency -- when the administration signaled leniency on enforcement of marijuana laws \citep{ogden_2009}, many states considered legalization prior to 2009.



%Whi of this resulted from mobilization of key activist organizations. National organizations like Drug Policy Alliance, the National Organization for the Reform of Marijuana Laws, and civil rights organizations, such as the ACLU and NAACP worked to support or sponsor marijuana legalization initiatives. 

%28 medicalized

%Besides mobilization by organizations, what other factors have led to marijuana legalization across the United States? This article examines how political opportunity structure, public opinion, and public discourse on marijuana affect the rate of legalization of marijuana from 1990 to 2016. While prior work tends to model these effects in isolation, it is important to account for their varied effects, in tandem, to demonstrate their relative impact on marijuana policy change. 

%Most scholars agree that varying degrees of political opportunity structure, social movements, and public opinion influence policy change \citep{burstein_and_linton_2002}, yet this literature tends to focus on singular causal factors -- few studies include all of these factors measured explicitly across time. In this article, I argue that it is the interaction of these three factors that impact the rate of legalization in the U.S.





%--------------------------------------------------------------------------------------------------------------------------------------
\section{The History of Marijuana Legalization}




In 1930, President Herbert Hoover established the Federal Bureau of Narcotics (FBN) and appointed Harry J. Anslinger as commissioner. According to some scholars, in preparing for inevitable end of prohibition and the upcoming Twenty-first Amendment, Anslinger sought to direct funds previously earmarked for maintaining and enforcing alcohol prohibition towards drugs in America \citep{hari_2015}. The agency's main focus was to prevent the smuggling, flow, distribution, and sale of illicit (hard) drugs, such as opium and heroin in the United States. Yet, opium and heroine had relatively few users. Therefore, to ensure the success of his Bureau, part of Anslinger's plan was to target another drug: cannabis. In order to discourage Americans from using cannabis, and to have a reason for the FBN, Anslinger planned to paint cannabis in a negative light by way of smear campaigns and ``yellow'' journalism \citep{mosher_and_akins_2019,newhart_and_dolphin_2018,rosenthal_and_kubby_1996}. 

Noticing that some Americans were enjoying Mexican and Native-American cannabis, Ansligner worked with William Randolph Hearst, using stories and advertisements in Hearst's newspapers to portray cannabis as the enemy of the people. Hearst, for his part, was on board because he stood to lose economically if American cannabis use expanded. Hearst relied on wood pulp for the manufacturing of his papers, and had his money tied up in wood pulp industries. The expansion of cannabis acceptance, however, was a threat to Hearst's newspaper business because it meant the expansion of hemp, the fiber of the cannabis plant, which could also be used for newspaper manufacturing, but came at a cheaper cost. Hemp, and thus, cannabis, threatened Hearst's fortune. Through a campaign of ``yellow'' journalism, which enabled Anslinger to rebrand the drug with the more Native-sounding name marihuana (or marijuana) instead of cannabis, Anslinger and Hearst could associate the drug with a source or group of people responsible for the drug problem: immigrants, Mexicans, and indigenous ``others.'' Through newspapers, Anslinger and Hearst were able to ``sell'' marijuana as danger -- relying on a fear narrative that argued that only through prohibition could America's children, women, and society be protected. In fact, in 1937, Anslinger was reported in the  \textit{New York Times} as having said the following, ``Primarily we want to protect our young people from a danger which is not apparent to them?''  In addition, Anslinger doubled-down on the race problem -- claiming that marijuana made Blacks believe themselves equal to whites, and that the drug forced minority races into fits of anger, rage, terror, and crimes of brutality. 

Over time, with enough public and legislative support behind him, worked with Congress to put forth the Marihuana Tax Act of 1937, which officially made the possession and sale of marijuana in the United States illegal, allowing only the restricted and sale of price-inflated hemp, which would now be taxed through the purchase of tax stamps enabled verification of the product's legitimacy, but also allowed the federal government to collect revenue from its sale. The problem was, with little enforcement power, the FBN was often unable to prosecute those who broke the law. This problem led to a flurry of legislation to support the enforcement of the Marijuana Tax Act. Over the years, even after Anslinger left the Bureau in 1962, prohibition of marijuana was strengthened and harsher penalties were applied. Between 1952 and 1956, the Boggs Act and the Narcotic Control Act strengthened penalties by instituting mandatory minimum sentences between two and ten years and up to \$20,000 in fines. Yet, in 1970, the Supreme Court found argued that mandatory minimum sentences were unconstitutional. Therefore, as a result, Congress enacted the Controlled Substances Act, which removed mandatory minimums, but in a slight of hand, reclassified marijuana as a Schedule I drug (those assumed to have a high potential for abuse or addiction and with no known medicinal purpose), on par with heroin, LSD, and peyote. Further, in 1973, President Richard Nixon, calling for a ``War on Drugs'' reorganized the FBN into the Drug Enforcement Agency (DEA), and provided increased resources and personnel for enforcement. President Ronald Reagan continued and intensified the drug war, working with Congress to enact the Comprehensive Crime Control Act of 1984 and the Anti-Drug Abuse Act of 1986, which, together, reestablished mandatory minimum sentences, and instituted the three-strikes law, which many have argued is the primary reason for the explosion of the the U.S. prison population \citep{alexander_2010}.

In the early 1970s, states began to push back on the issue of federal marijuana prohibition, using direct democratic processes like the ballot initiative to enact laws of their own. As \citet{newhart_and_dolphin_2018} argue, in many states, the opportunities for legalization took on a peculiar two stage process -- states would enact laws that enabled access to marijuana for medical purposes, and growing support surrounding medicalization would pave the way for or smooth the transition to legalization by way of ballot initiative \citep{kilmer_and_maccoun_2017}. In 1972, for example, California was the first state to take up the medicalization. This ballot initiative was unsuccessful, however, with nearly two-thirds of voters opposing medicalization, which proved costly for medicalization across the U.S. With the exception of a failed legalization initiative in Oregon in 1986 (Measure 5) and a formal criminalization initiative in Alaska in 1990, the marijuana was not seriously taken up until the 1990s, when, in 1996, California again attempted to medicalize. This time, however, a majority of Californians supported medicalization. In short, during the 1990s, medicalization of marijuana took off, with seven states voting to allow for the medical use of cannabis. Importantly, during that time, public opinion on marijuana underwent a rapid positive shift. From 2000 to 2016, eighteen states attempted to make marijuana for medical or recreational use.


\begin{center}
{\renewcommand\normalsize{\scriptsize}
\input{/Users/burrelvannjr/Dropbox/Professional/Research/Projects/dissertation/chapters/ch3.diffusion/paper/tables/table1_.tex}
}
\end{center}

\input{/Users/burrelvannjr/Dropbox/Professional/Research/Projects/dissertation/chapters/ch3.diffusion/paper/figures/figure1.tex}

\input{/Users/burrelvannjr/Dropbox/Professional/Research/Projects/dissertation/chapters/ch3.diffusion/paper/figures/figure2.tex}  



%--------------------------------------------------------------------------------------------------------------------------------------
\section{The Processes Behind Policy Diffusion}


%In addition, the movement has devoted much of it's resources not on protest as a means of changing public definitions of marijuana, but on local and statewide initiatives, gaining media coverage of their issue, and on holding local community-based meetings. 


What shapes the likelihood of policy change? The issue of policy change has been a central theme in political sociology. To understand variation in the adoption of marijuana legalization, I examine the relationship between political contexts% \citep{dobbin_et_al_2011}
, public opinion, and discursive opportunities.


\subsection{\it{Political Context}}

Theories about political contexts center on how aspects of the political environment affect the potential for policy outcomes. Often, these works center on the role of elite allies \citep{amenta_et_al_1994,amenta_2006}, arguing that successful policy change depends on the presence of sympathetic elites or state bureaucrats. For example, recent work has demonstrated the impact of party control -- in the form of Democratic governors and houses of state governments \citep{elliott_and_amenta_2019}. While these studies expand our understanding of policy change, they do not recognize that there may be different pathways to legalization, and that these pathways might affect the probability of legalization. One may ask, does policy change emerging from an absence of elite allies possess characteristics dissimilar to other sorts of policy changes? 




Policy change can result from two mechanisms. In nearly all cases, ``state'' level policy change is the result of (or mediated by) politicians and legislators \citep{amenta_et_al_1994,amenta_2006}. Therefore, these political officials have the final say on what does or does not become official policy. In this formulation the assumption is that, because politicians are principally concerned with either reelection or making good on their campaign promises \citep{elliott_and_amenta_2019,page_and_shapiro_1983,mayhew_1974,downs_1957,stimson_et_al_1995}, they will sponsor or support policies that accord with the interests of their constituents. 

In the second formulation, policy change can occur through direct democratic means, such as citizen initiatives, which renders political officials, or political allies less necessary. This is important for the current study, where changes in policy may require ground-up mobilization by citizens and voters rather than support by political officials. Important are the commonalities between these two mechanisms of policy change (politician-initiated versus voter-initiated), which is that change is more likely when it aligns with the interests of a large proportion of population. Given politicians' interest in fulfilling their campaign promises, they may possess genuine fears about supporting marijuana legalization -- during the ``war on drugs,'' they may support continued prohibition in order to appear tough on crime; seeming less than tough may undermine support from their constituents. Therefore, given the history of prohibition in the United States, and resistance amongst politicians, citizen initiatives provide an opportunity for marijuana legalization, and the presence or absence of allies in government should not be relevant for the rate of policy change. 

To be sure, for policies like prohibition \citep{andrews_and_seguin_2015,gusfield_1963}, same-sex marriage \citep{soule_2004}, and abortion rights \citep{mcveigh_and_diaz_2009}, policy change resulted from ballot initiatives rather than state or federal legislative action. For these cases, there are similarities regarding the expansion of rights and access. As such, it is reasonable to expect that marijuana legalization would employ the same, statewide process.



%In addition to public opinion, political opportunity structures are important for policy change. Scholars typically focus on the movements side of the equation, and argue that political opportunities are dimensions of the political environment that provide incentives for collective action by affecting people's perceptions of success or failure \citep{tarrow_1994}. These theories really center on how the political climate affects the potential for policy outcomes. Another version of this theory centers on the role of elite allies in the policy process. Political mediation theorists \citep{amenta_et_al_1994,amenta_2006} argue that successful policy change, particularly as it results from or is associated with social movement action, depends on the presence of sympathetic elites or state bureaucrats to move policy change forward. In this way, the effects of mobilized groups are mediated by political officials. However, because all states that enacted legalization did so through the initiative process. It is less important to consider the role of elites, relative to the overall political environment.

While the composition of state governments may not be important for marijuana policy change, political ideological contexts are important for identifying places where legalization is more likely. Relatedly, the proportion of voters who either stand to benefit from policy change, or whose position encompasses opening up civil rights or liberties advocate for change. Because Democrats typically believe that marijuana legalization promotes their interests in fairness, access, opportunity and removing barriers to accessing the franchise, I argue that states with larger percentages of Democrats, will be more likely to adopt marijuana legalization \citep{mccammon_et_al_2007,mccammon_et_al_2001,soule_and_olzak_2004}.


In sum, I expect aspects of the political environment to be related to the rate of passage in the states, including measures of Democrats, levels of electoral competition, and prior initiatives discussing the same issue \citep{key_1957,boushey_2016}. Importantly, there may also be external pressures on the diffusion of policy change. Researchers studying political institutions also demonstrate the remarkable trend towards conformity across geographic units, with regard to political and policy change -- with examples ranging from access to the franchise \citep{uggen_and_manza_2002,manza_and_brooks_1999}, extension of benefits \citep{amenta_2006,amenta_et_al_2005} and expansion of civil rights and liberties \citep{andrews_1997}, and importantly, policies. As such, legalization in neighboring states may affect the likelihood that policy change occurs in the focal state. 



\subsection{\it{Public Opinion}}

When incorporated in models of political change, public opinion also matters. According to some scholars, when a majority of the public supports a policy, other political institutional factors such as political context or the influence of interest groups recede in importance \citep{burstein_and_linton_2002,burstein_and_hirsch_2007}. Thus, public opinion also matters for state-level policy change. Public opinion is an important predictor of policy change insofar as it serves as a signal of constituents' preferences. Yet, while public opinion may best serve politicians' voting, it may also predict support for voter initiatives. Therefore, states with majority supportive public opinion on marijuana legalization may be prone to legalizing more quickly than others.


%The insight is that when a majority of the public supports a policy, other political institutional factors (e.g. party in power, etc.) recede in importance \citep{burstein_and_linton_2002,burstein_1999}. This means that politician may (or may not) slavishly follow public opinion of their constituents and propose or support policies that align with their constituents to ensure better odds of reelection. In this formulation, public opinion plays an important role in predicting the likelihood of policy change (mediated through political officials). 





\subsection{\it{Cultural Context}}


Finally, and likely equally as important, is the role of discursive opportunity structures. Discursive opportunity structures are those that increase the saliency of particular cultural forms, beliefs, or ways of talking about issues. A discursive opportunity structure is made up of elements in the cultural environment. This is includes hegemonic discourse or discussions about a particular issue in the broader field of discussion. Much of this work has focused on how the broader discourse about a subject may be critical for political and policy change. This literature speaks to the broader cultural context within which policy change occurs. \citep{mccammon_et_al_2007,mccammon_et_al_2001}. For example, recent work has argued how more coverage, and more positive coverage of controversial issues \citep{amenta_et_al_2019,amenta_et_al_2009} can alter discourse on a topic \citep{bail_2012,ghaziani_and_baldassarri_2011}, which can ultimately impact political outcomes \citep{vasi_et_al_2015}. Therefore, I expect positive discourse on marijuana to be related to the rate of legalization across the U.S. In sum, based on these theories, one can hypothesize that policy initiatives borne out of amenable political or cultural contexts may be more successful others. 



%--------------------------------------------------------------------------------------------------------------------------------------

%--------------------------------------------------------------------------------------------------------------------------------------
\section{Data \& Method}

To assess the relative impacts of public opinion, political contexts, and discursive contexts on the rate of legalization, I draw on longitudinal data from 1990 to 2016 for 49 U.S. states.\footnote{I exclude Alaska for data reliability issues.} %To mitigate the issue of multiple legalization initiatives during the observed years, I use voting data from a state's first initiative, which ensures that the voting data closely correspond with the Census and ACS data used as independent variables. 
States as units of analysis provide comparative leverage to explain variation in the over-time likelihood of legalization because I can compare 49 states across 27 years. %Given that the lived experience of residents in a state may be distinct in different parts of the state (e.g. see \citealt{mcveigh_and_sobolewski_2007}), a county level analysis allows me to account for intrastate heterogeneity that may be associated with views on marijuana legalization. 
State level demographic data come from the 1990 and 2000 U.S. Census and the American Community Survey (ACS) 2005-2009. %Legalization outcomes between 2000 and 2008 are matched with 2000 Census data and votes between 2009 and 2016 are matched with ACS 2005-2009 data. 

The dependent variable, whether or not a state legalized marijuana in a given year, comes from the Secretary of State website for each state.\footnote{These data are also confirmed through Ballotpedia, given that legalization in all states (through 2016) resulted from popular votes via the citizen initiative.} Because my main dependent variable of interest is dummy-coded and longitudinal -- whether or not a state legalized marijuana for recreational use in a given year -- I use event history analysis to estimate the models. I constrain my analysis to legalization between 1990 and 2016 because the first successful effort to legalize marijuana in any capacity (medical or recreational) appeared in California in 1996\footnote{California was the first to attempt medicalization, proposing a similar unsuccessful medicalization initiative in 1972. Given that this was the only case in the 1970s and 1980s, I exclude this from the analysis.} and the most recent election data end in 2016. The key test of my arguments involve relative comparisons between the effects of public opinion, political contexts and discursive opportunities on the rate of legalization across the U.S. Therefore, below, I highlight the various data incorporated to test these arguments. In event history analysis, the results are presented as hazard ratios. The coefficients from the models represent the rate of passage, with significant positive coefficients indicating an increasing effect on the rate of passage (or that legalization is likely to occur faster in the state), whereas significant negative effects indicate a decrease on the rate, or a slowing effect on passage. 

\subsection{\it{Public Opinion}}

I include data from the Roper Center for Public Opinion Research, which covers marijuana public opinion data from various polls from 1988 to 2013. The data come from various sources and years, including marijuana public opinion from 1988 (ABC News Poll), 2001 (Gallup/CNN/USA Today Poll), 2003 (Gallup Poll), 2009 (CBS News Poll), 2010 (60 Minutes/Vanity Fair Poll), 2011 (CBS News/60 Minutes/Vanity Fair Poll), 2012 (USA Today Poll), and 2013 (BS News/60 Minutes/Vanity Fair Poll). Importantly, data between polls are linearly interpolated. 


%might be 8 cases in full data set

%ADD A FOOTNOTE SAYING THAT I ALSO USED A MEASURE OF CHILDSEG, NOT FAMCHILDSEG, AND HAD SIMILAR RESULTS... MOREOVER, CHILDSEG AND FAMCHILDSEG ARE HIGHLY CORRELATED, INDICATING PLACES WITH CHILDREN ARE ALSO PLACES WITH FAMILIES. THEREFORE, I USE THE MOST CONSERVATIVE ESTIMATE OF FAMCHILDSEG

%\begin{center}
%\input{/Users/burrelvannjr/Dropbox/Professional/Research/Projects/dissertation/chapters/ch4.structure-and-politics/paper/figures/figure1.tex}
%\end{center}

\subsection{\it{Political Contexts}}

Political opportunities are those that signal the likelihood or potential for successful policy change. In the case of marijuana, this includes percentage of Democratic voters, political competition, and the number of prior marijuana initiatives. Political partisanship is also associated with support. Indeed, as informed by research at the individual level, compared to Republicans, Democratic voters are more supportive of legalization \citep{rosenthal_and_kubby_1996,caulkins_et_al_2012}. Therefore, I use data from Congressional Quarterly's {\it{America Votes}} to calculate the percentage of voters who voted for the Democratic candidate in the 1988, 2000, and 2008 presidential elections. For 1990, these data come from the 1988 election, and represent the percentage of votes for Michael Dukakis. Data from the 2000 election represent the percentage of votes for Al Gore. Data from the 2008 election represent the percent of votes for Barack Obama. These data are linearly interpolated for intermediate years. 


Secondly, political competition is an important signal of the openings or vulnerability of political systems to demands from advocacy organizations or citizens, as well as to policy change. This would signal the pressures facing political candidates for office. I hypothesize that more competitive elections increase turnout, and thus increase pressures on politicians to conform to a majority of constituents' demands. Therefore, I use the above voting data to construct a measure of political heterogeneity, or the amount of competition that exists in a state, based on Peter Blau's heterogeneity index \citep{blau_1977a}, measured as:

\begin{equation}
1 - \sum_{i = 1}^{k}{P_{i}}^2
\end{equation}
where $P_{i}$ is the proportion of the people voting for party/category, $i$, across $k$ number of total parties. The heterogeneity index can range from 0 to 1, where 0 represents complete homogeneity -- that all voters voted for the same party, and 1 indicates complete heterogeneity -- that voters are more evenly dispersed across parties. The index represents the probability that two members randomly selected from the population of voters will have voted for different Parties. Again, these data are based on the linearly interpolated voting data.

Relatedly, external political contexts may also affect the rate of legalization. Recent work in political science reveals the impact external political opportunities have on internal likelihoods of change. \citet{boushey_2016} for example shows that policy innovations were more likely to diffuse to a focal state if nearby or neighbor states had previously, or simultaneously, enacted similar policies. As such, we know that external political opportunities may increase pressure for compliance, thereby creating internal political opportunities. As such, I include a measure for the proportion of a state's neighbors that legalized marijuana in or before that year. To create this measure, I create a dummy code for each state for the status of recreational marijuana in that year (1 = marijuana legal in this year; 0 = marijuana not legal in this year). Next, for each state, I create a list of each focal state's neighbor states (e.g. the list of states you would enter if you crossed the state line for a focal state). Then, for each focal state, I calculate the proportion of neighbor states that had legalized in that year. For some scholars, this variable would also serve as a measure of regional influence.

A final measure of political context is whether or not (or the amount of times) marijuana has been brought up for a vote in the state. This can serve as a measure of the saliency of the marijuana issue for the general public. As such, I use data from the Secretary of State for each state to calculate the number of times marijuana was previously brought up, in a progressive way, using the ballot initiative (e.g. decriminalization, medicalization, taxation, or recreational legalization). 



\subsection{\it{Cultural Contexts}}

As I have argued, dominant cultural beliefs or discourse about marijuana may influence the likelihood of marijuana in a given year.  Recent research has demonstrated the impact of cultural beliefs or dominant discourse on political outcomes \citep{bail_2012,mccammon_et_al_2007,ghaziani_and_baldassarri_2011}. I use data from the previous chapter to measure positive discourse on marijuana. To compile these data, I rely on the ProQuest newspaper database. I constrain the analysis to 1990 and on because coverage on marijuana was relatively low prior to 1990, and because this time frame immediately followed Reagan's intensified ``War on Drugs'' and ``Just Say No'' campaign. 

To track discursive change, I rely on articles about marijuana in the Proquest database from 1990 to 2016. Because marijuana advocacy organizations may have had an impact on coverage, I separately searched for articles about marijuana in the absence of advocacy organizations, and articles about marijuana that included advocacy organizations. To accomplish this,  I wrote a Python script to identify and download all local articles from Proquest that mention ``marijuana'' between 1990 and 2016.\footnote{This does not include variants of the word marijuana, or the word cannabis} Because national newspapers may be more likely to cover national issues over local issues, I exclude national newspapers, including the \textit{New York Times}, the \textit{Los Angeles Times}, the \textit{Washington Post}, and the \textit{Wall Street Journal}. In addition, I exclude articles that mention at least one of the four main marijuana advocacy organizations. Therefore, I also exclude articles that mention National Organization for the Reform of Marijuana Laws (NORML), Marijuana Policy Project (MPP), Drug Policy Alliance (DPA), and Students for Sensible Drug Policy (SSDP), and their variants. In total, there were 14,163 articles mentioning marijuana. After removing duplicate articles, articles outside of the U.S. or located in the U.S. capitol\footnote{ProQuest sometimes mistakenly identifies non-U.S. articles when only-U.S. articles are specified.}, short articles (e.g. articles with fewer than 100 words), and articles that are not fully searchable,\footnote{Articles with fewer than about 900 words.}, I am left with 10,096 locally-based articles that mention marijuana in some fashion. In addition, I removed articles that come from ``alternative'' or sensationalized newspapers. To figure out whether or not the newspaper was an ``alternative newspaper,'' I searched the websites for each newspaper, removing any newspaper that claimed that it was an alternative newspaper. In sum, I am left with  5,893 articles about marijuana which do not include mention of marijuana advocacy organizations. %498 are positive, 74 have plagiarism


Because marijuana advocacy organizations' discussion of marijuana may be important for discursive change on marijuana, I also include coverage of ``marijuana'' alongside coverage of marijuana advocacy organizations. As such, I wrote a separate Python script to identify and download all articles from Proquest that mention ``marijuana'' and any one of the four largest marijuana advocacy organizations (and the variants of their names) between 1990 and 2016. Therefore, the script was able to capture all coverage of ``marijuana'' coupled with coverage of marijuana advocacy organizations, including the National Organization for the Reform of Marijuana Laws (NORML), Marijuana Policy Project (MPP), Drug Policy Alliance (DPA), and Students for Sensible Drug Policy (SSDP).\footnote{Importantly, I separate these sets of coverage for future empirical work on the impact of organizations on the discursive shift.} In total, there were 1,616 articles mentioning a marijuana movement organization. After cleaning the data set of articles by removing duplicate articles, I am left with 1,150 articles mentioning marijuana advocacy organizations. In addition, after removing and articles coming from alternative news sources, I am left with 787 marijuana organization-related articles. For these articles, I include a dummy code to represent that they include mentions of organizations. In sum, there are 6,680 articles used for the analysis. 



%As such, I use text from non-national print news media across the United States from 1990 to 2016. I focus on local rather than national level discourse in print media given criticism against relying on national media sources (\citealt{earl_et_al_2004})\footnote{I therefore exclude the \textit{New York Times}, \textit{Los Angeles Times}, \textit{Wall Street Journal}, and \textit{Washington Post}.}, and given that local coverage is often more substantive than national coverage. To capture positive marijuana discourse in each state in each year between 1990 and 2016, I use text data from all non-national news articles that, taken from ProQuest, that mention ``marijuana.'' In total, there were 14,163 articles mentioning marijuana. After removing duplicate articles, articles outside of the U.S. or located in the U.S. capitol\footnote{ProQuest sometimes mistakenly identifies non-U.S. articles when only-U.S. articles are specified.}, short articles (e.g. articles with fewer than 100 words), and articles that are not fully searchable,\footnote{Articles with fewer than about 900 words.} I am left with 10,096 locally-based articles that mention marijuana in some fashion. %498 are positive, 74 have plagiarism 
%To code each article as either having positive discourse or not, I categorize each article based on it's polarity.\footnote{To prepare all documents for textual analysis, following the procedure used by \citet{bail_2012}, I use software in R to transform each article into fully-searchable sets of words, and clean the textual data by eliminating excessive words (e.g. stop-words such as numbers, conjunctions, and determiners), and transforming each word into it's stem variant.}, with the assistance of a na\"{i}ve Bayes classifier. The na\"{i}ve Bayes algorithm uses a stock of trained text that has been associated with three types of polarity (positive, neutral, or negative) and classifies each document as one of the three polarities. %and.\footnote{Researcher coding of a random sample of the fully-automated codes revealed an inter-rater reliability, or Krippendorff's alpha, $\alpha = $.} I then dummy code each article as either positive or not.  


%employed civilian population 16 and over.


\subsection{\it{Control Variables}}

To assess the effect of public opinion, political contexts, and discursive contexts on legalization, it is necessary to account for several features of U.S. states that might also be associated with support for legalization. Unless otherwise noted, all variables come from the Census or the American Community Survey. %I include data from the {\it{Association of Religion Data Archives}} (ARDA) to calculate measures of Evangelical Protestants and Catholics as a percentage of the total population in a county. I include measures of religious adherence because opposition to marijuana remains strong among those affiliated with these religious denominations \citep{caulkins_et_al_2012,palamar_2014}. In order to ensure that religious adherence data precede marijuana voting data and the independent variables from the Census, I use data from the 1990 ARDA county file (aggregated to the state) for years between 1990 and 1999, and the 2000 ARDA county file (aggregated to the state) with data from 2000 to 2016, these values are also linearly interpolated for years between each county file. 
Support for legalization initiatives might also depend on population size. I therefore include a measure for the natural log of the total population in a state. I control for the percentage of the population that identifies as Black or Latino, given that these groups exhibit considerable variation with respect to their views on marijuana legalization.\footnote{A March 2010 Pew Research poll showed that Blacks and Hispanics had lower support, respectively 41 percent and 35 percent, for legalization than Whites (42 percent), although in 2013, Blacks showed the strongest support for legalization.} Education is associated with liberal attitudes towards marijuana \citep{pedersen_2009}, and increasing support for marijuana legalization may be attributed, in part, to increases in the size of the college-educated population \citep{rosenthal_and_kubby_1996}. I, therefore, include a measure of the percent of the population aged 25 or older with a bachelor's degree. %Finally, given recent arguments about the economic benefits of legalization \citep{mosher_and_akins_2019}, I include a measure of the percent of the state population that is employed. %Also from the Census, as a proxy for the age of the population\footnote{Because the 2000 Census and 2005-2009 American Community Survey use median age measures that are not comparable, I use the size of the aged population as a proxy.}, I include a variable measuring the percentage of the population that is age 65 or over. Descriptive statistics for these and all other variables included in the regression models are presented in Table 2, below.




%--------------------------------------------------------------------------------------------------------------------------------------
\section{Results}

Table \ref{tab:diffusion} presents event history results for the rate of legalization of marijuana in each state from 1990 to 2016. Model 1 includes the measures of public opinion, political context, including percent Democrat, political competition, history of marijuana initiatives in the state, and the proportion of neighbors legalized. We see that the coefficient for public opinion is non-significant in the presence of political context variables, which is contrary to claims by political scientists about the importance of public opinion over political contexts \citep{burstein_and_linton_2002}. In fact, we see that only the measures for political competition, and percent Democratic voters, and the number of prior initiatives on marijuana, are strong predictors of the rate of legalization. This coefficient indicates that there is a strong increase in the rate of passage in states with increasing competitiveness, with higher numbers of previous marijuana initiatives and with higher percentages of voters support for Democrats. No other, political context variables were significantly related to increasing the rate of passage. 



%The first column of Table \ref{tab:diffusion}  includes only the measure of public opinion. As shown, the coefficient for public opinion has a significant positive effect on the rate of legalization across the U.S, meaning that increases in public support for legalization is related to increases in the hazard rate, or rate of passage for legalization. %This provides support for my claim that public support for marijuana is relevant for understanding the overall growth in positive discourse about marijuana. 

%\input{//Users/burrelvannjr/Dropbox/Professional/Research/Projects/dissertation/chapters/ch3.diffusion/paper/tables/table3.tex}

%\input{//Users/burrelvannjr/Dropbox/Professional/Research/Projects/dissertation/chapters/ch3.diffusion/paper/tables/table4.tex}

\input{//Users/burrelvannjr/Dropbox/Professional/Research/Projects/dissertation/chapters/ch3.diffusion/paper/tables/table4new.tex}

%\input{//Users/burrelvannjr/Dropbox/Professional/Research/Projects/dissertation/chapters/ch3.diffusion/paper/tables/table5.tex}


%\input{/Users/burrelvannjr/Dropbox/Professional/Research/Projects/dissertation/chapters/ch2.movements-and-discourse/paper/tables/table6_08_29.tex}


%\input{/Users/burrelvannjr/Dropbox/Professional/Research/Projects/dissertation/chapters/ch2.movements-and-discourse/paper/tables/table1_11_18.tex}







Model 2 incorporates a measure of discursive contexts. We see that for this model, the measure of positive discourse about marijuana is significantly related to the rate of passage. 
Coefficients, $b$, in event history analysis are interpreted through exponentiation, where $(e^{b} - 1)*100$ gives the expected percent change in the dependent variable that is associated with a one-unit increase in the independent variable, in the presence of controls. For my measure, a one-unit increase in positive discourse about marijuana is associated with a 62.26 percent increase in the rate of passage of legalization.
We also see that the measure for political competition and the measure of percent Democrat remains significantly and positively related to the rate of passage. In addition, states with lower proporotions of neighbor states that legalized were more likely to legalize. No other political context variables have significant impacts on the outcome.

Finally, Model 3 is the full model which includes controls for the percent of state residents with a bachelors degree, the natural log of the total population, percent Black, and percent Latino. We see here that the measure of discourse, or positive discourse about marijuana remains a significant positive predictor of the rate of passage in a state, whereas measures for public opinion, political competition, percent Democrat, and the number of prior marijuana initiatives maintain their significance. Yet, based on the model fit indices, AIC and BIC, which are both measures of variance explained controlling for the number of parameters in the model (e.g. parsimony) -- where the best models explain the most variance with fewer parameters, the best model excludes controls (Model 2).


%--------------------------------------------------------------------------------------------------------------------------------------
\section{Conclusions}

Scholars of social policy often focus on the laggard pace of policy change in the United States. 
Yet, as I have demonstrated, each set of factors uniquely contributes to increases in the rate of legalization, and this rate varies substantially across states, namely as a function of positive discourse about marijuana and political competition. For example, legalization occurred more quickly in places like Nevada, where amenable public opinion, political and discursive contexts were lacking, but also occurred in places like Oregon and Washington, where amenable contexts were present. %%%%%%Bring this all up to the front end

In this article, I account for variation in the rate of legalization by considering how various aspects of political contexts, cultural/discursive contexts, and public opinion, influence the extent to which voters support the legalization of marijuana through citizen initiative. After controlling for numerous other attributes of U.S. states, I still find a strong, statistically significant relationship between discursive opportunities and whether a state legalizes marijuana or not, when it does. As I have argued, this relationship can be explained in terms of strong support amongst voters in places with competitive elections and positive discourse

The current study addresses gaps political sociology and political science by investigating  structural effects on policy change. First, given the longstanding tradition in studies of marijuana legalization to investigate the individual precipitants of support, this work follows a more recent line of inquiry devoted to understanding the structural influences on marijuana legalization, which provides general insights into patterns of support for policy change. Additionally, this work contributes to a growing chorus of scholarship on the consequences of discourse \citep{bail_2012,vasi_et_al_2015}, with a focus on political outcomes. In particular, this research broadens the scope of scholarly study by empirically investigating the impacts of discourse and political opportunities on the pace of policy change. 

In this article, I focused on how structural patterns of relations shape policy change. It is my hope that this work will stimulate research on discursive and political factors that influence policy change on controversial issues.

%\newpage

%\newpage
%--------------------------------------------------------------------------------------------------------------------------------------
%\section{References}}}

%\bibliographystyle{/Users/burrelvannjr/Dropbox/Professional/Research/References/asa_new}
%\renewcommand{\section}[2]{}%
%\setlength{\bibhang}{40pt}%matches the indentation above for references
%\bibliography{/Users/burrelvannjr/Dropbox/Professional/Research/References/library,/Users/burrelvannjr/Dropbox/Professional/Research/References/ext_library}
%\newpage



%--------------------------------------------------------------------------------------------------------------------------------------
%\section{Appendix}


