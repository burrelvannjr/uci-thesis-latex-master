\chapter{The Adoption of Recreational Marijuana Legalization in the United States, 1990--2016}



%\begin{abstract}
%\begin{singlespace}
%What drives the diffusion of drug policy in the United States? This study examines the dramatic increase in the number of states legalizing marijuana from 1990 to 2016. %I argue that, when politicians are immovable on an issue, social movements can impact policy change by instead targeting voters, by gaining standing on and through a strategic reframing of the issue. Results from negative binomial regression analyses of state-level discourse in local articles, compared to statements promulgated by marijuana movement organizations in local articles, demonstrate that the marijuana movement significantly influenced the shift towards positive national discourse about marijuana. 
%\end{singlespace}
%\end{abstract}
%\newpage

%\bibpunct[:]{(}{)}{;}{a}{}{,} % new punctuation for these, replace parentheses before and after citealt, and replace citealt with citep



%--------------------------------------------------------------------------------------------------------------------------------------
\section{Introduction}

What accounts for the rate of passage of marijuana legalization in the United States? In the absence of federal legislation, states have become sites for policy change. Yet more than a decade after the Obama Administration released a memorandum signaling a lax stance on state-level marijuana laws \citep{ogden_2009}, most states have not legalized. Those states that have moved on legalization, however, have done so through the statewide ballot initiative.

In this chapter, I seek to understand why some states embrace legalization and others do not. I explore the effects of political institutional contexts, policy feedback, and discourse on the rate of passage of marijuana legalization across the U.S. Literature in political sociology and political science highlight how political contexts and public opinion influence policy change \citep{burstein_and_linton_2002,amenta_et_al_2001,amenta_and_elliott_2019}. Although most studies have emphasized the role of public opinion and aspects of the political environment  \citep{burstein_and_linton_2002}, it ignores others. For example, the focus on political context centers, almost exclusively, on the makeup of larger legislative bodies (e.g. Congress or State Houses/Assemblies), which overlooks policy change that occurs by alternative means (e.g. public citizen initiative). Because of this, scholars may lack broader understanding of the causal factors associated with policy change and policy diffusion across the United States. 

Understanding the rate of adoption of marijuana legalization is important in its own right because the outcome of policy change has impacts on the lives millions of patients and users \citep{newhart_and_dolphin_2018,miron_2010,mosher_and_akins_2019} as well as those imprisoned for marijuana infractions under the federal system of prohibition \citep{sered_2019,gottschalk_2016,alexander_2010}. 

Importantly, understanding marijuana policy change is important for substantive discussions about progressive policy change and the set of cases to which it belongs. Marijuana legalization is an interesting case of progressive policy that bears some similarity to  contentious and morality-related policies such as abortion liberalization or, to a greater extent, tobacco restrictions, anti-prohibition \citep{gusfield_1963,andrews_and_seguin_2015} and lotteries \citep{pierce_and_miller_1999,berry_and_berry_1990}, which are often opposed for their perceived impacts on youth \citep{beisel_1997}. Marijuana legalization is similar to alcohol prohibition/anti-prohibition and lottery battles in that their perceived negative consequences and costs will only be incurred by those who make use of the policy \citep{mikesell_and_zorn_1986,berry_and_berry_1990}. Marijuana legalization differs, however, from anti-prohibition policy change because of alcohol's longstanding social and economic impact on American society \citep{andrews_and_seguin_2015}. Only now are we beginning to understand the potential economic benefits that could result from marijuana legalization, and public opinion polls indicate increasing marijuana use over time \citep{gallup_2013}. Marijuana legalization thus bears some similarity to taxation policy \citep{amenta_and_halfmann_2000} in that there are perceived economic benefits to policy change, yet longstanding opposition from citizens and powerful interests \citep{amenta_and_elliott_2019}. Finally, marijuana legalization can also be thought of as similar to, but a reversed trend of, tobacco policy. Tobacco policy, much like alcohol policy has been built into the social and economic fabric of the United States. However, in recent years, there have been increasing attempts to restrict tobacco use through increased taxation, removal from retail locations, and elimination in public use spaces -- which may be due to increasingly negative connotations and beliefs attached to smoking. On the contrary, over time, marijuana has become increasingly associated with medicinal uses for patients as well as economic benefits for locales. 

In short, given it's unusual history and evolving understandings of marijuana, legalization as progressive policy presents an intriguing case that is not identical to other policies but that can necessarily inform our understanding of the factors that contribution to the various categories of policy change. Moreover, given the recency of legalization, studying this policy can provide insight into the process of early adoption of policy in the long history of policy change. 

In this chapter, I ask how aspects of the political contexts, previous policy histories, and discourse on the issue affect the rate of passage of marijuana legislation across the American states, over time. In so doing, I account for the roles of marijuana public opinion, political competition in elections, a state's history with voting on the marijuana issue, the impact of legalization in nearby locales, as well as the effects of positive marijuana discourse on the passage of legalization across the States. I argue that political institutional contexts and discursive contexts matter for increasing the rate at which states legalize. In particular, I argue that competitive elections stimulate interest, which increases representation of interests previously excluded from the political system -- increasing legalization's chances of success. Moreover, positive discourse about marijuana creates a context in which voters not only are more willing to discuss marijuana, but may be more likely to perceive marijuana legalization positively and creates opportunities for legalization. 



I use original data to test these hypotheses about policy change. I draw on panel data from states in the U.S. between 1990 and 2016 to investigate the factors that influence the rate of passage of marijuana legalization. By 2016, eight states had legalized marijuana for recreational use. %Although many states passed recreational legalization through the ballot initiative process during the Obama presidency -- when the administration signaled leniency on enforcement of marijuana laws \citep{ogden_2009}, many states considered legalization prior to 2009.



%Whi of this resulted from mobilization of key activist organizations. National organizations like Drug Policy Alliance, the National Organization for the Reform of Marijuana Laws, and civil rights organizations, such as the ACLU and NAACP worked to support or sponsor marijuana legalization initiatives. 

%28 medicalized

%Besides mobilization by organizations, what other factors have led to marijuana legalization across the United States? This article examines how political opportunity structure, public opinion, and public discourse on marijuana affect the rate of legalization of marijuana from 1990 to 2016. While prior work tends to model these effects in isolation, it is important to account for their varied effects, in tandem, to demonstrate their relative impact on marijuana policy change. 

%Most scholars agree that varying degrees of political opportunity structure, social movements, and public opinion influence policy change \citep{burstein_and_linton_2002}, yet this literature tends to focus on singular causal factors -- few studies include all of these factors measured explicitly across time. In this article, I argue that it is the interaction of these three factors that impact the rate of legalization in the U.S.





%--------------------------------------------------------------------------------------------------------------------------------------
\section{A Brief History of Attempts to Legalize Marijuana}

At the hands of President Herbert Hoover and under the direction of the commissioner of the Federal Bureau of Narcotics, Harry J. Anslinger, in 1937, the Marihuana Tax Act officially made the possession and sale of marijuana illegal. Marijuana prohibition was strengthened over time, with increased penalties via the Boggs Act (1952) and the Narcotics Control Act (1956), and, through the Controlled Substances Act (1970), which reclassified marijuana as a Schedule I drug -- those assumed to have a high potential for abuse or addiction and with no known medicinal purpose -- on par with heroin, LSD, and peyote. 


%In 1930, President Herbert Hoover established the Federal Bureau of Narcotics (FBN) and appointed Harry J. Anslinger as commissioner. According to some scholars, in preparing for inevitable end of prohibition and the upcoming Twenty-first Amendment, Anslinger sought to direct funds previously earmarked for maintaining and enforcing alcohol prohibition towards drugs in America \citep{hari_2015}. The agency's main focus was to prevent the smuggling, flow, distribution, and sale of illicit (hard) drugs, such as opium and heroin in the United States. Yet, opium and heroine had relatively few users. Therefore, to ensure the success of his Bureau, part of Anslinger's plan was to target another drug: cannabis. In order to discourage Americans from using cannabis, and to have a reason for the FBN, Anslinger planned to paint cannabis in a negative light by way of smear campaigns and ``yellow'' journalism \citep{mosher_and_akins_2019,newhart_and_dolphin_2018,rosenthal_and_kubby_1996}. 

%Noticing that some Americans were enjoying Mexican and Native-American cannabis, Ansligner worked with William Randolph Hearst, using stories and advertisements in Hearst's newspapers to portray cannabis as the enemy of the people. Hearst, for his part, was on board because he stood to lose economically if American cannabis use expanded. Hearst relied on wood pulp for the manufacturing of his papers, and had his money tied up in wood pulp industries. The expansion of cannabis acceptance, however, was a threat to Hearst's newspaper business because it meant the expansion of hemp, the fiber of the cannabis plant, which could also be used for newspaper manufacturing, but came at a cheaper cost. Hemp, and thus, cannabis, threatened Hearst's fortune. Through a campaign of ``yellow'' journalism, which enabled Anslinger to rebrand the drug with the more Native-sounding name marihuana (or marijuana) instead of cannabis, Anslinger and Hearst could associate the drug with a source or group of people responsible for the drug problem: immigrants, Mexicans, and indigenous ``others.'' Through newspapers, Anslinger and Hearst were able to ``sell'' marijuana as danger -- relying on a fear narrative that argued that only through prohibition could America's children, women, and society be protected. In fact, in 1937, Anslinger was reported in the  \textit{New York Times} as having said the following, ``Primarily we want to protect our young people from a danger which is not apparent to them?''  In addition, Anslinger doubled-down on the race problem -- claiming that marijuana made Blacks believe themselves equal to whites, and that the drug forced minority races into fits of anger, rage, terror, and crimes of brutality. 

%Over time, with enough public and legislative support behind him, worked with Congress to put forth the Marihuana Tax Act of 1937, which officially made the possession and sale of marijuana in the United States illegal, allowing only the restricted and sale of price-inflated hemp, which would now be taxed through the purchase of tax stamps enabled verification of the product's legitimacy, but also allowed the federal government to collect revenue from its sale. The problem was, with little enforcement power, the FBN was often unable to prosecute those who broke the law. This problem led to a flurry of legislation to support the enforcement of the Marijuana Tax Act. Over the years, even after Anslinger left the Bureau in 1962, prohibition of marijuana was strengthened and harsher penalties were applied. Between 1952 and 1956, the Boggs Act and the Narcotic Control Act strengthened penalties by instituting mandatory minimum sentences between two and ten years and up to \$20,000 in fines. Yet, in 1970, the Supreme Court found argued that mandatory minimum sentences were unconstitutional. Therefore, as a result, Congress enacted the Controlled Substances Act, which removed mandatory minimums, but in a slight of hand, reclassified marijuana as a Schedule I drug (those assumed to have a high potential for abuse or addiction and with no known medicinal purpose), on par with heroin, LSD, and peyote. Further, in 1973, President Richard Nixon, calling for a ``War on Drugs'' reorganized the FBN into the Drug Enforcement Agency (DEA), and provided increased resources and personnel for enforcement. President Ronald Reagan continued and intensified the drug war, working with Congress to enact the Comprehensive Crime Control Act of 1984 and the Anti-Drug Abuse Act of 1986, which, together, reestablished mandatory minimum sentences, and instituted the three-strikes law, which many have argued is the primary reason for the explosion of the the U.S. prison population \citep{alexander_2010}.

During the 1970s, states began to push back on the issue of federal marijuana prohibition. Direct democratic processes such as the ballot initiative were used to reverse marijuana prohibition via medicalization and legalization. Yet as \citet{newhart_and_dolphin_2018} argue, marijuana policy change took on a peculiar two stage process -- states would first enact laws that enabled access to marijuana for medical purposes, and growing support for and use of medical marijuana would smooth the transition to legalization \citep{kilmer_and_maccoun_2017}. In 1972, for example, California became the first state to take up the issue of medicalization. Proposition 19, as it was called, was unsuccessful, however -- nearly two-thirds of voters opposed the initiative. 

The failure of Proposition proved costly for future marijuana policy change across the states -- the marijuana issue was not seriously considered again until the 1990s (two exceptions include a failed legalization initiative in Oregon in 1986 known as Measure 5 and a re-criminalization initiative in Alaska in 1990). In 1996, voters in California again attempted to be early adopters of marijuana policy change by placing another medicalization initiative, Proposition 215, on the ballot. This time, however, a majority of Californians supported medicalization. The success of Proposition 215 set off a wave of statewide ballot initiatives that would have allowed for the medical use of marijuana. In fact, during this time, seven states put medical marijuana on the ballot. 



During the 1990s, public opinion on marijuana underwent a positive shift. By the turn of the century, nearly three-quarters of Americans supported marijuana for medical use \citep{gallup_2003,gallup_2001}. But this relationship is recursive -- growing public opinion may have influenced, and been influenced by, medical marijuana ballot initiatives. As some have argued, this positive shift on both public opinion and the number of medicalization initiatives may have paved the way for future medicalization as well as legalization \citep{kilmer_and_maccoun_2017,newhart_and_dolphin_2018,bradford_and_bradford_2017}. In fact, between 2000 and 2016, the number of states attempting to medicalize via the ballot initiative increased to thirteen (up from seven in the 1990s). 

This growth, and the success of medical marijuana initiatives made marijuana policy expansion seem possible. As a case in point, between 2000 and 2016, there was also an increase in the number of states voting on initiatives to legalize marijuana for recreational use. Although no states tried to legalize marijuana from 1990 to 1999, as can be seen in Table \ref{tab:ballot_list}, from 2000 to 2016 twelve states had initiatives on the statewide ballot to do so. In addition, this table, sorted by date, displays the various characteristics of each recreational legalization ballot initiative. These characteristics include whether or not the initiative specified an age restriction, a possession limit (for loose cannabis, for concentrates, and plants), as well as whether or not the initiative also reduced previous penalties attached to marijuana, if the initiative also described or recognized the medical uses of marijuana, if the initiative specified a plan for revenue generation from marijuana (e.g. through taxed sales), whether the initiative would also establish or recognize a separate entity to serve as a regulatory board over marijuana production, distribution, sales, and tax collection in the state, and finally, whether or not the initiative passed. 



%Importantly, during that time, public opinion on marijuana underwent a rapid positive shift. From 2000 to 2016, eighteen states attempted to make marijuana for medical or recreational use.

\input{/Users/burrelvannjr/Dropbox/Professional/Research/Projects/dissertation/chapters/ch3.diffusion/paper/tables/table1_.tex}
%\begin{center}
%{\renewcommand\normalsize{\scriptsize}
%\input{/Users/burrelvannjr/Dropbox/Professional/Research/Projects/dissertation/chapters/ch3.diffusion/paper/tables/table1_.tex}
%}
%\end{center}

Additionally, Figures \ref{fig:state_legal_2000} and \ref{fig:state_legal_2016} below depict the status of marijuana legalization in the United States in 2000 and again in 2016. These two maps reveal the drastic changes in marijuana policy, particularly the spread of legalization. 


\input{/Users/burrelvannjr/Dropbox/Professional/Research/Projects/dissertation/chapters/ch3.diffusion/paper/figures/figure1.tex}

\input{/Users/burrelvannjr/Dropbox/Professional/Research/Projects/dissertation/chapters/ch3.diffusion/paper/figures/figure2.tex}  



%--------------------------------------------------------------------------------------------------------------------------------------
\section{The Processes Behind Progressive Policy Change}


%In addition, the movement has devoted much of it's resources not on protest as a means of changing public definitions of marijuana, but on local and statewide initiatives, gaining media coverage of their issue, and on holding local community-based meetings. 


What shapes the likelihood of progressive policy change? The issue of policy change has been a central theme in political sociology and political science. To understand variation in the adoption of marijuana legalization, I examine the relationship between political contexts% \citep{dobbin_et_al_2011}
, public opinion, and discursive opportunities.


\subsection{\it{Political Institutional Contexts}}

Theories of political context center on how aspects of the institutional political environment affect the potential for policy outcomes. Often, these works focus on the role of elite allies \citep{amenta_et_al_1994,amenta_2006}, and argue that successful policy change depends on the presence of sympathetic political officials or state bureaucrats. 

A specific version of this line of inquiry, institutional politics theory, focuses on the role of internal political processes, such as electoral results and public opinion on policy change. According to the theory, an important driver of progressive policy change is the presence of a left or reform-oriented regime in power \citep{amenta_et_al_2005,korpi_1983}. \citet{amenta_and_elliott_2019} argue that this model is in line with the ``responsible parties'' model \citep{schattschneider_1942} in which an elected party enacts policy to make good on their campaign promises, rather than because of concerns about reelection. The election of majority left-wing parties sends signals to voters and other political officials that progressive policy change is possible. Importantly, when left-wing parties take control of majorities in houses of government, they gain the ability to outvote right-wing political officials. This view is supported by examples such as the ratification of the Equal Rights Amendment \citep{soule_and_olzak_2004,soule_and_king_2006}, the rate of adoption of same-sex marriage bans \citep{soule_2004,mcveigh_and_diaz_2009}, and the expansion of old-age policy \citep{amenta_and_elliott_2019,amenta_et_al_2005}. Recent work has linked the institutional politics model to party control of state-level governments, such as Democratic control of the governorship and state legislative bodies \citep{amenta_and_elliott_2019}. This model is also in line with ``party control'' (aka trifecta) models in political science, which find that Democratic control of executive and legislative bodies increase the odds of progressive political changes \citep{ansolabehere_and_snyder2006,winters_1976,abramowitz_1983,campagna_and_grofman_1990}. 






%For example, recent work has demonstrated the impact of party control -- in the form of Democratic governors and houses of state governments \citep{amenta_and_elliott_2019}. While these studies expand our understanding of policy change, they do not recognize that there may be different pathways to legalization, and that these pathways might affect the probability of legalization. %One may ask, does policy change emerging from an absence of elite allies possess characteristics dissimilar to other sorts of policy changes? 


Institutional politics theory, in short, focuses on the role of political officials on progressive policy change. Because political officials have the final word on what does or does not become official policy, scholars argue that the process of policy change mediated by political officials \citep{amenta_et_al_1994,amenta_2006}. In this formulation, because political officials are concerned with being responsible, they will sponsor or support policies that accord with the interests of their constituents \citep{amenta_and_elliott_2019,page_and_shapiro_1983,mayhew_1974,downs_1957,stimson_et_al_1995}. 

Relatedly, while the composition of state governments may be important for policy change, election results may also reveal ideological contexts that may be more or less fertile ground for progressive change. As such, voters are an important factor in the political institutional contexts. Voters who either stand to benefit from policy change, or whose position reform-oriented or left-leaning may advocate for progressive policy change. Democratic voters, for example, are typically reform-oriented in relation to opportunity-creation as well as limiting barriers to the franchise \citep{amenta_and_poulsen_1996,key_1957,amenta_et_al_2005,hicks_1999,amenta_1998,amenta_et_al_2002}. Indeed, as informed by research at the individual level, compared to Republicans, Democratic voters are more supportive of legalization \citep{rosenthal_and_kubby_1996,caulkins_et_al_2012}. 

Related to pressure on political officials is the amount of competition involved in the election. Research on policy change has demonstrated that competitive elections increase turnout, and thus increase pressures on politicians to conform to a majority of constituents' demands \citep{berry_and_berry_1990,soule_and_olzak_2004,boushey_2016}. Much more than increasing turnout, in competitive elections, the interests of a large number of the members of society are better represented \citep{soule_and_olzak_2004}, especially those who are disadvantaged politically \citep{soule_and_olzak_2004,piven_and_cloward_1977}. In competitive electoral environments, it is likely that the future policies enacted by future political officials will be more liberal and inclusive. 


Another aspect of institutional politics centers on the influence of public opinion \citep{burstein_1998,burstein_2003}. Scholars often argue that politicians will make policy in accordance with public opinion, especially if an issue is salient \citep{pacheco_2012,nicholson-crotty_2009}. Public opinion is an important predictor of policy change insofar as it serves as a signal of constituents' preferences. When an issue is salient and has supportive public opinion, politicians are expected to abandon their previously opposing policy positions to avoid losing reelection \citep{jacobs_and_shapiro_2000,page_and_shapiro_1992}. It is for this reason, scholars argue that when incorporated in models of political change, factors such as the influence of interest groups or even the makeup of government recede in importance in favor of public opinion \citep{burstein_and_linton_2002,burstein_and_hirsh_2007}. 



Yet, institutionally-oriented policy change is but one pathway to policy change. Some issues may be especially risky for politicians to support in particular periods. During the ``War on Drugs,'' when the dominant perception if marijuana centered on danger and crime \citep{caulkins_et_al_2012,rosenthal_and_kubby_1996,alexander_2010}, politicians may have been more fearful of supporting marijuana legalization and therefore supported prohibition in order to appear tough on crime. Supporting legalization, during this period may give the impression that an official is not tough on crime which can undermine support from their constituents. However, the same signals for policy change that result from Democratic control of political institutions and Democratic voter support may also signal to \textit{voters} that the time is right to enact legalization by other means. Given the longstanding opposition to marijuana in the United States, and resistance amongst politicians, direct democratic processes such as citizen initiatives provide an opportunity for marijuana legalization, which renders allies in government less relevant for policy change. This is similar to the effect public opinion has on the influence of political officials on policy change. 


This is important for the current study, where changes in policy can result from citizen mobilization. Many policies have been initiated outside of political institutions, including prohibition \citep{andrews_and_seguin_2015,gusfield_1963}, same-sex marriage \citep{soule_2004}, and abortion rights \citep{mcveigh_and_diaz_2009}, all of which were set off by way of ballot initiatives rather that federal or state-level governmental action. Given these similarities, is reasonable to expect that marijuana legalization was adopted across the states through a similar process.










%In addition to public opinion, political opportunity structures are important for policy change. Scholars typically focus on the movements side of the equation, and argue that political opportunities are dimensions of the political environment that provide incentives for collective action by affecting people's perceptions of success or failure \citep{tarrow_1994}. These theories really center on how the political climate affects the potential for policy outcomes. Another version of this theory centers on the role of elite allies in the policy process. Political mediation theorists \citep{amenta_et_al_1994,amenta_2006} argue that successful policy change, particularly as it results from or is associated with social movement action, depends on the presence of sympathetic elites or state bureaucrats to move policy change forward. In this way, the effects of mobilized groups are mediated by political officials. However, because all states that enacted legalization did so through the initiative process. It is less important to consider the role of elites, relative to the overall political environment.
































%The insight is that when a majority of the public supports a policy, other political institutional factors (e.g. party in power, etc.) recede in importance \citep{burstein_and_linton_2002,burstein_1999}. This means that politician may (or may not) slavishly follow public opinion of their constituents and propose or support policies that align with their constituents to ensure better odds of reelection. In this formulation, public opinion plays an important role in predicting the likelihood of policy change (mediated through political officials). 

\subsection{\it{Policy Feedback}}

Policy feedback theory is concerned with how initial policies influence the likelihood of future policy change. Policy feedback theory holds that the creation of policy can initiate changes that reinforce or undermine the policy \citep{pierson_2000,skocpol_1992}. Policies provide rules, resources, and organization for enforcement, and retrenching spending or expansionary policies imposes direct costs on their beneficiaries \citep{pierson_1996}. Therefore, policymakers often avoid retrenchment because they fear beneficiaries will make them pay electorally \citep{pierson_1996}. Some of the positive feedback mechanisms that reinforce expansion, rather than retrenchment, include the support of organized constituencies, or gains in public opinion on the issue after policy implementation. In this formulation, initial policies on marijuana can lead to beneficiary groups -- people who benefit from the policy. This creates a situation in which it becomes difficult to repeal the programs created by the policy -- because those who are benefitting will likely oppose policy reversals (which are accompanied by their votes). In fact, it becomes more likely that the programs attached to the policy will be expanded. Medicalization of marijuana, therefore, presents an ideal case -- making marijuana legal for medical purposes created a beneficiary group of users which came to be defined as ``patients'' \citep{newhart_and_dolphin_2018}. 


\subsection{\it{External Diffusion Effects}}

Research in political science reveals the impact external environments have on internal likelihoods of policy change. A great deal of the work on policy diffusion has focused on progressive policy change, including taxes \citep{mikesell_and_zorn_1986}, lotteries \citep{berry_and_berry_1990}, and criminal justice policy \citep{boushey_2016}. Research about state tax innovation and lotteries is useful in theorizing about the determinants of marijuana legalization, as factors that create the need for revenue (e.g economic downturn) can prompt the adoption of policies that would generate said revenue. While many policies may be similar in the sense that they generate revenue, they differ in their likelihood of adoption and acceptance by voters \citep{berry_and_berry_1990}. Sales tax policy, for example, encounter resistance from voters because citizen payments are mandatory, whereas policy changes allowing lotteries encounter less resistance, given that participation in lotteries is voluntary, which leads to higher voter support \citep{mikesell_and_zorn_1986}. 


The diffusion literature discusses the aforementioned internal (political institutional context) as well as regional effects (those from nearby localities) on policy change \citep{bradford_and_bradford_2017,berry_and_berry_1990,glick_and_friedland_2014}. Similarly, researchers studying policy change demonstrate the remarkable trend towards conformity across geographic units, including access to the franchise \citep{uggen_and_manza_2002,manza_and_brooks_1999}, extension of benefits \citep{amenta_2006,amenta_et_al_2005} and expansion of civil rights and liberties \citep{andrews_1997}. Based on these bodies of work, internal effects align with the institutional politics theory, yet, scholars in political sociology often ignore the role of regional effects on policy change. As the theory goes, nearby states may enact progressive policies that put pressure on initial states to enact those same policies, and this is particularly the case for revenue-generating policies \citep{bradford_and_bradford_2017,berry_and_berry_1990}. Importantly, there is increased pressure for policy change in states where increasing numbers of neighbors are enacting policy change.  These pressures are related to the type of policy to be enacted as well as internal, political institutional factors \citep{boushey_2016,berry_and_berry_1990}. 

\subsection{\it{Cultural Context}}


Cultural or discursive contexts have impacts on policy change \citep{vasi_et_al_2015}. Discursive opportunity structure \citep{mccammon_et_al_2007,ghaziani_and_baldassarri_2011} is made up of elements in the cultural environment, which includes artifacts \citep{vasi_et_al_2015}, beliefs \citep{mccammon_et_al_2007}, and ways of talking about issues \citep{bail_2012,bail_et_al_2017} that increase their salience. This also includes hegemonic discourse or discussions about a particular issue in broader fields of discussion \citep{bail_2012,mccammon_et_al_2007}. 


Much of this work has focused on how discourse about a subject may be critical for political and policy change. For example, recent work has argued how more coverage, and more positive coverage of controversial issues \citep{amenta_et_al_2019,amenta_et_al_2009} can alter discourse on a topic \citep{bail_2012,ghaziani_and_baldassarri_2011}, which can ultimately impact political outcomes \citep{vasi_et_al_2015}. While this is sometimes related to the framing of issues in public discourse \citep{benford_and_snow_2000,snow_et_al_2007}, it is also relevant to consider the valence of these discussions \citep{vasi_et_al_2015,seguin_2016}. \citet{vasi_et_al_2015}, for example, find an effect of negative discussions about ``fracking'' on increasing the rate at which municipalities enacted \textit{bans} on fracking. Therefore, on the issue of marijuana, the valence (positive or negative) of broader discussions should matter for increasing or decreasing the rate of legalization. 


In chapter 1, I laid out a theoretical model for understanding how changes in morality policy can occur. Morality policies, like marijuana, are those that involve conflicts over what is ``right'' and what is ``wrong'' \citep{mooney_1999}, and can be included as an additional category outside of the traditional distributive-redistributive-regulatory policy set \citep{lowi_1964}. Yet, recent work has demonstrated that morality policies, like many others, cannot be wholly categorized as one of the ideal types, given that they are multidimensional and can possess characteristics of various policy types \citep{meier_2001,spitzer_1995}, and these characteristics vary over time \citep{greenberg_et_al_1977,roh_and_berry_2008,spitzer_1987,steinberger_1980}. 

I have argued that morality policies may have greater chance of success or passage when they adopt characteristics of these traditionalist policies. In particular, for the case of marijuana, I argue that legalization may be more likely when the issue is framed as a ``distributive'' policy. Distributive policies are those that serve to redirect resources, in the form of taxes collected from individuals, toward large-scale public programs. Importantly, as I demonstrated in the previous chapter, revenue-based frames of marijuana came to dominate its coverage. If I am correct in my assumptions, I believe that this revenue-based reframing of marijuana should increase the likelihood of passage for marijuana legalization








\section{Predictions}

Based on the above literature, I propose several arguments. 

Democratic control of state governments may signal that the opportunity is right to enact marijuana legalization. Democratic control of state government should be important for increasing the rate of policy change. However, because marijuana legalization, in nearly all cases, occurred by way of ballot initiative and not through political officials in government, party control should matter less for the rate of policy change. 

Because Democrats are left-leaning or reform-oriented, and reform-oriented parties tend to support progressive policy change, I argue that states with larger percentages of voters supporting Democrats, will be more likely to adopt marijuana legalization.

While public opinion may best serve politicians' voting, it may also predict support for voter initiatives. Therefore, states with majority supportive public opinion on marijuana legalization may be prone to legalizing more quickly than others. Public opinion, however, may track with Democratic voter percents, so may have less effect on rate of policy change. 

Given the dominance of narratives about revenue-generation in marijuana coverage (discussed in the previous chapter), I argue that states with higher numbers of articles related to revenue generation should see quicker rates of adoption. 

The literature on policy feedback suggests that previous policies can have positive effects on the likelihood of future policy. As such, I argue that states with longer histories of ballot initiatives devoted to medical marijuana should have faster rates of policy expansion towards marijuana legalization. 

Diffusion effects should matter for marijuana legalization. However, these effects might matter more for smaller states where interstate travel puts increased pressure for focal states to legalize. As such, and given the size of the states that have legalized, diffusion should matter less for the rate legalization. 


%Therefore, I expect positive discourse on marijuana to be related to the rate of legalization across the U.S. In sum, based on these theories, one can hypothesize that policy initiatives borne out of amenable political or cultural contexts may be more successful others. 


Finally, positive discourse may have a recursive relationship with public opinion, party control, Democratic voters, and policy feedback, such that the presence of any one of these could contribute to increased positive discourse about marijuana, and this positive discourse could also contribute to higher rates of the others. Importantly, however, is that positive discourse should contribute to the rate of legalization. 



%--------------------------------------------------------------------------------------------------------------------------------------

%--------------------------------------------------------------------------------------------------------------------------------------
\section{Data \& Method}

To assess the relative impacts of political institutional contexts, policy feedback, and discursive contexts on the rate of legalization, I draw on longitudinal data from 1990 to 2016 for 49 U.S. states.\footnote{I exclude Alaska for data reliability issues.} %To mitigate the issue of multiple legalization initiatives during the observed years, I use voting data from a state's first initiative, which ensures that the voting data closely correspond with the Census and ACS data used as independent variables. 
States as units of analysis provide comparative leverage to explain variation in the over-time likelihood of legalization because I can compare 49 states across 27 years. %Given that the lived experience of residents in a state may be distinct in different parts of the state (e.g. see \citealt{mcveigh_and_sobolewski_2007}), a county level analysis allows me to account for intrastate heterogeneity that may be associated with views on marijuana legalization. 
State level demographic data come from the 1990 and 2000 U.S. Census and the American Community Survey (ACS) 2005-2009. %Legalization outcomes between 2000 and 2008 are matched with 2000 Census data and votes between 2009 and 2016 are matched with ACS 2005-2009 data. 

The dependent variable, whether or not a state legalized marijuana in a given year, comes from the Secretary of State website for each state.\footnote{These data are also confirmed through Ballotpedia, given that legalization in all states (through 2016) resulted from popular votes via the citizen initiative.} Because my main dependent variable of interest is dummy-coded and longitudinal -- whether or not a state legalized marijuana for recreational use in a given year -- I use event history analysis to estimate the models \citep{box-steffensmeier_and_jones_1997,box-steffensmeier_and_jones_2004}. I constrain my analysis to legalization between 1990 and 2016 because the first successful effort to legalize marijuana in any capacity (medical or recreational) appeared in California in 1996\footnote{California was the first to attempt medicalization, proposing a similar unsuccessful medicalization initiative in 1972. Given that this was the only case in the 1970s and 1980s, I exclude this from the analysis.} and recent election data end in 2016. The key test of my arguments involve relative comparisons between the effects of political institutional contexts, feedback, and discursive opportunities on the rate of legalization across the U.S. Therefore, below, I highlight the data incorporated to test these arguments. In event history analysis, the results are presented as hazard ratios. The coefficients from the models represent the rate of passage, with significant positive coefficients indicating an increasing effect on the rate of passage (or that legalization is likely to occur faster in the state), whereas significant negative effects indicate a decrease on the rate, or a slowing effect on passage. 



%might be 8 cases in full data set

%ADD A FOOTNOTE SAYING THAT I ALSO USED A MEASURE OF CHILDSEG, NOT FAMCHILDSEG, AND HAD SIMILAR RESULTS... MOREOVER, CHILDSEG AND FAMCHILDSEG ARE HIGHLY CORRELATED, INDICATING PLACES WITH CHILDREN ARE ALSO PLACES WITH FAMILIES. THEREFORE, I USE THE MOST CONSERVATIVE ESTIMATE OF FAMCHILDSEG

%\begin{center}
%\input{/Users/burrelvannjr/Dropbox/Professional/Research/Projects/dissertation/chapters/ch4.structure-and-politics/paper/figures/figure1.tex}
%\end{center}

\subsection{\it{Political Institutional Contexts}}

Political contexts are those that signal the likelihood or potential for successful policy change. In the case of marijuana, this includes Democratic party control fo state governments, the percentage of Democratic voters, public opinion on marijuana, and political competition. 

First, I create create a dummy variable for every state and every year between 1990 and 2016 in which Democrats controlled the state government -- the Governorship, and Democratic majorities in the state Senate and state Legislature.\footnote{These data are confirmed using \citep{wiki_2019}. Note that the state government for Nebraska consists only of the Governorship.} 

I draw on data from Congressional Quarterly's {\it{America Votes}} to calculate the percentage of voters who voted for the Democratic candidate in the 1988, 1992, 1996, 2000, 2004, 2008, 2012, and 2016 presidential elections. Because my data are state-years between 1990 and 2016, values fro years between presidential elections are linearly interpolated. For 1990, these data come from the 1988 election, and represent the percentage of votes for Michael Dukakis.\footnote{The data for 1990 are interpolated from 1988 to 1992. Therefore, values for the year 1990 represent a linear trend for the number and percent of votes for the Democratic candidate between the 1988 and 1992.} For all other presidential elections, the data represent the percent of the vote for the Democratic candidate, including Bill Clinton (1992 and 1996), Al Gore (2000), John Kerry (2004) Barack Obama (2008 and 2012) and Hillary Clinton (2016). 

I include data from the Roper Center for Public Opinion Research, which covers marijuana public opinion data from various polls from 1988 to 2016. I follow \citep{weakliem_and_biggert_1999} and aggregate individual responses to obtain state-level measures of support. The data come from various sources and years, including marijuana public opinion from 1988 (ABC News Poll), 2001 (Gallup/CNN/USA Today Poll), 2003 (Gallup Poll), 2009 (CBS News Poll), 2010 (60 Minutes/Vanity Fair Poll), 2011 (CBS News/60 Minutes/Vanity Fair Poll), 2012 (USA Today Poll), 2013 (CBS News/60 Minutes/Vanity Fair Poll) and 2016 (CBS News Poll). Importantly, data between polls are linearly interpolated. 


Political competition is an important signal of the openings or vulnerability of political systems and political officials to demands from citizens, as well as to policy change. I use the above voting data to construct a measure of political heterogeneity, or the amount of competition that exists in a state in a given presidential election, based on Peter Blau's heterogeneity index \citep{blau_1977a}, measured as:

\begin{equation}
1 - \sum_{i = 1}^{k}{P_{i}}^2
\end{equation}
where $P_{i}$ is the proportion of the people voting for party/category, $i$, across $k$ number of total parties. The heterogeneity index can range from 0 to 1, where 0 represents complete homogeneity -- that all voters voted for the same party, and 1 indicates complete heterogeneity -- that voters are more evenly dispersed across parties. The index represents the probability that two members randomly selected from the population of voters will have voted for different parties. Again, because these data are based on presidential election data from Congressional Quarterly's {\it{America Votes}}, values are linearly interpolated between election years.

\subsection{\it{External Diffusion Effects}}

External political contexts may also affect the rate of legalization. \citet{boushey_2016}shows that policy innovations were more likely to diffuse to a focal state if nearby or neighbor states had previously, or simultaneously, enacted similar policies. External contexts increase pressure for internal compliance. As such, I include a measure for the proportion of a state's neighbors that legalized marijuana in or before that year. To create this measure, I create a dummy code for each state for the status of recreational marijuana in that year (1 = marijuana legal in this year; 0 = marijuana not legal in this year). Next, for each state, I create a list of each focal state's neighbor states (e.g. the list of states you would enter if you crossed the state line for a focal state). Then, for each focal state, I calculate the proportion of neighbor states that had legalized in that year.\footnote{Rather than using a measure of whether a neighboring state had legalized or not, I use the proportion of neighbors as a measure of increasing influence, as is the case for many scholars studying policy change \citep{boushey_2016,key_1949}.} For some scholars, this variable would also serve as a measure of regional influence.


\subsection{\it{Policy Feedback}}

As a measure of policy feedback theory, which states that initial policy change creates opportunities for future policy expansion and change, I focus on medical marijuana initiatives in a state. This can serve as a measure of the saliency of the marijuana issue for the general public. As such, I use data from the Secretary of State for each state to calculate the number of times medical marijuana was previously on the statewide ballot.


\subsection{\it{Cultural Contexts}}

As I have argued, dominant cultural beliefs or discourse about marijuana may influence the likelihood of marijuana in a given year. I use data from the previous chapter to measure positive discourse on marijuana. To recap these data, I searched the ProQuest newspaper database for mentions of ``marijuana'' between 1990 and 2016. I constrain the analysis to 1990 and on because coverage on marijuana was relatively low prior to 1990, and because this time frame immediately followed Reagan's intensified ``War on Drugs'' and ``Just Say No'' campaign. 

Because marijuana advocacy organizations may have had an impact on coverage, I separately searched for articles about marijuana in the absence of advocacy organizations, and articles about marijuana that included advocacy organizations. To accomplish this,  I wrote a Python script to identify and download all local articles from Proquest that mention ``marijuana'' between 1990 and 2016.\footnote{This does not include marijuana alternatives such as ``cannabis'' or ``hemp.''} Because national newspapers may be more likely to cover national issues over local issues \citep{earl_et_al_2004} and local stories may matter more for local audiences \citep{amenta_et_al_2012}, I exclude national newspapers, including the \textit{New York Times}, the \textit{Los Angeles Times}, the \textit{Washington Post}, and the \textit{Wall Street Journal}. In addition, I exclude articles that mention at least one of the four main marijuana advocacy organizations. Therefore, I also exclude articles that mention National Organization for the Reform of Marijuana Laws (NORML), Marijuana Policy Project (MPP), Drug Policy Alliance (DPA), and Students for Sensible Drug Policy (SSDP), and their variants. In total, there were 14,163 articles mentioning marijuana. After removing duplicate articles, articles outside of the U.S. or located in the U.S. capitol\footnote{ProQuest sometimes mistakenly identifies non-U.S. articles when only-U.S. articles are specified.}, short articles (e.g. articles with fewer than 100 words), and articles that are not fully searchable,\footnote{Articles with fewer than about 900 words.}, I am left with 10,096 locally-based articles that mention marijuana in some fashion. In addition, I removed articles that come from ``alternative'' or sensationalized newspapers. To figure out whether or not the newspaper was an ``alternative newspaper,'' I searched the websites for each newspaper, removing any newspaper that claimed that it was an alternative newspaper. In sum, I am left with  5,893 articles about marijuana which do not include mention of marijuana advocacy organizations. %498 are positive, 74 have plagiarism


Because marijuana advocacy organizations' discussion of marijuana may be important for discursive change on marijuana, I also include coverage of ``marijuana'' alongside coverage of marijuana advocacy organizations. As such, I wrote a separate Python script to identify and download all articles from Proquest that mention ``marijuana'' and any one of the four largest marijuana advocacy organizations (and the variants of their names) between 1990 and 2016. Therefore, the script was able to capture all coverage of ``marijuana'' coupled with coverage of marijuana advocacy organizations, including the National Organization for the Reform of Marijuana Laws (NORML), Marijuana Policy Project (MPP), Drug Policy Alliance (DPA), and Students for Sensible Drug Policy (SSDP).\footnote{Importantly, I separate these sets of coverage for future empirical work on the impact of organizations on the discursive shift.} In total, there were 1,616 articles mentioning a marijuana movement organization. After cleaning the data set of articles by removing duplicate articles, I am left with 1,150 articles mentioning marijuana advocacy organizations. In addition, after removing and articles coming from alternative news sources, I am left with 787 marijuana organization-related articles. For these articles, I include a dummy code to represent that they include mentions of organizations. In sum, there are 6,680 articles used for the analysis. 


From the previous chapter, I use data on positive coverage of marijuana over time. I code each article within each set with the assistance of a na\"{i}ve Bayes classifier in \textsf{R}'s \texttt{sentiment} package \citep{jurka_2012}. The na\"{i}ve Bayes algorithm uses a stock of trained text that has been associated with three types of polarity (positive, neutral, or negative), and categorizes each document as one of these polarities.\footnote{The stock of trained text comes from Janyce Wiebe's subjectivity lexicon \citep{wilson_et_al_2005}, which can be found at: \url{https://mpqa.cs.pitt.edu/lexicons/subj_lexicon/}.} The algorithm compares the word stems in each article to word stems in these dictionaries and classifies each word in the article as negative, neutral, or negative. Next, the algorithm calculates the log likelihood that a given article is positive or negative based on a score: below `1' is negative, `1' is neutral, and above `1' is positive polarity.  In sum, there are 357 positively coded articles about marijuana between 1990 and 2016. As can be seen in Figure \ref{fig:pos_plot} below, there has been an increase in the amount of positive attention to marijuana in newspapers from 1990 to 2016.

In addition, using the coding scheme from the previous chapter, I code each article as having the presence or absence of discussions about revenue-generation. To do so, I rely on a keyword search for all articles containing stems of words related to revenue generation. This technique is drawn from recent literature on automated text analysis from the burgeoning field of computational social science \citep{bail_2016,dimaggio_2015}. 

To identify the frames, I rely on keywords to select whether frames are absent or present in coverage of marijuana. In Table \ref{tab:rev_table} below, I outline the search terms used for identifying these frames. In Table \ref{tab:rev_table}, ``+'' represents the logical operator ``OR'' and ``*'' represents the logical operator ``AND.''

\input{/Users/burrelvannjr/Dropbox/Professional/Research/Projects/dissertation/chapters/ch3.diffusion/paper/tables/table1new.tex}


\input{/Users/burrelvannjr/Dropbox/Professional/Research/Projects/dissertation/chapters/ch3.diffusion/paper/figures/figure3.tex}  


Because each article is geocoded, and has a publication date, I link each article's valence to the state-years within which they were published. This means that I aggregate the valence of each article to the state-year, to arrive at a measure of the number of positive articles about marijuana in a given state and a given year. As seen in Table \ref{tab:state_pos} below, the 357 positively coded articles are distributed across 26 states between 1990 and 2016.  

\input{//Users/burrelvannjr/Dropbox/Professional/Research/Projects/dissertation/chapters/ch3.diffusion/paper/tables/table_state.tex}




%As such, I use text from non-national print news media across the United States from 1990 to 2016. I focus on local rather than national level discourse in print media given criticism against relying on national media sources (\citealt{earl_et_al_2004})\footnote{I therefore exclude the \textit{New York Times}, \textit{Los Angeles Times}, \textit{Wall Street Journal}, and \textit{Washington Post}.}, and given that local coverage is often more substantive than national coverage. To capture positive marijuana discourse in each state in each year between 1990 and 2016, I use text data from all non-national news articles that, taken from ProQuest, that mention ``marijuana.'' In total, there were 14,163 articles mentioning marijuana. After removing duplicate articles, articles outside of the U.S. or located in the U.S. capitol\footnote{ProQuest sometimes mistakenly identifies non-U.S. articles when only-U.S. articles are specified.}, short articles (e.g. articles with fewer than 100 words), and articles that are not fully searchable,\footnote{Articles with fewer than about 900 words.} I am left with 10,096 locally-based articles that mention marijuana in some fashion. %498 are positive, 74 have plagiarism 
%To code each article as either having positive discourse or not, I categorize each article based on it's polarity.\footnote{To prepare all documents for textual analysis, following the procedure used by \citet{bail_2012}, I use software in R to transform each article into fully-searchable sets of words, and clean the textual data by eliminating excessive words (e.g. stop-words such as numbers, conjunctions, and determiners), and transforming each word into it's stem variant.}, with the assistance of a na\"{i}ve Bayes classifier. The na\"{i}ve Bayes algorithm uses a stock of trained text that has been associated with three types of polarity (positive, neutral, or negative) and classifies each document as one of the three polarities. %and.\footnote{Researcher coding of a random sample of the fully-automated codes revealed an inter-rater reliability, or Krippendorff's alpha, $\alpha = $.} I then dummy code each article as either positive or not.  


%employed civilian population 16 and over.


\subsection{\it{Control Variables}}

To assess the effect of political institutional contexts, policy feedback, and discourse on legalization, it is necessary to account for several features of U.S. states that might also be associated with the rate of adoption of legalization. All control variables come from the Census or the American Community Survey. %I include data from the {\it{Association of Religion Data Archives}} (ARDA) to calculate measures of Evangelical Protestants and Catholics as a percentage of the total population in a county. I include measures of religious adherence because opposition to marijuana remains strong among those affiliated with these religious denominations \citep{caulkins_et_al_2012,palamar_2014}. In order to ensure that religious adherence data precede marijuana voting data and the independent variables from the Census, I use data from the 1990 ARDA county file (aggregated to the state) for years between 1990 and 1999, and the 2000 ARDA county file (aggregated to the state) with data from 2000 to 2016, these values are also linearly interpolated for years between each county file. 
Support for legalization initiatives might also depend on population size \citep{soule_and_olzak_2004,soule_and_king_2006,boushey_2016}. I therefore include a measure for the natural log of the total population in a state. I control for the percentage of the population that identifies as Black or Latino, given that these groups exhibit considerable variation with respect to their views on marijuana legalization.\footnote{A March 2010 Pew Research poll showed that Blacks and Hispanics had lower support, respectively 41 percent and 35 percent, for legalization than Whites (42 percent), although in 2013, Blacks showed the strongest support for legalization.} Education is associated with liberal attitudes towards marijuana \citep{pedersen_2009}, and increasing support for marijuana legalization may be attributed, in part, to increases in the size of the college-educated population \citep{rosenthal_and_kubby_1996}. I, therefore, include a measure of the percent of the population aged 25 or older with a bachelor's degree. %Finally, given recent arguments about the economic benefits of legalization \citep{mosher_and_akins_2019}, I include a measure of the percent of the state population that is employed. %Also from the Census, as a proxy for the age of the population\footnote{Because the 2000 Census and 2005-2009 American Community Survey use median age measures that are not comparable, I use the size of the aged population as a proxy.}, I include a variable measuring the percentage of the population that is age 65 or over. Descriptive statistics for these and all other variables included in the regression models are presented in Table 2, below.

Table \ref{tab:descriptives} below describes the variables in the data set. In addition, Table \ref{tab:cor_matrix} provides a correlation matrix between all variables. 


\input{//Users/burrelvannjr/Dropbox/Professional/Research/Projects/dissertation/chapters/ch3.diffusion/paper/tables/descriptives.tex}

%--------------------------------------------------------------------------------------------------------------------------------------
\section{Results}

Table \ref{tab:diffusion} presents event history results for the rate of legalization of marijuana in each state from 1990 to 2016. Coefficients, $b$, in event history analysis are interpreted through exponentiation, where $(e^{b} - 1)*100$ gives the expected percent change in the hazard rate of the dependent variable that is associated with a one-unit increase in the independent variable, in the presence of controls. 

Model 1 includes the measures of political institutional context, such as Democratic party control of state governments (a measure of Democratic control of the Governorship and state legislative bodies), percent Democrat, political competition, and marijuana public opinion, as well as the diffusion measure (the proportion of neighboring states that had legalized by that year), and the measure of policy feedback (the number of medicalization initiatives in that state up to and including that year). We see that state Democratic control is unrelated to the rate of legalization in a state. On the other hand, percent of Democratic voters in a state in a given year is significantly related to the outcome, such that a one-unit (one-percent) increase in percent of the state voting for Democrats in a given year is related to a 55 percent increase in hazard rate of legalization. In addition, the measure of political competition is significant, and positively related to the hazard rate, meaning that in contexts of high political competition, legalization is adopted more quickly. The coefficient for public opinion is significant in the presence of other political context, policy feedback, and diffusion variables. The policy feedback measure is significant, and positively related to the hazard rate -- a one unit increase in the number of medicalization initiatives in a given state in a given year (across the states prior history) increases the rate at which legalization is adopted. Finally, for Model 1, the diffusion measure (proportion of neighbors that have legalized) is unrelated to the outcome. 





%The first column of Table \ref{tab:diffusion}  includes only the measure of public opinion. As shown, the coefficient for public opinion has a significant positive effect on the rate of legalization across the U.S, meaning that increases in public support for legalization is related to increases in the hazard rate, or rate of passage for legalization. %This provides support for my claim that public support for marijuana is relevant for understanding the overall growth in positive discourse about marijuana. 

%\input{//Users/burrelvannjr/Dropbox/Professional/Research/Projects/dissertation/chapters/ch3.diffusion/paper/tables/table3.tex}

%\input{//Users/burrelvannjr/Dropbox/Professional/Research/Projects/dissertation/chapters/ch3.diffusion/paper/tables/table4.tex}

\input{//Users/burrelvannjr/Dropbox/Professional/Research/Projects/dissertation/chapters/ch3.diffusion/paper/tables/table4new1.tex}

%\input{//Users/burrelvannjr/Dropbox/Professional/Research/Projects/dissertation/chapters/ch3.diffusion/paper/tables/table5.tex}


%\input{/Users/burrelvannjr/Dropbox/Professional/Research/Projects/dissertation/chapters/ch2.movements-and-discourse/paper/tables/table6_08_29.tex}


%\input{/Users/burrelvannjr/Dropbox/Professional/Research/Projects/dissertation/chapters/ch2.movements-and-discourse/paper/tables/table1_11_18.tex}







Model 2 incorporates a measure of discursive contexts: positive discourse. We see that for this model, the measure of positive discourse about marijuana is significantly related to the rate of passage. For my measure, a one-unit increase in positive discourse (an increase of one positively categorized article about marijuana in a given state, in a given year) is associated with a 62 percent increase in the rate of passage of legalization. We also see that the measure for public opinion has dropped to non-significance, which is contrary to claims by political scientists about the importance of public opinion over political contexts \citep{burstein_and_linton_2002}. In addition, measure of percent Democratic voters, political competition, and the number of prior medicalization initiatives remain significantly and positively related to the rate of passage. No other political context variables have significant impacts on the outcome.


Model 3 adds the measure for revenue frames or discourse about marijuana. We see that for this model, the measure of revenue discourse is significantly related to the rate of passage, which makes the measure of positive discourse drop to nonsignificance. For my measure, a one-unit increase in revenue discourse (an increase of one article about marijuana in a given state, in a given year, that included discussion of revenue) is associated with a 12 percent increase in the rate of passage of legalization. This finding lends support to my theory that, as marijuana, as a morality issue, became increasingly associated with characteristics similar to ``distributive'' policies (e.g. discussions of revenue generation), there was an increasing likelihood that voters would support change in that policy. Beyond these findings, we see that all other variables maintain their relationship with the outcome. 


Finally, Model 4 is the full model which includes controls for the natural log of the total population, the percent of state residents with a bachelors degree, percent Black, and percent Latino. We see here that the measure of revenue discourse about marijuana remains a significant positive predictor of the rate of passage in a state, whereas measures for political competition, percent Democrat, and the number of prior marijuana initiatives maintain their significance. Yet, based on the model fit indices, AIC and BIC, which are both measures of variance explained controlling for the number of parameters in the model (e.g. parsimony) -- where the best models explain the most variance with fewer parameters, the best model excludes controls (Model 2).


%--------------------------------------------------------------------------------------------------------------------------------------
\section{Conclusions}

Scholars of social policy often focus on why the U.S. has been a laggard in its adoption of policy change.  Yet, in this chapter, I focus on the factors that increase the rate of policy change on marijuana legalization. I accomplish this by appraising various arguments about the likelihood of policy change, including those that focus on the role of political institutional contexts, the impact of policy feedback, and the impact of discourse or cultural contexts. As I have demonstrated, each set of factors uniquely contributes to increases in the rate of legalization, and this rate varies substantially across states. Importantly, I find that some degree of political institutional contexts, policy feedback, and discourse all contributed to increasing the rate of legalization across the United States from 1990 to 2016. 

For political institutional factors, I find that Democratic voting and political competition mattered, which are related to the distribution of voters' interests being represented during elections -- particularly those previously excluded from the political system \citep{soule_and_olzak_2004}. Importantly, this chapter highlighted variation in the rate of adoption across states. For example, legalization occurred more quickly in places like Nevada, where amenable public opinion, political and discursive contexts were lacking, but also occurred in places like Oregon and Washington, where amenable institutional contexts were present. %%%%%%Bring this all up to the front end

Beyond political institutional contexts, I find that policy feedback mechanisms have a critical role -- legalization occurred more rapidly in states that had increasing numbers of ballot initiatives devoted to marijuana medicalization. This indicates that for future contentious policies, and those in which users would have to pay (e.g. lotteries versus sales tax), it may be necessary to take a sort of ``foot-in-the-door'' approach \citep{cialdini_1984,freedman_and_fraser_1966}. In this process, a version of the policy is put forth in ways that establish and support a beneficiary group (e.g. patients), which opens the door for future policy expansion. 

Finally, I also find that marijuana's growing association with revenue discourse about contributed to increasing the rate at which states legalized. This finding, as well as the trend described in the previous chapter, help theorize the ways in which changes in morality policies is possible. Although marijuana legalization is a progressive policy, it can still be categorized as a morality policy. Yet, with so few users of the drug, what accounts for the rapid rate of passage for legalization. I argue that the adoption of legalization, or change in marijuana policy was, in part, the result of a long-time shift in the discussion of marijuana -- reframing the issue away from morality discussions to better align with characteristics of distributive policies through narratives that centered on the revenue-generative benefits of change. The presence of this discourse (e.g. these articles) in these states (across time) was associated with increasing the rate of adoption. Although revenue discourse may have occurred in states already experiencing supportive public opinion, or in Democratically controlled states (both in terms of government and voters), revenue discourse did aid in speeding up the process. 

In this chapter, I account for variation in the rate of legalization by considering how various aspects of state environments, influence the extent to which states adopt legalization of marijuana through citizen initiatives. After controlling for numerous other attributes of U.S. states, I still find a strong, statistically significant relationship between discursive opportunities, political contexts, and policy histories and whether a state legalizes marijuana or not, when it does. 

The current study addresses gaps political sociology and political science by investigating  political institutional, policy, and discursive effects on policy change. First, given the longstanding tradition in studies of marijuana legalization to investigate the individual precipitants of support, this work follows a more recent line of inquiry devoted to understanding the contextual influences on marijuana legalization, which provides general insights into patterns of support for policy change. Additionally, this work contributes to a growing chorus of scholarship on the consequences of discourse \citep{bail_2012,vasi_et_al_2015}, with a focus on political outcomes. In particular, this research broadens the scope of scholarly study by empirically investigating the impacts of discourse and political opportunities on the pace of policy change. 

%In this article, I focused on how structural patterns of relations shape policy change. It is my hope that this work will stimulate research on discursive and political factors that influence policy change on controversial issues.

%\newpage

%\newpage
%--------------------------------------------------------------------------------------------------------------------------------------
%\section{References}}}

%\bibliographystyle{/Users/burrelvannjr/Dropbox/Professional/Research/References/asa_new}
%\renewcommand{\section}[2]{}%
%\setlength{\bibhang}{40pt}%matches the indentation above for references
%\bibliography{/Users/burrelvannjr/Dropbox/Professional/Research/References/library,/Users/burrelvannjr/Dropbox/Professional/Research/References/ext_library}
%\newpage



%--------------------------------------------------------------------------------------------------------------------------------------
%\section{Appendix}

\input{//Users/burrelvannjr/Dropbox/Professional/Research/Projects/dissertation/chapters/ch3.diffusion/paper/tables/cor_matrix.tex}
