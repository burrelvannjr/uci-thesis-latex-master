\chapter{Parental Segregation, Marijuana Legalization, and Concerns over the Mobility of Children}


%\thanks{%This article is currently under review. 
%Please do not cite without author's permission.%I am grateful to Edwin Amenta, David Meyer, Rory McVeigh, Su Yang, Kraig Beyerlein, Bryant Crubaugh, Justin Van Ness, Melissa Warstadt and the Social Movements/Social Justice Workshop at the University of California, Irvine for their thoughtful comments on earlier versions of this article and to Neal Caren and the four anonymous reviewers for their useful feedback. An earlier version of this paper was presented at the 2015 annual meeting of the American Sociological Association. This research was made possible thanks to Architects and by generous funding by the Ford Foundation Predoctoral Fellowship Program and the Center for the Study of Democracy at the University of California, Irvine.
%\newline}
%\end{singlespace}}
%\author[]{Burrel Vann Jr\thanks{Burrel Vann Jr is a Ph.D. candidate in the Department of Sociology at the University of California, Irvine. Direct correspondence to: Burrel Vann Jr, Department of Sociology, University of California, Irvine, 3151 Social Science Plaza A, Irvine, CA 92697. Email: bvann@uci.edu}}
%\doublespacing
%\date{}
%\maketitle



%\begin{abstract}
%\begin{singlespace}
%From 2000 to 2012, recreational marijuana legalization hit the ballot in several states, ultimately passing in just two. 
%In this article, I examine the sources of variation in support for marijuana legalization and show that support is strong in communities characterized by the segregation of residents with children. I argue that the segregation of parents from nonparents results in perceptions of marijuana as non-threatening to children's economic mobility. The analysis also shows that the effects of parental segregation are particularly strong in communities with high prospects for economic mobility, where there are fewer perceived threats to children's mobility, as signified by low income inequality and high occupational differentiation. \newline

%{\color{red} Focus on families, children, and contact to explain how people form opinions about marijuana.}


%, less likely for community residents to develop a stake in protecting children from the perceived negative effects of marijuana. 
%\end{singlespace}
%\end{abstract}
%\newpage



\bibpunct[:]{(}{)}{;}{a}{}{,} % new punctuation for these, replace parentheses before and after citealt, and replace citealt with citep


%--------------------------------------------------------------------------------------------------------------------------------------
\section{Introduction}

%You are analyzing votes on a ballot initiative...which reflects individuals' attitudes pertaining to legalization.  

%In the intro section, take some time setting up a substantive and academic puzzle.  The substantive puzzle would have to do with varying opinion on a controversial issue.  That kind of thing is easy to write about.  It seems like the academic puzzle has to do with how patterns of social contact can shape attitudes on controversial issues.  While we may think it is just a matter of conservative versus liberal orientations (e.g., we would expect Republican counties to be strongly opposed and Democratic counties to be more favorable), you want to show that there is more to it than that.  The threat thing (but not the opportunity thing) seems to be fine.  What is it that makes legal marijuana seem threatening or ill-advised in some contexts, but not in others.

%If it is just which end you want to emphasize when writing about it, it makes sense to write about how segregation contributes to support.

%You are trying account for why some counties were supportive and others were not. it is just opposite sides of the same coin.  You are talking about how the segregation of families with children shapes both support and opposition (High segregation, support is more likely---low segregation, opposition is more likely).

Americans have become increasingly accepting of marijuana legalization \citep{caulkins_et_al_2012,rosenthal_and_kubby_1996,gallup_2013,pew_2013}. According to Gallup data, as recently as 2000 fewer than one third of Americans believed that that marijuana should be made legal. By 2013, the percentage of people supporting legalization had reached 58 percent. In light of these trends, in 2009, President Barack Obama's Deputy Attorney General, David W. Ogden, released a memorandum recommending that the Department of Justice only prosecute individuals or business that did not comply with state medical marijuana laws \citep{ogden_2009}. Although marijuana remained illegal at the federal level, this signaled a potential shift in marijuana policy priorities of the Obama Administration and was followed by a dramatic increase in the number of states placing medical and recreational marijuana on ballot initiatives. From 2000 to 2016, ballot initiatives in eleven states had initiatives to legalize the recreational use of marijuana. As Table \ref{tab:support} demonstrates, statewide ballot initiatives exhibited considerable variation in community level support for legalization -- from 17.18 percent in Kiowa County, Colorado to 73.73 percent in San Miguel County, Colorado. In addition to reflecting rapidly changing public opinion on a controversial political issue, the outcome of these votes has led to eight states legalizing marijuana for recreational use. 





%Despite longstanding federal prohibition, %and longstanding opposition, a majority of Americans have come to support marijuana legalization \citep{gallup_2013,pew_2013}. %Indicative of this shift, voters in many states sought to overturn marijuana prohibition by way of statewide ballot initiatives.
%between 2000 and 2012, legalization was put to a public vote in seven states. In Colorado, for example, a 2006 ballot initiative that would have made marijuana legal for personal use was defeated. Just six years later, however, a similar initiative garnered a majority of the vote. 
%Although the outcome of these votes reflect rapidly changing public opinion (see Figures 1 and 2), state-level voting patterns obscure considerable variation in how communities have addressed the marijuana issue. %Closer examination of voting data at the county level over this period show that support for legalization ranged from 17 percent in Kiowa County, Colorado to a high of 79 percent in San Miguel County, Colorado. %numbers of states continues to grow.

Why are some communities in the United States more supportive of marijuana legalization? Voting data for marijuana ballot initiatives provide a unique opportunity to examine the sources of support for legalization. While legalization is important in its own right because it involves conflicts over civil liberties, the outcome of the conflict also has implications for millions of Americans who rely on the drug as medicine and those who choose to use it recreationally. Supporters of legalization seek increased and safe access, while understanding that legalization could contribute to increased use \citep{ingraham_2017}. Generally, these data provide an opportunity to examine how structural features of local contexts shape aggregate support on a controversial issue. The fight over marijuana's legality is a battle over meanings of morality, medicine, and addiction. These meanings are the product of social construction processes, developed through everyday social interaction, and are dependent upon the social contexts in which people are embedded. The ways in which various meanings associated with marijuana resonate with individuals should depend, to a great extent, on the patterns of social relations across local contexts. In this paper, I identify structural features of local contexts that affect the extent to which legal marijuana is perceived as non-threatening to the community, and lead to collective support for marijuana legalization. 




Perceptions about marijuana are rooted in concerns about life outcomes, and are, to some extent, shaped by family structure. Parents, in particular, constitute a large source of opposition to marijuana legalization out of concern for the wellbeing of children \citep{elder_and_greene_2019,newhart_and_dolphin_2018,caulkins_et_al_2012,rosenthal_and_kubby_1996}. Many parents may oppose legalization because they believe marijuana threatens children's ability to lead successful lives \citep{elder_and_greene_2019,newhart_and_dolphin_2018,mosher_and_akins_2019,lynskey_and_hall_2000,kandel_et_al_1986,lifrak_et_al_1997,fergusson_et_al_2002,kandel_2002}. Indeed, while a majority of Americans may view marijuana relatively harmless, many may also believe that children should not be exposed to marijuana out of fear that the drug will hinder a child's life outcomes. I argue that support for legalization should be strong where residents with children are spatially segregated from those who are childfree, not just composition effects of large portions of residents that are childfree, and that these effects should be particularly strong in communities with high prospects for mobility. The absence of children in local contexts resulting from higher proportions of the childfree or the segregation of parents from nonparents, I propose, leads many community residents to view legalization as less threatening to the community as a whole. In communities where residents view marijuana as non-threatening, high prospects for economic mobility can contribute to a general sense that legal marijuana would do little to disrupt the social fabric of the community. 



%%% this is the argument from RORY
%To date, a sizable portion of Americans support legalization because of the perception that it is relatively harmless, yet most would not believe it wise for children to be exposed to marijuana. Therefore, while public opinion in favor of legalization has grown, when larger portions of the community do not have children, or when community residents with children are spatially segregated from those without children, marijuana may be perceived as less threatening to the community as a whole. Support for legalization should be high under these conditions because 1) residents with children can expect that others will be reluctant to use marijuana around children, and marijuana use in distant communities is of little concern and 2) residents without children, especially when spatially segregated from residents with children, may be more inclined to support legalization without regard thinking that marijuana may be perceived as harmful.

%Therefore, while many American's still hold negative opinions about marijuana, they might become more concerned with the potential ``harm'' associated with marijuana use for children where higher percentages of children reside in their local environment. 



%\input{/Users/burrelvannjr/Dropbox/Professional/Research/Projects/dissertation/chapters/ch4.structure-and-politics/paper/figures/figure1.tex}

%\input{/Users/burrelvannjr/Dropbox/Professional/Research/Projects/dissertation/chapters/ch4.structure-and-politics/paper/figures/figure2.tex}


\section{A Brief History of Legalization Across the U.S.}

In 1930, President Herbert Hoover created the Federal Bureau of Narcotics and appointed Harry J. Anslinger as commissioner. The agency was responsible for preventing the smuggling, flow, distribution, and sale of illicit drugs in the United States \citep{hari_2015,mosher_and_akins_2019,newhart_and_dolphin_2018,rosenthal_and_kubby_1996}. 

%Noticing that some Americans were enjoying Mexican and Native-American cannabis, Ansligner worked with William Randolph Hearst, using stories and advertisements in Hearst's newspapers to portray cannabis as the enemy of the people. Hearst, for his part, was on board because he stood to lose economically if American cannabis use expanded. Hearst relied on wood pulp for the manufacturing of his papers, and had his money tied up in wood pulp industries. The expansion of cannabis acceptance, however, was a threat to Hearst's newspaper business because it meant the expansion of hemp, the fiber of the cannabis plant, which could also be used for newspaper manufacturing, but came at a cheaper cost. Hemp, and thus, cannabis, threatened Hearst's fortune. Through a campaign of ``yellow'' journalism, which enabled Anslinger to rebrand the drug with the more Native-sounding name marihuana (or marijuana) instead of cannabis, Anslinger and Hearst could associate the drug with a source or group of people responsible for the drug problem: immigrants, Mexicans, and indigenous ``others.'' Through newspapers, Anslinger and Hearst were able to ``sell'' marijuana as danger -- relying on a fear narrative that argued that only through prohibition could America's children, women, and society be protected. In fact, in 1937, Anslinger was reported in the  \textit{New York Times} as having said the following, ``Primarily we want to protect our young people from a danger which is not apparent to them?''  In addition, Anslinger doubled-down on the race problem -- claiming that marijuana made Blacks believe themselves equal to whites, and that the drug forced minority races into fits of anger, rage, terror, and crimes of brutality. 

Anslinger advocated for marijuana prohibition, which ultimately led to the creation of the Marihuana Tax Act of 1937 \citep{newhart_and_dolphin_2018}. The Act officially made the possession and sale of marijuana in the United States illegal, and only allowed for the restricted sale of hemp, via tax stamps, which enabled the federal government to collect revenue. 

Between 1952 and 1956, the Boggs Act and the Narcotic Control Act strengthened penalties associated with the Marijuana Tax Act by instituting mandatory minimum sentences (between two and ten years) and fines up to \$20,000. When, in 1970, the Supreme Court found mandatory minimum sentences unconstitutional, Congress responded with the Controlled Substances Act, which reclassified marijuana as a Schedule I drug -- those assumed to have a high potential for abuse or addiction and with no known medicinal purpose -- which elevated the risk of marijuana use to that of heroin, LSD, and peyote \citep{newhart_and_dolphin_2018}. In 1973, President Richard Nixon, calling for a ``War on Drugs'' created the Drug Enforcement Agency, while President Ronald Reagan intensified the drug war, supporting the Comprehensive Crime Control Act of 1984 and the Anti-Drug Abuse Act of 1986, which reestablished mandatory minimum sentences, and instituted the three-strikes law \citep{alexander_2010}.

During this time, especially in the 1970s, states began to contest federal marijuana prohibition, using the ballot initiative to enact fewer restrictions on marijuana. As \citet{newhart_and_dolphin_2018} argue, in many states, legalization took on a peculiar two stage process -- states would enact laws that enabled access to marijuana for medical purposes, which would pave the way for legalization \citep{newhart_and_dolphin_2018,kilmer_and_maccoun_2017}. In 1972, for example, California was the first state to take up the medicalization. While this ballot initiative was unsuccessful, with nearly two-thirds of voters opposing medicalization, and setting the stage for failed medicalization attempts over in the 1970s and 1980s, in the 1990s, the tide began to turn -- in 1996, a majority of Californians supported medicalization. The ballot initiative became a means of policy change on marijuana. In fact, between 2000 and 2016, eighteen statewide ballots included initiatives that would have made marijuana legal for recreational use. As seen in Table \ref{tab:state_initiatives} and Figure \ref{fig:support} below, there was substantial variation in county level support for legalization. 



 



{\renewcommand\normalsize{\footnotesize}%makes table font smaller to fit on the page
%\normalsize

%\input{/Users/burrelvannjr/Dropbox/Professional/Research/Projects/dissertation/chapters/ch4.structure-and-politics/paper/tables/table1_.tex}

\input{/Users/burrelvannjr/Dropbox/Professional/Research/Projects/dissertation/chapters/ch4.structure-and-politics/paper/tables/table1_.tex}
}

\input{/Users/burrelvannjr/Dropbox/Professional/Research/Projects/dissertation/chapters/ch4.structure-and-politics/paper/figures/figure1_.tex}

%{\renewcommand\normalsize{\footnotesize}%makes table font smaller to fit on the page
%\normalsize

%\input{/Users/burrelvannjr/Dropbox/Professional/Research/Projects/dissertation/chapters/ch4.structure-and-politics/paper/tables/table1_.tex}
%}
%\vspace{-25pt}

%{\color{red}First, take time to establish that views on marijuana are likely to be shaped by family structure. (e.g. while many people may think that marijuana is inherently bad, they may become more concerned with the ``badness'' of marijuana when they think about the prospect of their own kids smoking. [this will let reviewers know that parenting is significant for my research.]}

%In this article, I shed light on marijuana policy change in the United States by considering whether the residential segregation of households without children (parent households) may facilitate support for marijuana legalization. I identify several structural features of communities that shape the extent to which marijuana is perceived as non-threatening to the community. I argue that the lack of contact between those with and those without children should promote support for legalization, and that these should be particularly strong in communities characterized by high prospects for economic mobility. 






%--------------------------------------------------------------------------------------------------------------------------------------




\section{Policy Positions and the Role of Communities}

%inclusive 

Research at the individual level shows that Americans' support for political issues centers on ideology. For example, support for liberalization policy issues like same-sex marriage or abortion rights is high amongst the politically liberal, college-educated, religiously unaffiliated, and younger populations \citep{baunach_2012,zucker_1999,pew_2017_ssm,pew_2017_ab}. A similar trend exists for  marijuana \citep{caulkins_et_al_2012,newhart_and_dolphin_2018,schnabel_and_sevell_2017,rosenthal_and_kubby_1996}. Support for legalization is highest amongst non-religious, never-married, younger, liberal, and college educated individuals \citep{schnabel_and_sevell_2017,pew_2015,elder_and_greene_2019,eagly_et_al_2004,newhart_and_dolphin_2018,rosenthal_and_kubby_1996}. %At a general level, many supporters of legalization view marijuana as a civil liberty issue -- that people should have the freedom to use marijuana for medicine or recreation without fear of retribution \citep{newhart_and_dolphin_2018,schnabel_and_sevell_2017,rosenthal_and_kubby_1996}. 
Importantly, however, individual level research also shows that positions on legalization are associated with parenthood -- parents often exhibit the stronger opposition to legalization than nonparents \citep{elder_and_greene_2019,mosher_and_akins_2019,newhart_and_dolphin_2018,caulkins_et_al_2012,rosenthal_and_kubby_1996}. According to recent work on social role theory, the the transition to parenthood, and the concerns the role of ``parent'' entails, can lead parents to develop conservative positions on political issues, generally, and on marijuana specifically \citep{elder_and_greene_2019}. %Concerns about how legal marijuana might impact their family, marriage, children, and other social relationships can lead parents to oppose marijuana legalization. 
Liberalization, thus, triggers concerns about the family. In particular, liberalization on social policy incites fear about children's likelihood of economic mobility and ability to reproduce their family's class and status \citep{beisel_1997,gusfield_1963,eskridge_and_speadle_2006}. Therefore, the moral stigma many parents attach to marijuana stems from fears about perceived negative consequences for their children and their families \citep{elder_and_greene_2019,lynskey_and_hall_2000,kandel_et_al_1986,lifrak_et_al_1997,fergusson_et_al_2002,kandel_2002,rosenthal_and_kubby_1996}. Similar to what \citet[5]{beisel_1997} has argued, parents may believe that exposure will render children ``unfit for desirable jobs and social positions,'' they will ``not be hired, or will be excluded from desired social circles,'' and will be ``excluded from the social networks vital to their future success'' \citep[199]{beisel_1997}. Therefore, parents develop a personal stake in protecting their children from behaviors that could hinder chances at leading successful lives \citep{beisel_1997,gusfield_1963}. 


Positions on policy issues do not form in a vacuum, however. There is a rich sociological literature on the role of social structure or contexts to opinion and attitude formation \citep{blau_1977a,blau_1977b,mcveigh_and_sobolewski_2007,mcveigh_and_diaz_2009}. That is to say that the formation of opinions is the result of social interactions and social construction processes in the contexts within which people are embedded \citep{mcveigh_and_sobolewski_2007,mcveigh_and_diaz_2009}. Social structure can impose constraints on social ties, interactions, and thus, the information people use to form opinions about political issues. For this reason, I am interested in the \textit{places} that foster aggregate support for marijuana legalization. To be sure, the composition and of these communities is important, such that support for legalization will be clustered in communities characterized by larger proportions of residents without children. And because nonparents may develop different stakes in the legalization debate, marijuana may be perceived as non-threatening in these communities. 

The structure of communities also relates to the distribution of people with various attributes within a community, and this distribution has important implications for the ways in which people gain information and form opinions about particular issues  \citep{blau_1977b}, above and beyond the simple size of the group sharing those attributes. Work in this vein has highlighted the role of segregation for attitudes and political outcomes \citep{olzak_et_al_1994,andrews_and_seguin_2015}. For this reason, I believe that, beyond the size of the population of nonparents in communities being relevant for support for legalization, the segregation of parents from nonparents structures and hinders contact between these groups. Because people often consider the potential influence on children when forming attitudes about marijuana, segregation has implications for constraining information available for opinion formation amongst each group, making it increasingly difficult for members of different groups to understand or develop a stake in the other's concerns \citep{herek_and_capitanio_1996}. Therefore, the segregation of parents from nonparents can influence aggregate patterns of support or opposition to legalization. Below, I outline the ways in which the ways in which parental segregation across local contexts results in support for legalization. 





%Yet, support for legalization is on the rise \citep{pew_2013,gallup_2013}. Given the aforementioned literature, nonparents may exhibit strong support for legalization. Indeed, literature has demonstrated that support is high amongst individuals without children \citep{newhart_and_dolphin_2018,caulkins_et_al_2012,rosenthal_and_kubby_1996}. This may result from nonparents developing different stakes in the legalization debate -- stakes other than the protection of children and family economic or class status. Specifically, I expect that support for legalization will be clustered in communities characterized by an absence of children. Thus, the structure of local communities shapes the extent to which marijuana is perceived as non-threatening.

%While not all nonparents share similar beliefs about marijuana, I argue that perceptions about the freedom to use marijuana (rather than beliefs about protecting children from harm) will resonate in communities where children are absent. Put another way, support for legalization should be high where nonparents predominate (e.g. where a large portion of the population is childfree). Perhaps more importantly, social structure can impose constraints on both social ties and the information people use to form opinions about political issues \citep{blau_1977b,blau_1977a,mcveigh_and_sobolewski_2007}. Segregation, for example, can decrease contact opportunities between those with distinct viewpoints, making it increasingly difficult for members of different groups to understand or develop a stake in the other's concerns \citep{herek_and_capitanio_1996}. This is why, in the case of marijuana legalization, I believe the spatial segregation of parents from nonparents also plays a role in the formation of perceptions about marijuana being nonthreatening to the community. 





%{\color{red}Talk instead about cross-sectional variation in parenting (e.g. across places), rather than temporal trends in parenting.}
%Dig into literature on changes in family structure.  E.g., demographic trends show that people are having fewer kids and they are having kids later in life.  Dig into literature on how parenting can sharply alter life trajectories, etc.  Dig into the literature on how the nature of parenting has changed over time (e.g., we used to let kids roam freely.  Now ---especially for the upper and middle class, children's lives are heavily structured and supervised).


%The dynamics of parenting in the United States have changed over the past century. For women, this is characterized by declining birthrates \citep{rindfuss_et_al_1996} and delayed childbearing \citep{martin_2000,bianchi_and_spain_1986}. This trend is, in part, the result of women postponing childbirth to meet the demands of growing educational and occupational opportunities \citep{rindfuss_et_al_1980,bianchi_and_spain_1986}, many of whom have opted for smaller families \citep{waite_and_stolzenberg_1976}. Increasingly tolerance toward premarital sex coupled with the growing use of effective contraception \citep{robinson_et_al_1982,taffel_1977} have contributed to an increase in the average age at first marriage as well as delayed childbearing \citep{wilkie_1981}. What is more, the percentage of Americans remaining voluntarily childless has steadily increased \citep{pew_2010,tunalilar_2016,bachu_1996,blake_1974}. In fact, recent Census data show that children reside in fewer than half of American households, reflecting a decline in the importance Americans place on becoming a parent \citep{tunalilar_2016}.




%\section{Trends in Legalization}}}
 
%Research at the individual level shows that support for recreational marijuana legalization remains highest amongst college educated, unmarried, and politically liberal individuals \citep{caulkins_et_al_2012}. Supporters view marijuana use as a civil liberty and believe that people should be allowed to use the substance without fear of retribution \citep{rosenthal_and_kubby_1996}. Perhaps unsurprisingly, support for legalization is correlated adherence to childfree lifestyles \citep{caulkins_et_al_2012}. Nonparents may feel that prohibition, under the guise of keeping children safe, is unwarranted. Indeed, the moral stigma associated with marijuana use stems from fears about the perceived negative consequences for children -- a belief that does not resonate with nonparents. Moreover, nonparents may also feel that legalization is consistent with their policy preferences. Nonparents, who have had higher turnout rates and greater support for liberal issues relative to parents over the last fifty years \citep{teixeira_2002,hewlett_and_west_1998}, may feel that legal marijuana is an issue that reflects progressive ideals. 

%For parents, marijuana exposure means their children could be exposed to danger, and negative life outcomes \citep{lynskey_and_hall_2000,kandel_et_al_1986,lifrak_et_al_1997,fergusson_et_al_2002,kandel_2002}. Yet parents' fears about legal marijuana extend beyond child endangerment and harm. Many parents believe legal marijuana undermines traditional family values and will spark many of society's ills \citep{rosenthal_and_kubby_1996}. Parental opposition is rooted in the perception that marijuana use by children threatens the reputation of their family and their children's ability to lead successful lives \citep{beisel_1997}. As \citet[5]{beisel_1997} argues, for example, parents believe that exposure will render children ``unfit for desirable jobs and social positions'' and they will ``not be hired, or will be excluded from desired social circles, because their habits, reputations, or appearance will make them seem untrustworthy.'' In addition, caring for children can impose costs on political participation, leading a trend towards parental nonparticipation since 1950 \citep{teixeira_2002}, and we know from social movement research that being biographically unavailable -- ``the presence of personal constraints, such as full-time employment, marriage, and family responsibilities'' \citep[70]{mcadam_1986} -- can decrease the likelihood of political participation. The absence of such constraints makes participation more likely, especially when concerning progressive politics.

%Many Americans understand parents' opposition to legalization. Nonparents' knowledge of these fears is evident in the fact that many believe that legalization initiatives could result in shrinking potentially dangerous illicit drug markets and, therefore, safer communities. To be sure, nonparents understand parents' fear that drug use may set children down a deleterious path. But similar to support for comprehensive sex education, abortion rights, and needle exchange programs, many nonparents believe that legalization provides and effective means of harm reduction by way of regulatory control. Parents, however, have a personal stake in the marijuana issue: drug use can impede a child's ability to reproduce their parents' economic privilege and class position (see \citealt{beisel_1997}. Politically, nonparents may view parents as inhabiting different social worlds and that their perception of marijuana as immoral is irrational. For this reason, I expect support for state-level regulation of legal marijuana, rather than prohibition, to be strong where a large proportion of the population is childfree. The structuring of segregated communities imposes constraints on the formation of social ties between groups, which makes it less likely for members of different groups to understand or develop a stake in the other's concerns (see \citealt{herek_and_capitanio_1996}. These beliefs are the result of being embedded in social networks that reinforce voters' convictions. I expect that this reinforcement is the result of living in segregated communities, which can promote aggregate patterns of support. 





%--------------------------------------------------------------------------------------------------------------------------------------


%\section{The Consequences of Segregation}}}
%Much of the research on the consequences of segregation focuses on its impacts on life outcomes, such as restricted access to quality education, employment \citep{massey_and_denton_1993,wilson_1987}, or health information \citep{brewster_et_al_1993}, and poorer health \citep{wilson_1991,biello_et_al_2012}. %hogan_et_al_1995
%Scholars have only recently begun to focus on segregation's impact on politics. Recent work in the conflict tradition demonstrates that segregation facilitates political activism because majority group members feel threatened that minority group members will be unjustly given access to jobs or social programs designed to redress economic, racial, class, or gender-based inequality \citep{mcveigh_et_al_2014a,andrews_and_seguin_2015,mcveigh_and_sobolewski_2007}. As the logic goes, the spatial concentration of individuals with similar backgrounds and social statuses allows for the resonance of beliefs about, in the case of racial segregation, minorities' poor social position and their undeservingness of distributive justice, or in the case of occupational sex segregation, traditional gender roles and the inappropriateness of women in the workplace. Intergroup contact is critical to the story. \citet{mcveigh_and_sobolewski_2007}, for example, find that reactive political action in the form of Republican voting is strongest where segregation is vulnerable to penetration -- where the size of the minority population is greatest -- and similarly, \citet{andrews_and_seguin_2015} find that voting on controversial prohibition policy was strongest in segregated communities closest to minority groups -- nearest the boundary between all-majority and all-minority communities. A similar effect is identified by Olzak and colleagues, showing that increases in whites' exposure to African Americans and decreases in the residential isolation of whites was responsible for increases in anti-civil rights political activity \citep{olzak_et_al_1994}. Although these works identify reactive political action as the consequence of segregation, taken together, the mechanism underlying the link between segregation and political outcomes is increased contact between in-group and out-group that signals threats to segregation. 

%Concurrent work has also explored alternatives to threat and competition arguments, such as, the political consequences that manifest when intergroup contact is limited? \citet{galbraith_and_hale_2008} show that segregation also matters for progressive politics. At high levels of income segregation, where high incomes are concentrated amongst a small and particularly isolated minority of people, households of unequal incomes are so separated that their residents do not frequently encounter one another. High segregation ensures a large proportion of low-income voters with an interest in redistributive policies or progressive candidates. Therefore, as I outline below, limited exposure between disparate groups increases political action geared towards progressive politics. I argue that this trend holds true in the case of marijuana legalization. 


\section{Parental Segregation and the Protection of Child Mobility Prospects}

%{\color{red}Talk about research that discusses CLASS differences.}

%for the theorizing, deal with Beisel's argument in more depth about how sociologists have not given enough attention to the importance that parents place on keeping their children from being exposed to things that may adversely affected their mobility prospects.  Read the recent ASR article by Ann Owen that shows how residential segregation by social class in recent decades has been completely driven by parents with children jockeying to live in neighborhoods with the best school districts.  Think about Luker's argument about the status values attached to parenting.

Residential segregation within communities is a manifestation of structural barriers that restrict access and choices individuals make about where to live -- a sorting process has implications for the structure of communities. Research has demonstrated that life cycle changes such as marriage, new career opportunities, and childbearing increase the odds of residential relocation \citep{clark_et_al_1994,rossi_and_shlay_1982,rossi_1980,davanzo_1976,clark_and_davies-withers_1999,dieleman_et_al_1995}. In fact, recent research shows that a majority of American parents reported relocating to new communities with a predominance of other families with children, and that they believed these areas were better places to raise children \citep{taylor_et_al_2008,clark_and_davies-withers_1999,rossi_and_shlay_1982}. Yet, only recently have scholars given attention to how family residential relocation is influenced by a desire to protect children's economic mobility prospects \citep{owens_2016,beisel_1997}. For many parents, family-predominant environments can provide a sense of solidarity, by increasing the likelihood of interaction with others who share similar lifestyles and perhaps goals for their families. These environments may also provide access to social and cultural capital necessary for maintaining and reproducing family privilege. Because many parents want their children to do as well as or better than they have in life, to increase their children's odds of economic mobility, and thus family privilege they may engage of strategies of individual or family-level segregation, such as the construction of private educational institutions \citep{andrews_2002,nevin_and_bills_1976} or relocating to areas with a predominance of families \citep{owens_2016}, which can result in aggregate level residential segregation of parents and their families from nonparents. 



Residential parental segregation results in a spatial concentration of parents which facilitates interaction between people of similar household compositions, and potentially, similar beliefs about the role of the family, and the consequences of legal drugs for their children. These relationships emerge and are maintained by local institutions including schools, clubs, and social events \citep{beisel_1997}. Residential segregation also limits opportunities for contact, communication, and interaction between individuals with distinct household compositions (see \citep{blau_1977b,blau_1977a}. In these segregated environments, nonparents are less likely to come into contact with parents, and thus, less likely to develop a similar stake in protecting children's (1) mobility prospects or (2) potential for reproducing family privilege. When people with children are integrated into the community, it can contribute to cultural understandings of marijuana as harmful, a perception that is different from those that may develop in places where few children reside. Contexts with high degrees of parental segregation can be conducive to the development and resonance of attitudes about marijuana's impact on children's future mobility.

%Those who do not have kids of their own, or have limited contact with parents and their children, are free to use marijuana without the perception of harming children, and develop less of a stake in the protection of children's mobility. 

\subsection{\it{Parental Segregation and Politics}}

The segregation of parents can shape political behavior \citep{beisel_1997}. Recent work shows that segregation's political impact extends into voting, particularly where exposure to threats or outgroups is increasingly prevalent (see \citealt{andrews_and_seguin_2015}). Concern for children's ability to reproduce class status is a source of anxiety, one that encourages opposition to various morality-based policy issues, including abortion, anti-prohibition, obscenity, gambling \citep{owens_2016,schnabel_and_sevell_2017,luker_1984,gusfield_1963,beisel_1997}. Not only will communities characterized by a predominance of nonparents exhibit higher support for marijuana legalization, but where nonparents are spatially segregated from parents, the lack of interaction can increase the likelihood that nonparent residents will 1) not develop a stake in protecting children's mobility prospects and 2) view marijuana as non-threatening. Thus, in these communities, nonparents are free to embrace marijuana legalization, given that they are not regularly exposed to children. Similarly, in segregated spaces, parents feel less threat of having their children exposed to marijuana and, therefore, may be less opposed to legalization than they might otherwise be. These two patterns, in segregated communities, should result in aggregate patterns of support.


%Moreover, in segregated communities, the concerns many parents have about the protection of children may not resonate with the larger population. As a result, the size of the nonparent population, and the spatial segregation of parents from nonparents should be related to support for legalization. 



Support for legalization should be high under these conditions because 1) residents with children can expect that others will be reluctant to use marijuana around children, and marijuana use in distant communities is of little concern and 2) residents without children, especially when spatially segregated from residents with children, may be more inclined to support legalization without regard thinking that marijuana may be perceived as harmful.


\subsection{\it{Parental Segregation and Community Mobility}}


Worldviews that position marijuana as non-threatening might ring true in communities where actual prospects for economic mobility exist, which further strengthens support for legalization. Because many parents are often concerned with their children's prospects for mobility, and many nonparents are not concerned with these prospects, actual prospects for mobility in the local context should diminish parents' concerns about legal marijuana's perceived negative consequences for their children's life outcomes. Scholars of economic opportunity have shown that both low inequality as well as high diversification in positions of employment signal economic expansion and opportunity \citep{blau_and_duncan_1967,moore_1966}. Given this trend, residents in areas of high economic opportunity could be especially resistant to claims about the threats of legal marijuana to the child (and therefore, family) mobility. Therefore, I expect support for marijuana legalization to be especially strong in areas of low income inequality and areas with high levels of diversification in occupations. 
%Indeed, threats are an important part of the story. Where potential for exposure to threats exist, opposition to legalization should increase. For this reason, I also consider how parental segregation may combine with other community attributes, particularly the impermanence of the community, to strengthen perceptions of legal marijuana as threatening. One factor to consider is the likelihood of residential turnover. While relatively stable communities allow community residents to reasonably predict daily occurrences, demographic changes disrupt quotidian activities \citep{snow_et_al_1998}. Residential turnover, through the availability of housing, could undermine overall support for legalization in segregated communities, given that residents can neither predict who will move into their segregated communities, nor the values new residents hold. A community with a greater potential for turnover might signal change in the orientation of the community and undermine beliefs that legal marijuana poses little threat. To capture the housing availability, I consider the percent of housing units available for rent. In addition, other community attributes introduce doubt about whether or not neighbors will remain in the community. Given that unmarried are the most likely to relocate \citep{taylor_et_al_2008}, I consider how the size of the unmarried population in a community.



%This literature suggests that counties vary substantially in terms of residential segregation of nonparents. A closer inspection of Census data shows that parental segregation ranges from zero percent to nearly forty percent. These segregation patterns create limited opportunities for daily contact between parents and nonparents. I expect that this trend, coupled with decreased electoral participation amongst parents \citep{teixeira_2002}, can lead to substantial variation in individuals' perceptions of marijuana as non-threatening in their own lives. In turn, these segregation patterns can influence perceptions of legalization as a means of imposing control over drug use while reducing harm to potential users, adults and children alike.

%Segregation should be particularly relevant for voting outcomes. While the spatial concentration of nonparents facilitates interaction and communication between people who most likely to support progressive policies and who have fewer constraints on their time, parental segregation also provides a favorable context for support because it shapes perceptions of legalization as an opportunity to create non-threatening, harm-reducing social control policy. 

%%%%%%Maybe move this section up to the part where more info/both sides are needed.

%I also consider how parental segregation may combine with other community attributes, particularly the impermanence of the community, to weaken perceptions that legal marijuana is non-threatening to the rest of the community. One factor to consider is the likelihood of residential turnover. While relatively stable communities allow community residents to reasonably predict daily occurrences, demographic changes disrupt quotidian activities \citep{snow_et_al_1998}. Residential turnover, through the availability of housing, could undermine support for legalization, given that community residents can neither predict who will move into their segregated communities, nor their position on the issues. A community with a greater potential for turnover might signal change in the orientation of the community and undermine beliefs that legal marijuana poses little threat. To capture the housing availability, I consider the percent of housing units available for rent. In addition, other community attributes introduce doubt about whether or not neighbors will remain in the community. Given that singles are the most likely group to relocate \citep{taylor_et_al_2008}, I consider how the size of the unmarried population in a community.










%--------------------------------------------------------------------------------------------------------------------------------------
\section{Data \& Method}

To assess the importance of the distribution of parents and nonparents on support for legalization, I draw on cross-sectional data from 2000 to 2016 for U.S. counties in the ten states where recreational marijuana legalization initiatives appeared on the ballot.\footnote{I exclude Alaska for data reliability issues.} To mitigate the issue of multiple legalization initiatives during the observed years, I use voting data from a state's first initiative, which ensures that the voting data closely correspond with the Census and ACS data used as independent variables. 




Counties as units of analysis provide comparative leverage to explain variation in support for legalization because I can compare 413 counties across ten states. Given that the lived experience of residents in a state may be distinct in different parts of the state (e.g. see \citealt{mcveigh_and_sobolewski_2007}), a county level analysis allows me to account for intrastate heterogeneity that may be associated with views on marijuana legalization. County level demographic data come from the 2000 U.S. Census and the American Community Survey (ACS) 2005-2009. Voting outcomes between 2000 and 2008 are matched with 2000 Census data and votes between 2009 and 2016 are matched with ACS 2005-2009 data. 

The dependent variable, the percent of votes in support of marijuana legalization in each county, comes from the Secretary of State website for each state. Because my main dependent variable of interest is a percentage, I use ordinary least squares regression to estimate the models. I constrain my analysis to ballot initiatives between 2000 and 2016 because the first recreational use initiative appeared in 2002 and the most recent election data end in 2016. To account for boundary changes in counties and county equivalents in Colorado, I construct county clusters for each period by aggregating data for those counties. This procedure amounted to one Broomfield County\footnote{In 2001, Colorado's Broomfield County was created from portions of Adams County, Boulder County, Jefferson County, and Weld County.} cluster from five county units in Colorado, which results in 409 counties for the analysis. Because variation in voting may be associated with state level differences, such as the phrasing of the initiative or the year it was on the ballot, I control or variation between states by holding state effects constant, using state level fixed effects models. Doing so is analytically similar to including a dichotomous variable for each state. The state level fixed effects approach controls for all unobserved, time-invariant state level characteristics.

\subsection{\it{Parental Segregation}}

The key test of my argument involves determining whether counties with higher percentages of parent households, and segregation of parent from nonparent households, exhibit higher support for recreational marijuana. I take data from the Census and the American Community Survey and use the index of dissimilarity to construct a measure of parental segregation.\footnote{A household with at least one related or unrelated child under the age of 18 is considered a parent household. I consider the extent to which households with children are distributed unevenly across Census block groups within each county.} The dissimilarity index for each county is represented as:
\begin{equation}
\sum_{i = 1}^{d} \frac{t_{i} \left| p_{i}-P \right|}{2TP (1-P)}
\end{equation}
where $t_{i}$ is the total population of households in a Census block group, $p_{i}$ is the proportion of households in a Census block group with children, $T$ is the total population of households in the county, and $P$ is the proportion of county households with children (see \citealt{massey_and_denton_1988}). The dissimilarity index can range from 0 to 100, where 0 represents complete integration and 100 indicates complete segregation. The value indicates the percentage of households with children that would have to be relocated to a different Census block group to create an even distribution of parent households across all Census block groups in a county. %{\color{red}Think of an example but in terms that don't deal with exposure.} 
As shown in Table 2, parental segregation varies substantially across the U.S. counties, with at least one county having a high of over thirty-five percent. In addition, a total of five counties contain a single block group and, therefore, have a value of zero on the segregation measure since there are no block groups across which segregation can occur. I estimate the models with these cases included and obtain similar results to the findings when these cases are excluded.\footnote{These include San Juan County, Colorado; Hinsdale County, Colorado; Mineral County, Colorado; Campbell County, South Dakota; Jones County, South Dakota.}
%might be 8 cases in full data set

%ADD A FOOTNOTE SAYING THAT I ALSO USED A MEASURE OF CHILDSEG, NOT FAMCHILDSEG, AND HAD SIMILAR RESULTS... MOREOVER, CHILDSEG AND FAMCHILDSEG ARE HIGHLY CORRELATED, INDICATING PLACES WITH CHILDREN ARE ALSO PLACES WITH FAMILIES. THEREFORE, I USE THE MOST CONSERVATIVE ESTIMATE OF FAMCHILDSEG

%\begin{center}
%\input{/Users/burrelvannjr/Dropbox/Professional/Research/Projects/dissertation/chapters/ch4.structure-and-politics/paper/figures/figure1.tex}
%\end{center}

\subsection{\it{Prospects for Economic Mobility}}

As I have argued, fears about children's prospects for mobility are at the heart of positions on moral issues, including marijuana. It is therefore necessary to include cross-sectional measures of mobility. Many studies of mobility incorporate longitudinal measures, capturing intergenerational mobility by comparing children's wealth, job, and income to that of their parents. While important, these measures of actual mobility tell us nothing about how parents' perceptions of their children's future mobility equally inform and shape parents' behavior. I, therefore, use data from the Census and ACS to construct two measures of the county's overall potential for economic mobility. First, I use the Gini coefficient of income inequality as an estimate of inequality in economic opportunity \citep{corak_2013,kuznets_1955}. Put another way, with more equality in income, residents may be more likely to view their community as having the potential for maintaining or improving their class status. Residents may also view their communities as having the potential for mobility where heterogeneity in occupations exists. The placement of residents across a diverse set of occupations can signal opportunities for mobility through various occupations. As such, I use Census and ACS data to construct a measure of occupational heterogeneity, based on Peter Blau's heterogeneity index \citep{blau_1977a}, measured as:

\begin{equation}
1 - \sum_{i = 1}^{k}{P_{i}}^2
\end{equation}
where $P_{i}$ is the proportion of the population in each occupational category, $i$, across $k$ number of occupational categories. The data are aggregated to each of 13 occupational categories: (1) management, business, and financial (2) professional, (3) healthcare support, (4) protective service, (5) food preparation and serving-related, (6) building, grounds cleaning, and maintenance, (7) personal care and service, (8) sales and related, (9) office and administrative support, (10) farming, fishing, and forestry, (11) construction, extraction, and maintenance, (12) production, and (13) transportation and material moving. The heterogeneity index can range from 0 to 1, where 0 represents complete homogeneity -- that all population members are in the same occupational category, and 1 indicates complete heterogeneity -- that the population members are more evenly dispersed across occupational categories. The index represents the probability that two members randomly selected from the population will be in different occupational categories.
%employed civilian population 16 and over.


\subsection{\it{Control Variables}}

To assess the effect of parental segregation, it is necessary to account for several features of U.S. counties that might also be associated with support for legalization. Unless otherwise noted, all variables come from the Census or the American Community Survey. Through a lack of contact, the spatial segregation of parents from nonparents facilitates support for marijuana legalization. Therefore, in a similar fashion, the percentage of county households without children should increase support for legalization. Political partisanship is also associated with support. Indeed, compared to Republicans, Democratic voters are more supportive of legalization \citep{rosenthal_and_kubby_1996,caulkins_et_al_2012}. I use data from Congressional Quarterly's {\it{America Votes}} to calculate the percentage of voters who voted for the Democratic candidate in the presidential elections that coincide with or immediately precede the decennial data. For example, for marijuana votes during the 2000 period, I use the percent of the vote for Al Gore in 2000 and for the ACS 2005-2009 period marijuana votes, I use percent of the vote for Barack Obama in 2008. 
I also include data from the {\it{Association of Religion Data Archives}} (ARDA) to calculate measures of Evangelical Protestants and Catholics as a percentage of the total population in a county. I include measures of religious adherence because opposition to marijuana remains strong among those affiliated with these religious denominations \citep{caulkins_et_al_2012,palamar_2014}. In order to ensure that religious adherence data precede marijuana voting data and the independent variables from the Census, I match data from the 2000 ARDA county file with county data from both decennial periods. 

Support for legalization initiatives might also depend on population size. I therefore include a measure for the natural log of the total population in a county. Importantly, parental segregation may result from parents' movement to suburban communities. Because measures of percentage of urban and rural land area are not comparable across Census and ACS data, I use population density as a proxy for percent urban. To ensure that the effects of parental segregation do not reflect differences in income, I also include a measure of median household income. I control for the percentage of the population that identifies as African American or Latino, given that these groups exhibit considerable variation with respect to their views on marijuana legalization.\footnote{A March 2010 Pew Research poll showed that Blacks and Hispanics had lower support, respectively 41 percent and 35 percent, for legalization than Whites (42 percent), although in 2013, Blacks showed the strongest support for legalization.} Education is associated with liberal attitudes towards marijuana \citep{pedersen_2009}, and increasing support for marijuana legalization may be attributed, in part, to increases in the size of the college-educated population \citep{rosenthal_and_kubby_1996}. I, therefore, include a measure of the percent of the population aged 25 or older with a bachelor's degree. Also from the Census, as a proxy for the age of the population\footnote{Because the 2000 Census and 2005-2009 American Community Survey use median age measures that are not comparable, I use the size of the aged population as a proxy.}, I include a variable measuring the percentage of the population that is age 65 or over. Descriptive statistics for these and all other variables included in the regression models are presented in Table 2, below.

\begin{center}
%\input{/Users/burrelvannjr/Dropbox/Professional/Research/Projects/dissertation/chapters/ch4.structure-and-politics/paper/tables/table2.tex}
\input{/Users/burrelvannjr/Dropbox/Professional/Research/Projects/dissertation/chapters/ch4.structure-and-politics/paper/tables/table2_.tex}
\end{center}
\vspace{-50pt}


%--------------------------------------------------------------------------------------------------------------------------------------
\section{Results}
%Do: Regression model with same variables
%Possibilities: Use combined variable for TP activity
%Possibilities: If I don't do a Repub variable, add a negation of the outcome variable

Table 3 presents ordinary least squares estimates (with controls for state level effects) of the percentage supporting legalizing recreational marijuana. In the first column, I include the measure of parental segregation, which has a significant and positive relationship with the outcome. This finding lends support for my argument that the spatial segregation of parents from nonparents shapes perceptions of marijuana as non-threatening -- every one unit increase in parental segregation is associated with a .44 percent increase in support for legalization. 

Model 2 omits the segregation measure but includes all control variables. Support for legalization is strongest in communities with high percentages of households without children. In addition, support is higher in counties with high percentages of Democratic voters and high percentages of college graduates. Among the other control variables, coefficients for the size of the Latino and the African American population are negative, suggesting that support for legal marijuana is lower in counties with large minority populations. The age variable is significant and negatively related to the outcome, indicating that support is weaker in counties with higher percentages of elderly residents. Similarly, counties with higher percentages of married residents have significantly lower support. Finally, support for legalization is weak in communities with high rates of employment. 

%The second column of Table 3 presents results after removing measures of community impermanence and including the measure for parental segregation. With these changes, the coefficient for the size of the population of parents remains significant and negative. In other words, the size of the population of parents is associated with decreases in support for legalization. The measure of parental segregation has a positive effect on the vote for recreational marijuana legalization in U.S. counties, supporting my claim that the spatial distribution of parents and children is relevant for support for legalization. All other control variables maintain their significance levels and direction of their relationship with the outcome. 

In column 3 of Table 3, the measure of parental segregation, net of controls, maintains a positive effect on support for legalization. This finding yields support for my argument that perceptions of marijuana as non-threatening to ``ring true'' in places where higher percentages of nonparents reside and in places where parents are spatially segregated from nonparents. Thus, negative perceptions of marijuana do not resonate in communities where residents have limited contact with parents and their children. In this model, all other controls maintain similar relationships with the outcome.\footnote{The size of the effect is small, where Cohen's $f^{2}$ = .018.} % r2 full - r2 reduced/ 1 - r2 full =(.665-.659)/(1-.665) = .01791045

\begin{center}
%\input{/Users/burrelvannjr/Dropbox/Professional/Research/Projects/dissertation/chapters/ch4.structure-and-politics/paper/tables/table3b.tex}
\input{/Users/burrelvannjr/Dropbox/Professional/Research/Projects/dissertation/chapters/ch4.structure-and-politics/paper/tables/table3_.tex}
\end{center}

\vspace{-50pt}
%\subsection{\it{Strengthening Perceptions of Threat}}

%I have argued that parental segregation should facilitate support by shaping collective perceptions that legal marijuana is non-threatening to the community. Residential changes in segregated communities, however, provide opportunities for residents to move into areas of the community with or without a predominance of children, which can influence perceptions about marijuana posing a threat to the socialization of children. If I am correct about the underlying mechanisms, I expect the effect of parental segregation to be weakened in settings with high levels of community impermanence. As Table 4 shows, these expectations are partially confirmed. The first column, with an interaction between parental segregation and the percent of homes available for rent, shows that the main effect of parental segregation remains positive and significant. This indicates that when the percent of renters in a county is zero, parental segregation has a predicted positive effect on support for recreational marijuana legalization. Interestingly, the non-significant coefficient for the interaction term indicates that the positive effect of parental segregation is unaffected in counties with higher percentages of vacant, for-rent property. The second column of Table 4 shows that parental segregation has a positive effect on support for legalizing recreational marijuana when the size of the county's unmarried population is zero. Importantly, the effect of parental segregation is weaker in counties with higher unmarried populations. This means that, perceptions of marijuana as non-threatening are weakened in places with higher percentages of people most likely to vacate the community.

%\begin{center}
%\input{/Users/burrelvannjr/Dropbox/Professional/Research/Projects/dissertation/chapters/ch4.structure-and-politics/Drafts/Manuscript/tables/table4.tex}
%\end{center}

%--------------------------------------------------------------------------------------------------------------------------------------

\subsection{\it{Strengthening Perceptions about Legalization}}

I have argued that parental segregation increases support for legalization by shaping perceptions about the threat of marijuana to children's mobility prospects. If I am correct about these underlying mechanisms, I expect the effect of parental segregation to be strengthened in locales where high prospects for future mobility exist. That is, I expect support to be strongest in places with a high degree of occupational diversity and where income inequality is low. As seen in Table 4, these expectations are confirmed. The first column of Table 4 introduces an interaction between the parental segregation measure and income inequality. The variables in the interaction are centered at their mean values. When included, the main effect of parental segregation remains positive and significant, which indicates that when income inequality is at its mean, parental segregation has a predicted positive effect on support for marijuana legalization. The significant coefficient for the interaction term indicates that this positive effect is weaker in places with higher than average levels of income inequality. In places with high inequality, there are diminished beliefs about prospects for mobility. Thus, legal marijuana would further hinder mobility through what limited opportunities that currently exist. Relatedly, in the second column, parental segregation has a positive effect on support for legalization when occupational heterogeneity is set at its mean, and this positive effect is stronger in counties with higher than average levels of occupational diversity. Where actual opportunities for mobility exist, by way of occupational diversity, residents may view issues like marijuana legalization as less threatening to children's ability to lead successful lives.

\begin{center}
%\input{/Users/burrelvannjr/Dropbox/Professional/Research/Projects/dissertation/chapters/ch4.structure-and-politics/paper/tables/table4b.tex}
\input{/Users/burrelvannjr/Dropbox/Professional/Research/Projects/dissertation/chapters/ch4.structure-and-politics/paper/tables/table4_.tex}
\end{center}

\vspace{-50pt}

\section{Conclusions}

Longstanding parental opposition to morality-related issues remains a common argument for the laggard pace of marijuana policy change in the United States. 
Yet, as I have demonstrated, the character of opposition varies substantially across local contexts, namely as a function of the size and distribution of parents (and their children) across counties. For example, there is strong support for legalization in places like San Francisco County, California and San Juan County, Washington, where parental households are in the minority, but it is also strong in places like Shannon County, South Dakota, where parents predominate. What these counties have in common is high levels of parental segregation. %%%%%%Bring this all up to the front end

In this article, I account for variation across local contexts by considering how the organization of social life in local settings influences the extent to which community residents view legal marijuana as threatening or not. After controlling for numerous other attributes of U.S. counties, I still find a strong, statistically significant relationship between parental segregation and support for legalizing recreational marijuana. As I have argued, this relationship can be explained in terms of strong support amongst nonparents. Segregated communities, in my characterization, are contexts where characterizations of marijuana as non-threatening to children resonate, particularly because daily interactions less frequently parents and children.

The current study addresses broad gaps in literature on sociology of the family and policy change by investigating the structural effects of parenthood on voting for controversial issues. First, given the longstanding tradition in studies of marijuana legalization to investigate the individual precipitants of support, this work follows a more recent line of inquiry devoted to understanding the structural influences on marijuana legalization, which provides general insights into patterns of support for policy change. 

Additionally, this work contributes to a growing chorus of scholarship on the consequences of parenthood \citep{beisel_1997,owens_2016}, with a focus on political outcomes. In particular, this research broadens the scope of scholarly study by empirically investigating the impacts of parenthood on voting. Continuing along this line of inquiry will allow scholars of the family and of political sociology to generate stronger theoretical claims about the role of parenthood in politics more generally.

In this article, I focused on how structural patterns of relations shape perceptions of and voting on controversial political issues. Although public  support for marijuana legalization is above fifty percent nationally, it is my hope that this work will stimulate research on why marijuana policy change has stalled in the face of a growing positive discourse around, and majority public support for, marijuana legalization.


\newpage
%--------------------------------------------------------------------------------------------------------------------------------------
%\section{References}}}

%\bibliographystyle{/Users/burrelvannjr/Dropbox/Professional/Research/References/asa_new}
%\renewcommand{\section}[2]{}%
%\setlength{\bibhang}{40pt}%matches the indentation above for references
%\bibliography{/Users/burrelvannjr/Dropbox/Professional/Research/References/library,/Users/burrelvannjr/Dropbox/Professional/Research/References/ext_library}
%\newpage



%--------------------------------------------------------------------------------------------------------------------------------------
\section{Appendix}
\begin{singlespace}
{\renewcommand\normalsize{\tiny}%makes table font smaller to fit on the page
\normalsize
{\input{/Users/burrelvannjr/Dropbox/Professional/Research/Projects/dissertation/chapters/ch4.structure-and-politics/paper/tables/tableXX3_.tex}}}
\end{singlespace}

