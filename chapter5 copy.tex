\chapter{Conclusion}
 
 Essential to understanding policy change is also understanding the cultural/discursive factors, political environments, and structural contexts within which policies can change.  This dissertation aimed to improve our understanding of the processes of policy change through an exploration of discursive change over time, the role of discourse on the pace of legalization across the United States, and the impact of the segregation on local support for legalization. 
 
 In the second chapter, I show that discourse about marijuana changed over time, becoming less negative and more neutral. Importantly, over time, discussions of marijuana became linked with frames of revenue creation and political behavior. While this framing shift may have contributed to the decrease in negative coverage, in this chapter, I focused only on describing whether or not a shift occurred. 
 
 In the third chapter, I found that the amount of positive discourse about marijuana was critical for the rate of adoption of legalization. Typically, legalization happened more rapidly in states with direct democratic processes (e.g. citizen initiatives). Yet, as I demonstrate, positive discourse about marijuana was critical for the pace of legalization.
 
 Finally, in the fourth chapter, I show how structural inequality shapes support for legalization. Literature has demonstrated that parenthood shapes positions on marijuana policy change. Taking a structural perspective, I show that the distribution of parents and non-parents, and how the segregation of these groups across local contexts shapes aggregate support for legalization. 
 
 This thesis provides many findings related to policy change. Among them are three findings that improve our understanding of policy change:
 
 \begin{itemize}
 \item Contentious political issues may benefit from connecting to ``traditional'' frames, such as revenue and political behavior. Doing so may broaden the beneficiary group of the policy change. 
 \item Discourse on an issue may shape the likelihood of adoption or diffusion for contentious political issues. 
 \item Segregation across local contexts remains a critical factor for aggregate support for contentious political issues. 
 \end{itemize}
 
 While these findings improve our understanding of marijuana policy change, this work was also imperfect because of a series of practical limitations:
 
  \begin{itemize}
 \item Much of the discursive data collected from Proquest was hamstrung by the inability of researchers to collect the population of coverage without serious commitment. As a result, I was unable to collect the population of coverage of marijuana, including articles in national newspapers, without encountering significant challenges or without requests for additional resources. As such, I am limited to only local coverage of marijuana discourse.
 \item There are state-level factors, such as the saliency of marijuana, over time, that may be important influences on the rate of legalization. However, no known data set tracks the saliency of issues without significant manipulation. As such, these data were not included in models assessing the rate of adoption.
 \end{itemize}
 
 These findings suggest a series of improvements for future work, such as
 
 \begin{itemize}
 \item Including coverage of national news discourse on marijuana
 \item Improving scraping, coding, and analytic technologies to ensure replicable results
 \item Comparing coverage of marijuana in the Proquest database to similar coverage in additional databases
 \item Expanding scope conditions to include coverage of marijuana from 1930 and beyond, and incorporating recent coverage of marijuana on social media such as Facebook or Twitter.
 \item Using the Google API to track the saliency of issues over time
 \item Tracking the saliency of marijuana over time, relative to the saliency of other issues during that same time
 \item Improving measures of static perceptions of mobility, and how those relate to support for legalization
 \end{itemize}
 
 
 Through a series of descriptive and empirical investigations, using new methodologies, I have aimed to improve our understanding of policy change on contentious political issues. Through these findings, I hope to have contributed to the technical and theoretical development of social science, particularly that of political change. 
 
 
%%% Local Variables: ***
%%% mode: latex ***
%%% TeX-master: "thesis.tex" ***
%%% End: ***
