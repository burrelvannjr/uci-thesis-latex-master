\chapter{Conclusion}

What explains policy change? More specifically, what explains progressive change on morality policy; that is, how does support for contentious issues evolve? In this dissertation, I have approached this inquiry by way of an investigation of marijuana legalization in the United States. In particular, I centered this investigation on the period between 1990 and 2016 -- a period characterized by heightened attention to marijuana in news media as well as in local level politics. For this investigation, I focused on news coverage of marijuana as well as voting records on ballot initiatives. 

Given the unique character of the marijuana issue, I have attempted to address three puzzles related to marijuana legalization: 1) how did the cultural context or discourse around marijuana change? 2) how did this shift affect the rise in passage of legalization initiatives? and 2) what explains why some places are supportive of these initiatives than others? In the preceding chapters, I have made many claims about that address these questions, but these can be synthesized into two main arguments central to sociological theory: one about culture or discourse, and the other, social structure. 

\section{Summary of the Argument}

In the preceding chapters, I have argued that support for change in morality policies, generally, and marijuana legalization specifically, are the result of changing discourse on the issue. This is not to say that any shift in discourse is relevant for change in morality policy. Rather, insofar as morality policies constitute a distinct policy type, one with organized interests on either side of the issue making their case for what is ``right'' and what is ``wrong,'' these policies are more likely to undergo progressive change more rapidly, when the issue becomes more closely aligned with traditional policy types. In the case of marijuana, I find that, over time, discourse around the issue evolved to incorporate characteristics of ``distributive'' policies. That is, news coverage of marijuana tended to increasingly focus on the revenue benefits of legalization. Although this coverage was often critical of extant drug policy, focusing on revenue fostered a dominant discussion wherein marijuana legalization could be viewed as a way to fund public programs, projects, and services. Unlike redistributive policies that often redirect resources from one social group to another, distributive programs often rely on taxation of goods to fund publicly-available projects. As such, when discourse around the issue took on these frames, voters and political officials may have been more amenable to progressive policy change. In fact, in places where marijuana was framed in terms of its revenue-generating capacity, legalization was adopted at a quicker rate. This finding provides insight into the ways in which morality policies can undergo progressive change. 

The second main argument relates to social structure. In particular, I argue that the distribution of oppositional groups on a contentious political issue, including morality policies, is critical for progressive change. Prior work at the individual level finds that, net of political or ideological factors, people with children are most opposed to marijuana legalization, whereas those without children are more supportive. As prior theory has suggested, parents fear the perceived negative consequences of marijuana in their children's lives, in terms of impeding a child's mobility, as well as the reproduction of their family's status and privilege. This work, however, often ignores the role of social structure on the formation of opinions on political issues. As such, I find that, in places where these oppositional groups are segregated from one another, there is increased support for policy change. That is, I find higher support for legalization initiatives where parents and nonparents are spatially segregated from one another across U.S. counties. The logic is, where these groups are separated from each other, opportunities for contact are limited, which also inhibits the ability of any member of one group to develop a stake in the marijuana debate that is similar to those of the other group. When these groups are separated, parents can support legalization without fear of increased exposure by those perceived most likely to use the drug (nonparents), and nonparents can support legalization without fear that they may inadvertently expose children, in their local environment, to the drug. 


Below, I also recap some of the perspectives that guided my analyses before turning to some of the limitations of the study as well as discussing future directions for this study, the study of marijuana legalization, and the study of morality policies more generally. 

\section{Evaluating Perspectives}

Throughout my analyses, I was guided by five perspectives from the policy change literature that offered potential explanations for why policy change occurs. My goal in each chapter of the dissertation has been to evaluate the effectiveness of the perspectives. I now turn to describe the effectiveness of each approach. 


\subsection{Cultural Context/Discourse}

Central to my argument was the role of culture or discourse -- in particular, discursive change -- in the policy change process. In Chapter 2, I explore variation in discourse about marijuana over time. I find that local newspaper coverage of marijuana, and organization-related marijuana coverage differed in terms of the amount of coverage each set received as well as the frames deployed and the valence of their coverage. Through this chapter, I was able to examine temporal trends in coverage as well as the changing nature of that coverage. I find that general marijuana coverage deployed more frames, and that the of the frames deployed, a majority contributed to decreasingly negative attention to marijuana in news articles. This change in discourse mattered for increasing the rate of adoption of legalization. I find that in addition to the effects of other perspectives, positive and revenue based discourse about marijuana helped to increase the rate of adoption. 


\subsection{Political Institutional Contexts}

Although political institutional contexts have been central to theorizing policy change generally, and progressive change specifically, this work tends to focus on the ways in which policy change occurs from inside these institutions. Prime examples include voting rights legislation, the New Deal programs, and old-age insurance. Yet, for marijuana legalization, until recently, policy change occurred on a state-by-state basis by way of the ballot initiative. 

Across the empirical chapters, I was able to show that, even in the case of marijuana legalization, political institutional contexts matter \citep{amenta_and_elliott_2019}. In chapters 3 and 4, I incorporated measures of political contexts, including state-level party control, support for Democrats, political competition, and even public opinion. In most cases, political institutional factors matter for support for and the adoption of marijuana legalization. However, public opinion did not matter for policy change. Public opinion, in fact, had no effect on the rate of legalization in Chapter 3, while at the same time, measures of Democratic support, and political competition mattered. Moreover, while not directly measured in this dissertation, direct democracy in the form of the citizen initiative and referendum was critical for legalization. Every state that legalized from 1990 to 2016,  did so through the statewide ballot initiative.

\subsection{Policy Feedback}

Policy feedback focuses on the trajectory or history of policy related to the political issue. Importantly, policy feedback perspectives were only explored in Chapter 3. The idea behind policy feedback is that initial policy begets future policy. Specifically, once a policy is set into motion, benefits are conferred to a constituency in the form of financial or legitimacy. The presence of beneficiary groups makes it difficult for politicians to retrench those policies \citep{pierson_2000}. In Chapter 3, policy feedback arguments faired well -- in places where initial marijuana policy had passed in the form of marijuana medicalization, there was an increased likelihood of marijuana legalization over time. 

In sum, it seems as though arguments about policy feedback prove useful for understanding controversial policies, especially those associated with stigma. Policy feedback centers on the assumption that lower-impact policies are required for future expansion -- which is similar to the psychological phenomenon of ``foot-in-the-door.'' As such, controversial political issues may benefit from this policy feedback process. 


\subsection{Advocacy Organizations}

Although much of the literature on advocacy organizations has focused on their activity, marijuana advocacy organizations engaged in very little protest activity. In fact, a majority of their action was limited to gaining coverage in newspapers so that they could disseminate their own perspectives on the marijuana issue. As such, I examined the differences between organization-related coverage of marijuana as well as general coverage. In short, advocacy organizations had little impact on policy change for marijuana. Where other organizations may have sponsored legislation, the four main marijuana advocacy organizations stayed relatively distant from the legalization process. 


\subsection{Structural Contexts}

I find, most of all, that social structure matters for political outcomes, as expected from the sociological literature. I find that measures of segregation or the spatial distribution of interests across localities are relevant for support and ultimate adoption of legalization in the United States. The story that emerges from these findings is that political contexts, discourse, and social structure all contribute to support for and ultimate adoption of policy change on controversial issues. I find that both positive discourse, as well as the distribution of beliefs about marijuana help shape aggregate support for legalization. Not only do I describe the ways in which discourse about marijuana has evolved over time, but I also take a longitudinal approach to understand the rate of and support for legalization. 

\section{Limitations}

This project has several limitations that prevent a fuller analysis. First, my analysis of discourse is restricted. In particular, I focus on local level articles about marijuana-only, as well as local \textit{and} national articles about marijuana and marijuana advocacy organizations. These discrepancies limit a full analysis of coverage, given that my data represent a sample of the coverage of marijuana. Additionally, my analysis is restricted to the search terms I used to locate newspapers and the database I used. I relied on the ProQuest database to locate the articles used in my analysis. While ProQuest provides a relatively comprehensive collection of newspapers, identifying the population of articles can be tricky due to their system. As such, I relied on searching for the term ``marijuana'' instead of ``cannabis'' and the like. 

My analysis of frames is limited, as well. I rely on keyword searches as measures of frames about marijuana. Ideally, I would have liked to read each article and code the frames using a content analytic approach. In chapter 2, I focus on 6 main frames as explained by prior research. Ideally, I would have liked to develop frames inductively, generating frames about marijuana from the data. 

While marijuana issues were rarely if ever brought up for a vote in state houses of government, they were discussed by political officials. Therefore, I would have liked to include any state-level legislative discussion of marijuana into the data set. Doing so would have allowed me to examine the ways in which politicians discussed this and other contentious political issues, and how their frames differed from general marijuana discourse and organization-related discussions of marijuana.

Relatedly, it was difficult to amass data on organizational mentions of marijuana. The main marijuana advocacy organizations rarely publish newsletters or pamphlets. As such, I had to rely on coverage of marijuana \textit{in addition to} mentions of any one of the marijuana advocacy organizations. Ideally, I would have liked to include data from their social media posts, as well as more detailed information about their chapters across the United States.

Finally analysis was limited to between 1990 and 2016, given data availability and the takeoff in the medicalization and recreational legalization of marijuana. A fuller analysis should incorporate a breadth of historical data on changes in marijuana policy. Related to this, the limited time frame of this investigation necessarily captures changes in legalization at the beginning of the process. Within this time frame, eight states had legalized marijuana for recreational use, and all had done so by way of the ballot initiative. Through 2019, one additional state, Michigan, had also legalized through the ballot initiative, whereas two states, Vermont and Illinois, had legalized through their state legislatures. From these trends, we see that the tide is changing, and political officials at the state level may be more willing to take on the marijuana issue. 



\section{Future Directions}

My results, as well as the limitations, provide suggestions for future research. While, in Chapter 2, I have shown that discourse has changed over time, this discourse is limited to local level (regional and ethnic newspapers). A project that incorporates all news media coverage of marijuana would provide a richer analysis of discursive change over time. Moreover, such an analysis should include data from social media to allow for the inclusion of discourse not shaped by news values. Finally, these data should include discussions of marijuana by political officials as a way to incorporate their take and stake in the legalization battle. Moreover, future work should augment the sample of discourse by using additional search terms as well as cross-checking results with additional databases (e.g. America's News). 

In Chapter 3, I explored the factors that contributed to the adoption of legalization. While I included measures of discourse, scholars of diffusion typically include measures of salience. Unfortunately, measures of salience usually include data on Google searches on a specific topic, yet, given the period I examined, I could not included these data. Therefore, future studies should go beyond measures of public opinion to include measures of salience from Gallup's Most Important Problem, or the like. 

Finally, in Chapter 4, while my analysis is structural, many of my arguments about the role of parents in the marijuana legalization process can be explored using individual level data. As such, future research should incorporate individual level data, with respect to parents, on support for marijuana legalization. Doing so will provide a richer theoretical narrative about the process by which people come to support or oppose marijuana legalization, and how this contributes to the evolving nature of policy change on marijuana. 


These steps, if taken, would provide a fuller account of marijuana policy change, and constitute the next stages of this work. Moreover, these directions provide valuable insight into the ways in which morality policy -- its discourse and structural characteristics -- can be investigated to understand how progressive change is possible. 


%%% Local Variables: ***
%%% mode: latex ***
%%% TeX-master: "thesis.tex" ***
%%% End: ***
