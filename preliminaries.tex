
\thesistitle{The Evolution of Marijuana Politics in the United States}

%"Dissertation" for PhD, "Thesis" for master's
\documenttitle{Dissertation}

\degreename{Doctor of Philosophy}

% Use the wording given in the official list of degrees awarded by UCI:
% http://www.rgs.uci.edu/grad/academic/degrees_offered.htm
\degreefield{Sociology}

% Your name as it appears on official UCI records.
\authorname{Burrel James Vann Jr}

% Use the full name of each committee member.
\committeechair{Edwin Amenta}
\othercommitteemembers
{
  David S. Meyer\\
  Charles Ragin\\
  Rory McVeigh
}

\degreeyear{2019}

\copyrightdeclaration
{
  {\copyright} {\Degreeyear} \Authorname
}

% If you have previously published parts of your manuscript, you must list the
% copyright holders; see Section 3.2 of the UCI Thesis and Dissertation Manual.
% Otherwise, this section may be omitted.
% \prepublishedcopyrightdeclaration
% {
% 	Chapter 4 {\copyright} 2003 Springer-Verlag \\
% 	Portion of Chapter 5 {\copyright} 1999 John Wiley \& Sons, Inc. \\
% 	All other materials {\copyright} {\Degreeyear} \Authorname
% }

% The dedication page is optional
% (comment out to exclude).
\dedications
{
  To my communities...\newline
  \newline
  ... and for Riley
}

\acknowledgments
{
  I would like to thank numerous individuals and organizations, without whom this dissertation would not have been possible. 
  
  I am grateful to my advisors and committee: Edwin Amenta, David Meyer, Rory McVeigh, and Charles Ragin for their mentorship, support, and sage advice on this and other projects over the years. 
  
   I would like to thank those shadow advisors who, oftentimes, aren't given the credit they deserve. I thank Davin Phoenix -- without whom I would likely not have a job, and Christopher Parker, who has been completely supportive during my graduate school career. As people of color in academia, you both provided valuable insight into our profession, and I am grateful for your support and mentorship. 
  
  
  %I am grateful to Edwin Amenta, whose advisement steered the direction of this work. I am also thankful for David Meyer, who, even before graduate school was willing to help me devise good research projects, while also caring about my wellbeing. Finally, I would like to thank Rory McVeigh, whose guidance, advice, mentorship, and work has heavily influenced this work. 
  
  I would like to thank faculty at the University of California, Irvine who have provided feedback and support along the way, including Evan Schofer, Nina Bandelj, Carter Butts, John Hipp, David Snow, Yang Su, Andrew Noymer, Sabrina Strings, Rocio Rosales, Jacob Avery, Katie Bolzendahl, and Katie Faust. In addition, I would like to thank the staff who've made my time at Irvine seamless, including John Sommerhauser, Ekua Arhin, Maryann Zovak-Wieder, and Hannah Absher.
  
  Of course, I cannot forget my UCI student life family. These people have been critical to my success and sanity during graduate school. I would like to thank Mart\'{i}n Jacinto, Miles Davison, Jess Lee, Sara Villalta, Claudia Campos, Alma Garza, Monique Kelly, Dana Moss, Thomas Elliott, Amber Tierney, Eulalie Laschever, Jonathan Lui, Kara Placek, Ksenia Gracheva, Ted Watson, John McCollum, Bonnie Bui, Jessica Kizer, Connor Strobel, Chris Gibson, Aaron Tester, Katt Hoban, Steph Jones, Mariam Ashtiani, Alice Motes, Dana Nakano, Hector Y. Martinez, Peter Owens, Amanda Pullum, Rottem Sagi, Beth Gardner, Matt Rafalow, Daisy Verduzco Reyes, Nolan Phillips, Jayson Hunt, and Tyson Patros. In addition, I would like to thank those amazing friends in my cohort: Emma Smith, Sean Drake, Tanya Sanabria, Chris Zoeller, Zaib Tufail, and David Kong. Importantly, my life would have not been the same without the intellect, wit, skills, and connections with my good friends Rodolfo Lopez and Ben Gibson. I am truly grateful for their friendship. Finally, I thank my ``cudidu'', Nicky Jones. I have no words... but we rode this out together. My life is immeasurably changed because of you. 
  
  In 2011, I was lucky enough to conduct research at the University of Notre Dame. During that Summer, and throughout the next year, I was able to develop strong relationships with many people. Among the faculty, I would like to thank Kraig Beyerlein, Richard Williams, David Sikkink, and Bill Carbonaro for early and continued support and mentorship. For their wisdom and friendship, I would like to thank Bryant Crubaugh, (San Diego-native) Austin Choi-Fitzpatrick, (brother) Jeffrey Seymour, (friend and occasional-roommate) David Everson, Marshall Taylor, Kevin Estep, Heather Price, Peter Barwis, Kristi Donaldson, Jonathan Schwarz, Nicole Perez, Ana Velitchkova, Ellen Childs, Josh Dinsman, Amy Jonason, Jade Avelis, Linda Kawentel, Megan Rogers, Megan Austin, and Matthew Chandler. Most importantly, I am so appreciative of my dude... my co-metalcorehead/co-insert-subgenre-here-head/$\emptyset$ fam, Justin Van Ness. Over the years, we've crashed on each other's couches, gone to shows, explored cities, eaten good food, suggested new bands, tried out new breweries, and rocked tf out together, while being professional by presenting our work at conferences, networking, and crankin' papes. Cheers to keeping it strong for many years to come.
  
I am grateful for my Ford Foundation family, which has been integral to my success. Without them, I would not be where I am today. A huge reason why I persisted in graduate school was because of these people. I am truly appreciative of these people of color. I would like to thank the people who stayed so close despite our distance: Victor Ray, Piko Ewoodzie, and Anthony Ocampo. These three have only been a text/phone call away. I am also grateful for so many others in my Ford family, including Vanessa D\'{i}az, Elliott Powell, Jonathan Rosa, Elizabeth Hinton, Yasmiyn Irizarry, Rashwan Ray, Abigail ``Aly'' Sewell, Augustin Diaz, Vlad Medenica, and Armand Gutierrez. 

  Beyond these institutions, I have been lucky enough to develop relationships with people who've continued to provide advice on my work, insight into academia, and lasting friendships. I am thankful for Sarah Soule, Isaac Martin, Andy Andrews, John D. McCarthy, Doug McAdam, Chris Bail, Aliza Luft, Ali Kadivar, Jen Schradie, Alex Hanna, Sharon Quinsaat, Tarun Banerjee, Jaime Kucinskas, Haj Yazdiha, Kirk D. Rogers Jr., Dang Do, Margaret Burns, and Andrew Matschiner. 
  
 I would like to give a shout out to all the musicians and artists whose work kept me motivated to complete this work. Most of all, Architects, which has consistently provided heavy, politically-motivated music throughout my graduate career. I am also thankful for music created by letlive., Bring Me the Horizon (2004-2013), The Devil Wears Prada (2005-2010), Drop Dead Gorgeous, Dance Gavin Dance, Chelsea Grin (2018), Kendrick Lamar, Saosin, Lions Lions, Wind in Sails, Vanna (2004-2010), Jamie's Elsewhere, Pierce the Veil, Before Today, and Four Year Strong. 
  
 % My Family:
   I would also like to thank those people who have shaped me into the person I am, but may not recognize their impact. First, I would like to thank my closest friends from when I was a music major during my undergraduate career, including Loren Gamarra and John Urban. It was difficult to transition out of music, but I think I made a good choice. In addition, I would like to thank my brothers Joe Salcido, Reginald Webber Jr., Javier, Mike Ouellet, Kirshon Asborno, Jesse Glick, Matt McDavid, Chris Benzen, Jon Keith, Ib Salleh, Ish Salleh, other brothers Owen Griffin, Chad Morris, and Jon Locke, and (of course) my littles -- Alison Bellman and Derek Dobbs. 
   
  % I chose to move on from a degree in music to a degree in sociology, in part, because of the inequality experiences shared by my fellow musician friends. %I would like to thank Sean Murillo, John Whitt Jr., and all those others from the Tulare Western High School Drumline, and Band \& Colorguard.  %I am also thankful for those who helped me sustain my passion in music, including Raymond Ferenci, Matt Boyett, Darren Suey, Yolanda Flores, Nick Perry, Chris Ramos, Brian Frasquillo, Johanna Ilano, Jonathan Victorino, Richard Vowell, and Darcy XXX and many others from Tulare Western High School who shaped me. 
   %I would like to also thank Michaela and Talina, and all the members of the innercircle: David Scott (TEDS), Brandon Lee Taylor (RANDO), and Jordan Aguirre (NADROJ). To you all, I say: LPCG. 
   
  
  I would also like to thank the Ford Foundation for funding earlier versions of this project through the Predoctoral Fellowship and the Dissertation Fellowship. I would also like to thank people at various institutions for their valuable insight on versions of this project, including the faculty at New York University, University of Southern California, University of Washington, California State University, Los Angeles, San Diego State University, California State Polytechnic University, Pomona, and Mills College. 
  
  Importantly, I would like to thank my family: To my tia and my cousins, I thank you for staying supportive, even when you didn't know what I was doing in school for so long. To my dad, Burrel Sr, my sister Alyssa, and my brother Serric, thank you for sticking by and being willing to show your support when necessary. To my new family, the Warstadts: Dale, Mitch, Brandon, and Danny -- I can't thank you enough for encouraging me throughout this journey, and for watching me grow for these past twelve years. 
  
  Finally, I thank my wife Melissa. Only you have seen the ins and outs of this struggle, and you are still around. I love you for constantly reminding me of the goal of our work, who we're trying to serve, and when its okay to take breaks. Thank you for pushing through all those late nights with me, the constant travel, as well as the date nights, vacations, and our walks. 
  
  You, and our tiny, Riley, keep me going. 
}


% Some custom commands for your list of publications and software.
%\newcommand{\mypubentry}[3]{
%  \begin{tabular*}{1\textwidth}{@{\extracolsep{\fill}}p{4.5in}r}
%    \textbf{#1} & \textbf{#2} \\ 
%    \multicolumn{2}{@{\extracolsep{\fill}}p{.95\textwidth}}{#3}\vspace{6pt} \\
%  \end{tabular*}
%}
%\newcommand{\mysoftentry}[3]{
%  \begin{tabular*}{1\textwidth}{@{\extracolsep{\fill}}lr}
%    \textbf{#1} & \url{#2} \\
%    \multicolumn{2}{@{\extracolsep{\fill}}p{.95\textwidth}}
%    {\emph{#3}}\vspace{-6pt} \\
%  \end{tabular*}
%}

% Include, at minimum, a listing of your degrees and educational
% achievements with dates and the school where the degrees were
% earned. This should include the degree currently being
% attained. Other than that it's mostly up to you what to include here
% and how to format it, below is just an example.
%
% CV is required for PhD theses, but not Master's
% comment out to exclude
\curriculumvitae
{
\includepdf[pages=-,pagecommand={}]{cv.pdf}
%\textbf{EDUCATION}
  
%  \begin{tabular*}{1\textwidth}{@{\extracolsep{\fill}}lr}
%    \textbf{Doctor of Philosophy in Computer Science} & \textbf{2012} \\
%    \vspace{6pt}
%    University name & \emph{City, State} \\
%    \textbf{Bachelor of Science in Computational Sciences} & \textbf{2007} \\
%    \vspace{6pt}
%    Another university name & \emph{City, State} \\
%  \end{tabular*}

%\vspace{12pt}
%\textbf{RESEARCH EXPERIENCE}

%  \begin{tabular*}{1\textwidth}{@{\extracolsep{\fill}}lr}
%    \textbf{Graduate Research Assistant} & \textbf{2007--2012} \\
%    \vspace{6pt}
%    University of California, Irvine & \emph{Irvine, California} \\
%  \end{tabular*}

%\vspace{12pt}
%\textbf{TEACHING EXPERIENCE}

%  \begin{tabular*}{1\textwidth}{@{\extracolsep{\fill}}lr}
%    \textbf{Teaching Assistant} & \textbf{2009--2010} \\
%    \vspace{6pt}
%    University name & \emph{City, State} \\
%  \end{tabular*}

%\pagebreak

%\textbf{REFEREED JOURNAL PUBLICATIONS}

%  \mypubentry{Ground-breaking article}{2012}{Journal name}

%\vspace{12pt}
%\textbf{REFEREED CONFERENCE PUBLICATIONS}

%  \mypubentry{Awesome paper}{Jun 2011}{Conference name}
%  \mypubentry{Another awesome paper}{Aug 2012}{Conference name}

%\vspace{12pt}
%\textbf{SOFTWARE}

%  \mysoftentry{Magical tool}{http://your.url.here/}
%  {C++ algorithm that solves TSP in polynomial time.}

}




%\includepdf[pages=1-]{cv.pdf}

% The abstract was previously limited to a maximum of 350 words, 
% but the UCI manual at https://etd.lib.uci.edu/electronic/td2e#2.2.1.
% currently does not indicate that there is any word limit for the abstract
\thesisabstract
{
  %In the absence of comprehensive drug policy reform or political advocates, states and counties have become venues for policy change. 
  Whether as medicine or for recreation, marijuana is used by relatively few Americans, yet, in recent years, a growing number of states have attempted to legalize the drug. To that end, this dissertation addresses three important puzzles related to marijuana legalization in the United States: 1) How did discourse around the issue of marijuana evolve? 2) How can we explain the rapid rise in the passage ballot initiatives dedicated to marijuana legalization? And 3) why are some places more supportive of legalization than others? The analysis presented in the dissertation focuses on the American states from 1990 to 2000; a period of increased political and discursive attention to marijuana. I develop a theory about how support for contentious issues evolves. First, I find that the marijuana issue has become increasingly characterized as a distributive rather than morality policy. Second, I find that electoral competition, liberal voting, and policy legacies contribute to quicker adoption of those legalization initiatives. Beyond these standard theoretical arguments, I demonstrate that legalization was adopted more rapidly in states where marijuana was reframed as a distributive policy -- places with higher positive discourse about marijuana and those with increasing discourse about marijuana's ``revenue'' benefits. Finally, this research also sheds light on the clustering of support for legalization. I argue that segregation of residents who may be more likely to oppose marijuana (e.g. parents) from others, creates a situation in which those groups come into less frequent contact, and thus develop lower stakes in the other's perception of marijuana. This spatial distribution of oppositional groups reduces the fears associated with the perceived negative consequences of exposure to marijuana, and increases local level support for legalization.
  
  
  
  %This dissertation is a study of the rise of marijuana legalization from 1990 to 2016. My research on marijuana legalization is guided by three questions: 1) what explains the change in cultural understandings of marijuana, 2) how did that change contribute to the passage of legalization, and 3) what structural factors contributed to support for legalization? Across three chapters, I argue that the rise in support for marijuana legalization is best understood as the result of a sudden shift in positive public sentiment towards marijuana, as well as a shift in the distribution of oppositional groups. 
  
  
  %%This dissertation investigates the causes of support for marijuana legalization across the United States. Across three empirical studies, I argue that increased support for and passage of legalization resulted from 1) increased attention to marijuana, and shifting frames associated with marijuana combined with 2) structural and political conditions that allowed for the resonance of claims about the benefits of marijuana legalization. 
  %The first chapter examines the dramatic shift in coverage of marijuana from 1990 to 2016. I show that media coverage of marijuana shifted from from negative narratives of criminality and delinquency to more neutral coverage that focused on legalization's ability to generate revenue creation as well as the political aspects of legalization. The second study links these discursive data to data on political contexts, policy feedback, and the structure of local environments and demonstrates that positive discourse about marijuana contributed to a quicker rate of adoption of marijuana legalization. In the final chapter, I take a deeper look at the structural determinants of support for legalization initiatives in U.S. counties and find that support his highest in counties where oppositional groups (parents and nonparents) are spatially segregated from each other.  This dissertation builds on prior research that centers on the impact of political institutions, to create a fuller model of progressive policy change by incorporating both cultural and structural determinants of policy change.   
}


%%% Local Variables: ***
%%% mode: latex ***
%%% TeX-master: "thesis.tex" ***
%%% End: ***
