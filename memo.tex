%%%%%%%%%%%%%%%%%%%%%%%%%%%%%%%%%%%%%%%%%
% Long Lined Cover Letter
% LaTeX Template
% Version 1.0 (1/6/13)
%
% This template has been downloaded from:
% http://www.LaTeXTemplates.com
%
% Original author:
% Matthew J. Miller
% http://www.matthewjmiller.net/howtos/customized-cover-letter-scripts/
%
% License:
% CC BY-NC-SA 3.0 (http://creativecommons.org/licenses/by-nc-sa/3.0/)
%
%%%%%%%%%%%%%%%%%%%%%%%%%%%%%%%%%%%%%%%%%

%----------------------------------------------------------------------------------------
%  PACKAGES AND OTHER DOCUMENT CONFIGURATIONS
%----------------------------------------------------------------------------------------

\documentclass[12pt,stdletter,dateno,sigleft]{newlfm} % Extra options: 'sigleft' for a left-aligned signature, 'stdletternofrom' to remove the from address, 'letterpaper' for US letter paper - consult the newlfm class manual for more options
\usepackage{csquotes}
\usepackage{charter} % Use the Charter font for the document text
\usepackage{setspace}
%\linespread{1.5}
%\newsavebox{\Luiuc}\sbox{\Luiuc}{\parbox[b]{1.75in}{\vspace{0.5in}
%\includegraphics[width=1.2\linewidth]{logo.png}}} % Company/institution logo at the top left of the page
%\makeletterhead{Uiuc}{\Lheader{\usebox{\Luiuc}}}

\newlfmP{sigsize=0pt} % Slightly decrease the height of the signature field
%\newlfmP{addrfromphone} % Print a phone number under the sender's address
%\newlfmP{addrfromemail} % Print an email address under the sender's address
%\PhrPhone{Phone} % Customize the "Telephone" text
%\PhrEmail{Email} % Customize the "E-mail" text

%\lthUiuc % Print the company/institution logo

%----------------------------------------------------------------------------------------
%	YOUR NAME AND CONTACT INFORMATION
%----------------------------------------------------------------------------------------

%\namefrom{C. Ben Gibson} % Name

\addrfrom{
\today\\[12pt] % Date
%4100 Calit2 Building \\ % Address
%UCI, California, 92617
}

%\phonefrom{(251) 510-0864} % Phone number

%\emailfrom{cbgibson@uci.edu} % Email address

%----------------------------------------------------------------------------------------
%	ADDRESSEE AND GREETING/CLOSING
%----------------------------------------------------------------------------------------

\greetto{Dear Committee:} % Greeting text
\closeline{Sincerely, \newline Burrel} % Closing text

\nameto{Edwin Amenta, David Meyer, Charles Ragin, and Rory McVeigh} % Addressee of the letter above the to address

\addrto{
\emph{Dissertation Committee} \\ % To address
%University of California, Irvine \\
%123 Pleasant Lane \\
%City, State 12345
}

%----------------------------------------------------------------------------------------

\begin{document}
\begin{newlfm}

%----------------------------------------------------------------------------------------
%	LETTER CONTENT
%----------------------------------------------------------------------------------------

Please find my responses to your prior comments below. \newline%the suggestion of Reviewer 3, we have changed the title of the paper to ``The Bootstrapped Robustness Assessment for Qualitative Comparative Analysis. Due to these changes, we have also changed the name of the package (braQCA) and it's components (baQCA and brQCA).

%Thank you. \newline

\textbf{Chapter 1}:

\textit{Rory's Comments}:

I have added an introduction that discusses the impetus for the study. I focus on adjudicating between various theories about progressive policy change. I also focus on the importance of studying marijuana legalization specifically -- and how marijuana legalization fits into (and is different from) a larger set of cases of policy change. 

Although I don't provide a history here, I have adapted/removed the brief histories in the various chapters. Outside of the various actions by Congress, and a few citizen initiatives in the 1970s, until the 1990s, not much action was taken with regard to marijuana policy change. 

I have removed the focus on discourse. \newline 

\textit{David's Comments}:

I have stated the problem and puzzle, and described why studying marijuana legalization is of import. 

I also now describe the relevant approaches to studying policy change. I do not necessarily put forth a set of theoretical expectations as much as I am identifying a set of literatures and stating that I am testing them on the marijuana legalization case. 

In addition, I describe how marijuana fits (and does not fit) into our typical understandings of the set of progressive policy cases. 



\textbf{Chapter 2}:

\textit{Charles' Comments}:

The charts are now broken out by org/non-org. 

For non-org coverage, I use only local articles, however, because marijuana advocacy organizations don't really have press releases, and therefore don't have much coverage, I augment their set with national coverage. As such, I have created two tables to break down the coverage for both sets. 

The main figure regarding frames was hard to decipher. As such, I have broken all figures out into high, medium, and low frequency frames -- thus there are now three figures. 

Finally, I describe the ways in which these frames were selected. Although these selections don't necessarily fit with the Snow and Benford version of framing, they do provide some insight into the ways in which marijuana was being discussed. \newline


\textit{Rory's Comments}:

The chapter has become descriptive, and I sort of flesh out some of the frames, and parse why some articles were less likely to be classified as negative. \newline

\textit{Edwin's Comments}:

I have given examples of the various frames. 

You had some issues with the coding of the valence. Given your trepidation regarding the coding, I've added more explanation (as well as a link to the coding dictionaries) to the chapter. 

Moreover, I was able to break out the frames by local versus org coverage. This can be found in Figures 2.6 through 2.8. Moreover, the valences of coverage are shown in Figures 2.9 and 2.10. \newline

\textit{David's Comments}:

You're right, this analysis of frames doesn't quite gel with Snow and Benford's or Gamson's interpretation of framing. Instead, I focus on keywords (as described above). 

You also mention being hesitant about the valence coding. However, your example doesn't really get at the ways in which ``positive'' versus ``negative'' valences are coded. But, I have included a much more in-depth discussion of the ways in which articles are coded, and have provided links to the dictionaries. 

As many have said, the graph was hard to read. As such, I have broken the graph out into high, medium, and low frequency frames.


\textbf{Chapter 3}: 

\textit{Edwin's Comments}:

Edwin's first question focuses on whether or not there are political leaders involved in the passage process. The short answer is no. The longer answer is that this process has been complicated, with various representatives within the U.S. House, the U.S. Senate, and State legislatures proposing bills, but all would die in committee. Interestingly, there doesn't seem to be a social movement story to the passage process, insofar as we focus on social movements initiating the initiative process. There may be, however, a ``movement' sponsorship'' story. The problem is, there aren't good data on \textit{when} marijuana movement organizations threw in their support (e.g. beginning/signature gathering stages, or after polls were released, or near elections). Because it's hard to pinpoint the timing of their role, it's also difficult to incorporate their impact into the analyses.

I have removed mentions of movements from the literature. Instead, I focus on political institutions (and public opinion) as well as partisanship. 

Yes, coverage is central to my claims about policy adoption. You do bring up a great point about NYT and LAT, yet I removed these cases given their overrepresentation in the larger data set. I understand that this is and could be a problem. As such, I have addressed this in the Limitations section as well as the Future Directions section. My goal is to incorporate these data in the analysis. 

Unfortunately, because most of these newspapers are local or ethnic newspapers, it's difficult to gauge their position on marijuana -- incorporating NYT and LAT might help with this for future studies. 

I've altered the discussion to focus on policy ``adoption'' rather than diffusion. To highlight what is going on in states, I provided a table which describes the various characteristics of successful and unsuccessful marijuana legalization initiatives. This speaks to your next point to include a table. 

And the policy history section has been adapted.  In addition, I've included a descriptives table.

Toward the end of Edwin's comments, he suggested many things. First, because the control measures make model fit take a hit, he suggested I work with a smaller set of more plausible controls over those that seem less plausible. As such, in Table 3.2, we can see that I've dropped some of the measures... but generally the addition of any controls is a detriment to model fit. As you can see, and from my interpretation of the models, we see that the second model (without controls) is still the best fit model for explaining the rate of passage for legalization in the U.S.

Regarding these models, Edwin mentioned that I should line up all the variables together, then add the positive coverage measure. As such, I've removed the previous Model 1 (which included only the measure of public opinion).

Edwin also asked if states are available to adopt if they do not have the initiative/referendum as an option. The obvious answer is ``yes,'' yet, in practice, it has not worked out this way. Of course, we are in the midst of the policy change process, so there are (still) various routes to legalization that may not have been utilized yet. But, generally, states legalize marijuana \textit{outside} of Federal or State legislative action. In fact, during the period of analysis, states only passed legalization via the referendum. Thus, for this analysis, there is a perfect overlap for the subset of cases that legalized marijuana, and the set of cases that have the initiative/referendum. Importantly, this relates to one of Edwin's first questions about whether this process only occurred via the referendum/initiative, or whether there were other ways. Again, because we are in the midst of this process, all cases of statewide legalization during the period of analysis resulted from initiatives. The one negative case is in Maine, where, in 2018, marijuana was legalized via the state legislature. This case is outside the period of analysis and would require a separate project to identify the mechanisms that set off the legalization process. As states begin to legalize via state legislatures, I believe there will be more room for identifying the processes of legalization that are distinct for non-direct democratic states. Therefore, there would be no results for states without initiatives where legalization passed.\newline


\textit{David's Comments}:

Thank you for the references to include in my literature review. I have included these, in addition to others. I do not think that referenda are the only route to policy change. However, as of 2018, of all the states that had legalized, only Maine did so outside of the ballot initiative. 

I describe how legalization fits into the broader policy environment in Chapter 1. I also explain why this may have been the case for legalization -- why the initiative was important for legalization -- in Chapter 5. 


\textit{Charles' Comments}:

You're right! Interpolation will only go until the final value. In previous iterations, my models duplicated values between 2013 and 2016. Instead, I have now included public opinion date from 2016. Now interpolation is not a problem.

I misrepresented the Democratic vote data in the previous version. As can be seen in Chapter 3, I use data from all presidential elections between 1988 and 2016, interpolating values between elections. 

I use political competition because 1) analysts typically use competition measures and 2) the focus is not simply about turnout -- rather, it's about the distribution of interests represented in elections. This is laid out in the chapter. 

Recent research on diffusion effects uses proportions/percentage measures. The logic is that there is increased pressure to enact policy change when higher percentages of neighbors have enacted those policies. A dichotomy would not get at that. I do, however, run the models with both a percent and dichotomy, and the models are identical. 

Above, I have described the breakdowns for the discourse date. 

I use ln(pop) as a measure given diffusion research that includes it as a control. If you recommend I remove it, I'd be happy to do so. 

As described in the chapter, positive discourse comes from the algorithm used in the previous chapter (which is based on positive, neutral, and negative dictionaries). 

Yes, states exited the analysis when they legalized. 

As described in the chapter, a one unit increase in positive discourse is actually a one-article increase. 

Your last two questions are more comments. I'm happy to share my R code for you to examine the data yourself. And I also spend time discussing the results. 


\textbf{Chapter 4}: 

\textit{David's Comments}:

I've included literature about how parents tend to be more conservative than nonparents on most political issues... yet marijuana legalization still remains unique --- leading to unique levels of support or opposition. 

In the sections on Parental Segregation and Mobility, I discuss literature that highlights the ways in which contexts of high inequality tend to be less conducive for mobility, while high occupational differentiation contributes to perceptions of increased mobility. \newline


\textit{Charles' Comments}:

There are no predicted values below 20 or above 80. We're good. OLS is good for this. 

For equation 4.1, I've provided an example for your clarification. 

As mentioned above, I've included discussions of literature that highlights the roles of inequality and job prospects on perceptions of mobility, and how these play out for support/opposition to marijuana legalization. 

Yes, the heterogeneity index ranges from 0 to 1...  although these values are nearly impossible... much like a correlations.... (unless all people in a county have the same job). 


In the chapter, I've included two interaction tables, as you suggested. For the values I select low (1st quartile), medium (median) and high (3rd quartile). 

Finally, I have not run tests of curvilinearity. Is there really an impetus to do so? If so, why? and what variables would you suggest testing? \newline


\textit{Edwin's Comments}:

I have removed the canned history. I focus on how there were various votes. 

The next section focuses on the various determinants of support for liberal or progressive policy change. As such, I have removed discussions of marijuana legalization, per se, and focused on progressive policies as a class of cases. 

I discuss the ways in which parents oppose liberalization on progressive/morality-based policies. 

I line up the predictions about race, age, education, religion, and partisanship, and set up my models in this way as well. 

For you and David, I have selected a few cases that are similar on various aspects, but differ on the parental segregation measure, and thus, support for legalization. 


I end the chapter with trying to place the case in the larger set of cases and try to understand how processes important for legalization may inform policy change on other contentious political issues. \newline



\textbf{Chapter 5}: 

I have added a conclusion that discusses the take-aways from the previous three chapters, as well as appraises the various arguments discussed in Chapter 1.

%\setlength\parindent{24pt}
%\indent C. Ben Gibson \\ 
%\indent Department of Sociology \\ 
%\indent University of California, Irvine \\
%\indent 4100 Calit2 Building\\
%\indent UCI, California, 92617\\
%\indent Phone: (251) 510-0864 (cell)\\
%\indent Email: cbgibson@uci.edu\\

%----------------------------------------------------------------------------------------

\end{newlfm}
\end{document}