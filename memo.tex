%%%%%%%%%%%%%%%%%%%%%%%%%%%%%%%%%%%%%%%%%
% Long Lined Cover Letter
% LaTeX Template
% Version 1.0 (1/6/13)
%
% This template has been downloaded from:
% http://www.LaTeXTemplates.com
%
% Original author:
% Matthew J. Miller
% http://www.matthewjmiller.net/howtos/customized-cover-letter-scripts/
%
% License:
% CC BY-NC-SA 3.0 (http://creativecommons.org/licenses/by-nc-sa/3.0/)
%
%%%%%%%%%%%%%%%%%%%%%%%%%%%%%%%%%%%%%%%%%

%----------------------------------------------------------------------------------------
%  PACKAGES AND OTHER DOCUMENT CONFIGURATIONS
%----------------------------------------------------------------------------------------

\documentclass[12pt,stdletter,dateno,sigleft]{newlfm} % Extra options: 'sigleft' for a left-aligned signature, 'stdletternofrom' to remove the from address, 'letterpaper' for US letter paper - consult the newlfm class manual for more options
\usepackage{csquotes}
\usepackage{charter} % Use the Charter font for the document text
\usepackage{setspace}
%\linespread{1.5}
%\newsavebox{\Luiuc}\sbox{\Luiuc}{\parbox[b]{1.75in}{\vspace{0.5in}
%\includegraphics[width=1.2\linewidth]{logo.png}}} % Company/institution logo at the top left of the page
%\makeletterhead{Uiuc}{\Lheader{\usebox{\Luiuc}}}

\newlfmP{sigsize=0pt} % Slightly decrease the height of the signature field
%\newlfmP{addrfromphone} % Print a phone number under the sender's address
%\newlfmP{addrfromemail} % Print an email address under the sender's address
%\PhrPhone{Phone} % Customize the "Telephone" text
%\PhrEmail{Email} % Customize the "E-mail" text

%\lthUiuc % Print the company/institution logo

%----------------------------------------------------------------------------------------
%	YOUR NAME AND CONTACT INFORMATION
%----------------------------------------------------------------------------------------

%\namefrom{C. Ben Gibson} % Name

\addrfrom{
\today\\[12pt] % Date
%4100 Calit2 Building \\ % Address
%UCI, California, 92617
}

%\phonefrom{(251) 510-0864} % Phone number

%\emailfrom{cbgibson@uci.edu} % Email address

%----------------------------------------------------------------------------------------
%	ADDRESSEE AND GREETING/CLOSING
%----------------------------------------------------------------------------------------

\greetto{Dear Committee:} % Greeting text
\closeline{Sincerely, \newline Burrel} % Closing text

\nameto{Edwin Amenta, David Meyer, Charles Ragin, and Rory McVeigh} % Addressee of the letter above the to address

\addrto{
\emph{Dissertation Committee} \\ % To address
%University of California, Irvine \\
%123 Pleasant Lane \\
%City, State 12345
}

%----------------------------------------------------------------------------------------

\begin{document}
\begin{newlfm}

%----------------------------------------------------------------------------------------
%	LETTER CONTENT
%----------------------------------------------------------------------------------------

Please find my responses to your prior comments below. Each committee member's comments are block-quoted and in {\color{red}red}, and my responses immediately follow the comment. \newline%the suggestion of Reviewer 3, we have changed the title of the paper to ``The Bootstrapped Robustness Assessment for Qualitative Comparative Analysis. Due to these changes, we have also changed the name of the package (braQCA) and it's components (baQCA and brQCA).

%Thank you. \newline

\textbf{Chapter 1}:

\textit{Rory's Comments}:

\begin{quotation}{\color{red}\noindent \footnotesize
You need to clearly articulate your general research question and preview the theoretical argument that will guide the three empirical chapters (modifications, of course, will be made in each of those three chapters).  You need to motivate the question.  Why should we care?  What is the substantive importance of the change?  Beyond the substantive issue, why should this work be important to academics.
}
\end{quotation}



I have added an introduction that discusses the impetus for the study. I focus on adjudicating between various theories about progressive policy change (pp 3--7). I also focus on the importance of studying marijuana legalization specifically -- and how marijuana legalization fits into (and is different from) a larger set of cases of policy change (pp 2--3). 



\begin{quotation}{\color{red}\noindent \footnotesize
You should provide rich historical description of the way in which Marijuana has been perceived throughout history (your brief historical sketches in the chapters fall way short).  You should develop and describe the overarching theoretical approach you are taking in the dissertation as a whole.
}
\end{quotation}



Although I don't provide a history here, I have adapted/removed the brief histories in the various chapters. Outside of the various actions by Congress, and a few citizen initiatives in the 1970s, until the 1990s, not much action was taken with regard to marijuana policy change. Within each chapter, I discuss what was going on with marijuana with regard to the question at had in the chapter. For example, in chapter 2, I trace the change in discourse by highlighting critical moments in the marijuana's legality and connecting them with periods of dominant narratives (pp 11--14). In chapter 3, I track the actual historical development of marijuana's legality, federal and Congressional attention to keeping marijuana illegal, as well as discussing what states were doing to counter these policies (pp 42--44). Because chapter 4 relies on similar data (voting data on initiatives), it would likely repeat much of the history presented in chapter 3. Therefore, I removed this historical discussion to give more attention to the theory of support for legalization.



\begin{quotation}{\color{red}\noindent \footnotesize
I think you should abandon all of this talk that suggests a causal effect of media discourse.  I can't see any place in the dissertation where you have made that case (theoretically or empirically) convincingly.
}
\end{quotation}




I have removed the focus on solely discourse: I treat this chapter as a way of explaining the various perspectives on policy change (pp 3--7) and how I intend to use the chapters to test these arguments. \newline 



\textit{David's Comments}:

\begin{quotation}{\color{red}\noindent \footnotesize
Presently, the introduction and conclusion are basically empty. I agree with Rory that you want the introduction to establish the problem you want to address or question you plan to answer, as well as to situate your investigation in a broader literature. You nod to policy literature, but don?t describe what prevalent approaches in this vast literature are, nor how your investigation relates to them. I agree with Edwin that you also want to push to theoretical and/or substantive implications of your findings. It would be worthwhile to outline a theoretical set of expectations that you plan to test, refine, or refute, and to situate policy on marijuana in a larger category of some other kind of policies. Do you expect regulation/legalization of marijuana to be similar to policies on public education or Medicaid or the death penalty? Why should someone care about your dissertation if they aren?t particularly concerned with marijuana?
}
\end{quotation}

I have stated the problem and puzzle, and described why studying marijuana legalization is of import. I also now describe the relevant approaches to studying policy change (pp 3--7). I do not necessarily put forth a set of theoretical expectations as much as I am identifying a set of literatures and stating that I am testing them on the marijuana legalization case. Here, I also describe the importance of studying marijuana as a case of policy change (pp 2--3). I describe how marijuana fits (and does not fit) into our typical understandings of the set of policy cases, and in particular, progressive policy change. 

In the conclusion, I wrap up by discussing how well or poorly each perspective performed in explaining marijuana policy change. \newline



\textbf{Chapter 2}:

\textit{Charles' Comments}:






\begin{quotation}{\color{red}\noindent \footnotesize
States that he did not use national newspapers, but then includes them in the list in the appendix (table 2.2)
}
\end{quotation}


For non-org coverage, I use only local articles, however, because marijuana advocacy organizations don't really have press releases, and therefore don't have much coverage, I augment their set with national coverage. As such, I have describe how each set of articles was selected (pp 17--18), and have created two tables to break down the coverage for both sets (pp 19-20), and include a complete list as well (p 21). In the analysis, I treat both sets separately, and as such, the descriptive charts are broken out by organization-related coverage and local (non-organization-related) coverage (pp 32--37).



\begin{quotation}{\color{red}\noindent \footnotesize
The analysis of frames is sketchy and incomplete. Why revenue? What does politics include? etc.
}
\end{quotation}


Finally, I describe the ways in which these frames were selected. Although these selections don't necessarily fit with the Snow and Benford version of framing, they do provide some insight into the ways in which marijuana was being discussed. I include reasoning (pp 22--24) for why the six frames were selected, much of it based on the historical literature about discussions of marijuana, as well as on the empirical focus on studies of marijuana public opinion. These topics were used to inform the frames I selected for description. In addition, show a table (p 24) describing the keywords used to code the articles. 


\begin{quotation}{\color{red}\noindent \footnotesize
Needs more examples of articles and codings
}
\end{quotation}

I have included examples for each type of frame that was coded for analysis (pp 24--27). I select examples from sections of articles that have been coded as each frame. 

I have given more attention to describing the ways in which I use the algorithm to code for valence -- I've provided an example, given a breakdown of the ways in which the algorithm selects valence or polarity, and provided links to the algorithm and its dictionaries. 





\begin{quotation}{\color{red}\noindent \footnotesize
Figure 2.4 hard to decipher and needs to be broken into 2 or 3 figures, with the same y scale in each
}
\end{quotation}

The main figure regarding frames was hard to decipher. As such, I have broken the figure into three figures -- high, medium, and low frequency frames -- based on their frequency across time (pp 30--31). 

In addition, I maintain this breakdown of three figures for the comparisons between org- versus non-org/local coverage (pp 32--34). Relatedly, when comparing the change in the valence or polarity of coverage over time, I break the graphs out by org-vs-local for each frame (pp 35--37).\newline


\textit{Rory's Comments}:

\begin{quotation}{\color{red}\noindent \footnotesize
It is an interesting thing to see how media coverage of marijuana changed over time and how the issue became connected to different issues (e.g., medical marijuana, etc). \newline

\noindent It is ok for you to try to reveal some of the broad patterns in coverage......but it would be so much better if you also dug into the substance of the coverage.  Present us with some actual media frames so we can see examples of the different ways in which marijuana was framed.
}
\end{quotation}




The chapter has become descriptive, and I sort of flesh out some of the frames, and parse why some articles were less likely to be classified as negative.  As can be seen, I dig in a bit more to the substance of the frames by providing examples of marijuana discussions being connected to different topics (pp 24--27) , which is followed by figures representing the amount of coverage each one of those topical frames received over time (pp 28--34). \newline

\textit{Edwin's Comments}:


\begin{quotation}{\color{red}\noindent \footnotesize
It would be good to see examples of the different media frames. 
}
\end{quotation}


As mentioned above, I have given examples of the various frames (pp 24--27). 


\begin{quotation}{\color{red}\noindent \footnotesize
Note that the valence example is a little troubling in that ?unfairness? is a treated as a negative word, yet in this context it seems to connote a positive valence toward marijuana liberalization.  Is this a general issue? Also, who is the ?I? in this example? Is this a quote from someone?
}
\end{quotation}



You had some issues with the coding of the valence. Given your trepidation regarding the coding, I've provided more information regarding how the algorithm codes the valence/polarity of articles -- based on words in the document -- and have provided links to the software as well as to the coding dictionaries on which it relies (p 22). Moreover, I've added more explanation of the sample quote (p 22) and give an example of how the quote would be coded, based on the logic of the software (p 22). 


\begin{quotation}{\color{red}\noindent \footnotesize
One direction would be to connect frames to the organizations.  Are there any indications regarding the frames being put forward by the movement organizations?  This could come in national coverage as well as the local coverage on which you are focusing.   Which frames are they advancing.  Although you can?t prove anything about why it is happening, it would be interesting to see if some of the frames advanced by the movement organizations appeared more frequently in coverage.   It would be especially interesting if the organizations differed on the frames they advanced and that some of them were getting more traction than others, regardless as to why that was happening.
}
\end{quotation}

I was able to break out the frames and valences by local versus org coverage (pp 30--37). The tricky part of this work was in focusing on the organization-related coverage. Unfortunately, when these organizations are being covered (based on a subsample of ~100 articles), I saw that most of the time, organizations were mentioned in passing (e.g. ``X person was a member of MPP.''). As such, I struggled to argue that movement organizations are putting forward this coverage -- rather, marijuana is being discussed in a specific way, and organizations are mentioned alongside that marijuana coverage. As such, I removed claims about organizational influence over discursive change. I do keep advocacy organization influence arguments in the introduction as part of the canon of literature on policy change, however, I simply examine whether or not organizations are more frequently mentioned alongside specific frames about marijuana. 




\begin{quotation}{\color{red}\noindent \footnotesize
Are there valences in the strictly organizational coverage?  Are some basically treated better than others.  This would be good to know.\newline

\noindent Another issue is whether the frames have different valences?  Do they change over time in their valences? 
}
\end{quotation}

You bring up two good points: do org articles and non-org articles deploy similar or different frames, and what are the valences of those frames. To answer this question, I break the analysis into subsets (pp 35--37), looking at the percent of coverage that is either negative or neutral over time (as seen on p 29, coverage of marijuana became less negative and more neutral over time). I first look at the percent of coverage of each frame (all six) that is negative, broken out by org versus non-org, over time in Figure 2.9 (p 35). Next, I ok at the percent of coverage of each frame that is neutral, broken out by org versus non-org, over time in Figure 2.10 (p 37). As described in the text, org coverage became less negative when including politics and revenue frames, while non-org/local coverage became less negative when using the patients frame. Moreover, the increase in neutral coverage looks more dramatic for org-related coverage (compared to local coverage), when using revenue, politics, rights, or patients frames. \newline

\textit{David's Comments}:


\begin{quotation}{\color{red}\noindent \footnotesize
I?m not clear that the identification of key words is equivalent to ?frames? as described by Snow or Gamson, for example. This is an argument you could conceivably make. I?m also not clear about the identification of valence based on the key terms. For example, if an article reports someone saying that revenue from legalization would not be as great as advocates argue, I fear that would come up positive because tax revenue is mentioned. It could well be that you?ve got a way to solve it, and I just don?t understand the explanation. It?s worth explaining.
}
\end{quotation}




You're right, this analysis of frames doesn't quite gel with Snow and Benford's or Gamson's interpretation of framing. Instead, I focus on keywords (as described above) and provide examples of the frames (pp 24--27). 

You also mention being hesitant about the valence coding. However, your example doesn't really get at the ways in which ``positive'' versus ``negative'' valences are coded. But, I have included more explanation of the ways in which articles are coded via an example, and have provided links to the dictionaries (pp 22). I also provide some of the reasoning for the selection of the frames (pp 22--24).





\begin{quotation}{\color{red}\noindent \footnotesize
Presently, there is a cool graph on p. 17 which describes how the prevalence of key words changes over time, but it?s hard to make sense of what?s going on from just looking at the graph?aside from the point that the prevalence of the terms fluctuates over time. It seems likely that which issues dominate could reflect and influence politics and public opinion.
}
\end{quotation}


As many have said, the graph was hard to read. As such, I have broken the graph out into high, medium, and low frequency frames (pp 30--31), as well as by organization-related and non-org/local coverage of marijuana (pp 31--34).\newline


\textbf{Chapter 3}: 

\textit{Edwin's Comments}:

\begin{quotation}{\color{red}\noindent \footnotesize
You have a somewhat unusual case in that political leaders or parties do not seem to be taking a stand or at least they do not bring the issue up in legislative bodies or take the lead in the initiative process. Or do they?  If political leaders do not start the initiation process, who does exactly and does this make a difference?  Possibly the unusual nature of the policy and how it works can explain what happens with it. \newline

\noindent It seems that you have a series of standard arguments about what causes policy to be adopted.   Your argument is that something about public discourse is influencing not just the issue to be discussed, which is the standard argument about its influence, but for it to be passed.  I am pretty sure that Soule etc. think movements can help to influence the political agenda, but usually not further in the process. You should say more about why you think this is so.  
}
\end{quotation}

Edwin's question focuses on whether or not there are political leaders involved in the passage process. The short answer is no. The longer answer is that this process has been complicated, with various representatives within the U.S. House, the U.S. Senate, and State legislatures proposing bills, but all would die in committee. Interestingly, there doesn't seem to be a social movement story to the passage process, insofar as we focus on social movements initiating the initiative process. There may be, however, a ``movement' sponsorship'' story. The problem is, there aren't good data on \textit{when} marijuana movement organizations threw in their support (e.g. beginning/signature gathering stages, or after polls were released, or near elections). Because it's hard to pinpoint the timing of their role, it's also difficult to incorporate their impact into the analyses.


\begin{quotation}{\color{red}\noindent \footnotesize
As you are lining up the standard arguments, note that the partisan influence argument is not the same as the public opinion one. It holds that parties differ on issues of public policy and when one of them takes power they will adopt that policy.  They are not doing it for reasons of reelection, but because they were elected and are being responsible.  Also, the partisanship argument can be made wholly without any reference to social movement influence or social movement scholars for that matter, and so you should change that discussion.  

Political opportunity arguments generally are about helping movements to mobilize and organize, which is not something you are discussing or modeling.  Your discussion of the literature on policy change is too heavily relying on social movement people. 

Public opinion is a separate route to policy influence, which is supposed to influence politicians of any party, regardless of their party?s position on the issue.  It is surprising that you do not have discussed a movement influence here.  Most political influence of movement arguments and models indicate whether the movement organization or organizations or actors are relatively well organized, mobilized, resourced, or active, etc., depending on what they are arguing. E.g., organization/protest/whatever is influential in policymaking. Is there any movement argument to be made here or at least some control measure available to employ in models across states?  
}
\end{quotation}

As mentioned above, I have removed mentions of movements from the literature. I, instead, focus on political institutional arguments which include public opinion, as well as partisanship, diffusion, and discourse arguments. As suggested, I line up these arguments as a way of appraising their utility in explaining the adoption of legalization (pp 44--51). 


Given that there doesn't seem to be much of a movement explanation (which I have also modeled), based on the historical literature as well as on data from ballot initiative sponsors, I have removed political opportunity arguments, as well as much of the literature by social movements scholars. Instead, I focus on political institutions (and public opinion) as well as partisanship (pp 44--51). 


\begin{quotation}{\color{red}\noindent \footnotesize
Given that media coverage it is central to your account and constitutes the greatest value added here, in additional to theorizing it further, can you give some examples of positive coverage?  Or at least show which newspapers were connected to which states in the analyses?  Would not the NY Times be an important newspaper for New York and the LA Times for California?  Especially regarding local stories?  
}
\end{quotation}


Yes, coverage is central to my claims about policy adoption. You do bring up a great point about NYT and LAT, yet I removed these cases given their overrepresentation in the larger data set. I understand that this is and could be a problem. As such, I have addressed this in the Limitations section as well as the Future Directions sections of the dissertation. My goal is to incorporate these data in the analysis at a later date. 



\begin{quotation}{\color{red}\noindent \footnotesize
Also: Is there anything that accounts for positive coverage?  Such as editorial position of the newspaper?  Some of them may have editorially favored the passage of the bills.
}
\end{quotation}

Unfortunately, because most of these newspapers are local or ethnic newspapers, it's difficult to gauge their position on marijuana -- incorporating NYT and LAT might help with this for my inevitable future extension of this project. 


\begin{quotation}{\color{red}\noindent \footnotesize
When you get to the explanation part, I would refer to policy adoption rather than policy diffusion.  The policy does not seem to be spreading in any consistent way, or even very far, and your diffusion measure does not come out as significant in the model.  And some states are adopting a policy and others not. 
}
\end{quotation}



I've altered the discussion to focus on policy ``adoption'' rather than diffusion. To highlight what is going on in states, I provided a table which describes the various characteristics of successful and unsuccessful marijuana legalization initiatives. This speaks to your next point to include a table. 


\begin{quotation}{\color{red}\noindent \footnotesize
Note that there does not seem to be a competition reason for nearby states to adopt the policy, as is true for some policies like tax breaks for businesses.  You should probably supply a potential reason for diffusion.
}
\end{quotation}

This is noted, and as you can see, not only do I discus the ways in which marijuana legalization is similar and different from other kinds of progressive policies (pp 40--42), I also provide more description of how this affects its likelihood of diffusion (pp 49--50)



\begin{quotation}{\color{red}\noindent \footnotesize
I am wondering is there is some easier, preliminary way to analyze these data, such as tables with states that considered legislation, those that adopted it, those that did not adopt, and those that did not consider it at all?   You need a table at the very least regarding which states brought the question up for consideration and what happened. \newline

\noindent It would be helpful to have some differences in means or something for the other measures of interest. \newline

\noindent To lead into this, you need to augment your policy history discussion.  It is fine as far as it goes, but needs to be supplemented by a discussion of which states adopted which sorts of marijuana laws when they adopted them and the method by which they came into being.
}
\end{quotation}



The policy history section has been adapted (pp 42--44). In this section, I've included a table that provide a preliminary way of examining the similarities and differences between the initiatives in different states, as well as whether or not the initiative passed (p 44). In addition, I've included a descriptives table depicting the variables used in the analysis (p 60).

\begin{quotation}{\color{red}\noindent \footnotesize
If the control measures generally make the model fit less well, can you dump some of the ones that do not help or are less plausible?  It is a little worrisome for interpretation that the coefficients bounce around in significance.  Probably you should stick with a smaller set of plausible controls and use them in each model.\newline

\noindent I am not sure why you have public opinion separately, unless you want to break up the other arguments into their own models as well.  You may want to add all the plausible measures in one model and then just add in the positive coverage measure in another, given the set up, and say what it means technically and substantively to have it included.\newline

\noindent Also, are states really available to adopt the policy if they do not have the initiative/referendum as an option? What happens if these states are dropped from the analyses? 
}
\end{quotation}



Toward the end of Edwin's comments, he suggested many things. First, because the control measures make model fit take a hit, he suggested I work with a smaller set of more plausible controls over those that seem less plausible. As such, in Table 3.2, we can see that I've dropped some of the measures (p 62)... but generally the addition of any controls is a detriment to model fit. As you can see, and from my interpretation of the models, we see that the second model (without controls) is still the best fit model for explaining the rate of passage for legalization in the U.S.

Regarding these models, Edwin mentioned that I should line up all the variables together, then add the positive coverage measure (p 60). As such, I've removed the previous Model 1 (which included only the measure of public opinion).

Edwin also asked if states are available to adopt if they do not have the initiative/referendum as an option. The obvious answer is ``yes,'' yet, in practice, it has not worked out this way. Of course, we are in the midst of the policy change process, so there are (still) various routes to legalization that may not have been utilized yet. But, generally, states legalize marijuana \textit{outside} of Federal or State legislative action. In fact, during the period of analysis, states only passed legalization via the referendum. Thus, for this analysis, there is a perfect overlap for the subset of cases that legalized marijuana, and the set of cases that have the initiative/referendum. Importantly, this relates to one of Edwin's first questions about whether this process only occurred via the referendum/initiative, or whether there were other ways. Again, because we are in the midst of this process, all cases of statewide legalization during the period of analysis resulted from initiatives. The one negative case is in Maine, where, in 2018, marijuana was legalized via the state legislature. This case is outside the period of analysis and would require a separate project to identify the mechanisms that set off the legalization process. As states begin to legalize via state legislatures, I believe there will be more room for identifying the processes of legalization that are distinct for non-direct democratic states. Therefore, there would be no results for states without initiatives where legalization passed.\newline


\textit{David's Comments}:

\begin{quotation}{\color{red}\noindent \footnotesize
There is a large literature on policy diffusion in political science; you mention some of it, but don?t really engage it. For example, during the time period you study, only states with initiative/referenda opportunities in place legalized. Do analysts think that makes for likelier innovation on policy generally? Or only on specific sorts of policies? Further, a very cursory internet search turned up a few articles on the specifics of diffusion of marijuana legalization?as well as on medical marijuana. You cite only Boushey, and don?t explain what he found. Your paper will be stronger if you can describe what others have done, and why your approach is different?and presumably superior, and whether it leads to different conclusions. If the preexisting literature is essentially scant, you should explain why.
}
\end{quotation}

Thank you for the references to include in my literature review. I have included these, in addition to others. I did not engage the diffusion literature in the previous version. Importantly, in this version, I have dove in more on the process of testing the competing mechanisms for policy change (pp 44--51), one of which is diffusion (pp 49--50). In addition, at Edwin's suggestion, I've moved away from language of diffusion, and instead focus on diffusion effects on adoption (because legalization itself isn't really diffusing very far during the period of study). 

I do not think that referenda are the only route to policy change. However, as of 2018, of all the states that had legalized, only Maine did so outside of the ballot initiative (pp 42--44). 


\begin{quotation}{\color{red}\noindent \footnotesize
You will want to explain what we can learn from this analysis that affects our understanding of something about this case, and something about a category of cases (policy diffusion? Morality policies?)
}
\end{quotation}


I describe how legalization fits into the broader policy environment in Chapter 1. I also explain why this may have been the case for legalization -- why the initiative was important for legalization -- in Chapter 5.  Yet, I more adequately propose my theoretical reasoning for the adoption of legalization in the intro section of the chapter (pp 40--41), and wrap up the chapter with expectations for other policies bearing characteristics similar (if at all) to marijuana legalization (pp 63--65). \newline


\textit{Charles' Comments}:

\begin{quotation}{\color{red}\noindent \footnotesize
the dep var goes to 2016, yet public opinion polls stop in 2013; interpolation not possible post 2013
}
\end{quotation}

You're right! Interpolation will only go until the final value. In previous iterations, my models duplicated values between 2013 and 2016. Instead, I have now included public opinion date from 2016, as described in the chapter (p 54). Now interpolation is not a problem.

\begin{quotation}{\color{red}\noindent \footnotesize
ditto for Democratic vote; stops in 2008; why not 2012 or 2016?
}
\end{quotation}

I misrepresented the Democratic vote data in the previous version. As can be seen in Chapter 3 (p 54), I use data from all presidential elections between 1988 and 2016, interpolating values between elections. 

\begin{quotation}{\color{red}\noindent \footnotesize
for political competition, why not use turnout?
}
\end{quotation}

I use political competition because 1) analysts typically use competition measures and 2) the focus is not simply about turnout -- rather, it's about the distribution of interests represented in elections. This is laid out in the chapter as part of a political institutional model or rather, political contexts sympathetic to diverse interests (pp 47) 



\begin{quotation}{\color{red}\noindent \footnotesize
neighboring states w/ pot: I would have preferred a dichotomy (any neighboring state) instead of proportion of neighbors
}
\end{quotation}

Recent research on diffusion effects uses proportions/percentage measures. The logic is that there is increased pressure to enact policy change when higher percentages of neighbors have enacted those policies (p 50). A dichotomy would not get at that, or it would not be as sophisticated a measure of pressure. I do, however, run the models with both a percent and dichotomy, and the models are identical. 


\begin{quotation}{\color{red}\noindent \footnotesize
Same Proquest problem as in chapter 2? (national media included in counts)\newline

\noindent How did you measure positive mj coverage by state?
}
\end{quotation}

Above, I have described the breakdowns for the discourse data. These are the same data as from chapter 2, coded as positive, neutral, or negative. As described in the chapter, positive discourse comes from the algorithm used in the previous chapter (which is based on positive, neutral, and negative dictionaries). I describe this process in the chapter (pp 56--58). Additionally, I clump all data together for the analyses (p 21). \textit{I would be happy to re-include the tables here}, but I don't want to repeat the tables throughout the dissertation. I do, however, include a table with the counts of positive articles by state (p 60).


\begin{quotation}{\color{red}\noindent \footnotesize
I fail to see why ln(pop) is needed as a control variable
}
\end{quotation}


I use ln(pop) as a measure given diffusion research that includes it as a control. If you recommend I remove it, I'd be happy to do so. 


\begin{quotation}{\color{red}\noindent \footnotesize
I assume states exited the analysis once they legalized mj
}
\end{quotation}

Yes, states exited the analysis when they legalized. In total, 10 state-years exited the analysis (1,323 cases in total -- 49 states across 27 years, 1,313 cases in the analysis). 

\begin{quotation}{\color{red}\noindent \footnotesize
What?s a one unit increase in positive discourse? (unknown metric)
}
\end{quotation}

As described in the chapter(p 63), a one unit increase in positive discourse is actually a one-article increase. 


\begin{quotation}{\color{red}\noindent \footnotesize
I find it hard to believe that proportion neighbors gets two stars in model 3; there are others that seem questionable, as well. Example: prior initiatives in model 3?its coef is just barely 2 times its s.e., which is usually one star, not 3 \newline


\noindent chapter ends abruptly; needs more discussion; maybe discuss some states?
}
\end{quotation}

Your last two questions are more comments. I'm happy to share my R code for you to examine the data yourself. And I also spend time discussing the results and their implications (pp 61--65). \newline


\textbf{Chapter 4}: 

\textit{David's Comments}:

\begin{quotation}{\color{red}\noindent \footnotesize
Regarding your clarification of my comment on the chapter on parental segregation: I was asking whether parents are just generally more conservative than non-parents, when demographic variables are controlled, or whether they are only more conservative on marijuana? In other words, are parents who are liberal on taxes or abortion, for example, still more likely than comparable non?parents to oppose legalizing marijuana?
}
\end{quotation}



I've included literature about how parents tend to be more conservative than nonparents on most political issues (p 71) ... yet marijuana legalization still remains unique --- leading to unique levels of support or opposition. 

\begin{quotation}{\color{red}\noindent \footnotesize
I also asked why lower inequality would lead to better prospects of social mobility. I had not heard this premise before, and don?t see why it would be true.
}
\end{quotation}

In the sections on Parental Segregation and Mobility, I discuss literature that highlights the ways in which contexts of high inequality tend to be less conducive for mobility, while high occupational differentiation contributes to perceptions of increased mobility (pp 74--75). \newline


\textit{Charles' Comments}:

\begin{quotation}{\color{red}\noindent \footnotesize
Actually, running OLS on a dependent variable that is a percentage is acceptable only if the predicted y values aren?t negative or greater than 100. In short, there?s a ceiling and a floor that must be respected. I would be wary of predicted y values less than 10 or greater than 90.
}
\end{quotation}

There are no predicted values below 20 or above 80. We're good. OLS is good for this. 

\begin{quotation}{\color{red}\noindent \footnotesize
I had a hard time deciphering eq. 4.1. Please run through an example using real data.
}
\end{quotation}

For equation 4.1, I've provided an example for your clarification (p 79). Using real data would be difficult because I would have to aggregate across all block groups within a given county. So I run through an example using hypothetical data from one block group.

\begin{quotation}{\color{red}\noindent \footnotesize
You argue that less inequality yields more mobility. I think you need more justification. There?s not much inequality in poor neighborhoods, mining towns, blue collar neighborhoods, etc.
}
\end{quotation}


As mentioned above, I've included discussions of literature that highlights the roles of inequality and job prospects on perceptions of mobility (pp 74-75), and how these play out for support/opposition to marijuana legalization. 


\begin{quotation}{\color{red}\noindent \footnotesize
A score of exactly 1 on the heterogeneity index is not possible. Very close to 1 is, of course.
}
\end{quotation}


Yes, the heterogeneity index ranges from 0 to 1...  although these values are nearly impossible... much like a correlations.... (unless all people in a county have the same job, or all people have different jobs/the distribution of occupations is evenly spread out such that the number of unique jobs is equal to the population with one person in each occupation). 



\begin{quotation}{\color{red}\noindent \footnotesize
Regarding interaction models, I like descriptive tables, like this one for an equation with x1, x2, and x1x2:
}
\end{quotation}


In the chapter, I've included two interaction tables, as you suggested (pp 87--89). For the values I select low (1st quartile), medium (median) and high (3rd quartile). I think this makes it clearer to see the relationship. I can also include in interaction plot if necessary.


\begin{quotation}{\color{red}\noindent \footnotesize
Your arguments about parents, parental segregation, etc. make me think that some of the relevant variables might be nonlinear, especially with increasingly steep slopes. Did you conduct any tests for curvilinearity?
}
\end{quotation}

Finally, I have not run tests of curvilinearity. Would would be the impetus? What variables would you suggest testing? \newline


\textit{Edwin's Comments}:

\begin{quotation}{\color{red}\noindent \footnotesize
it is not clear why there is the canned history of marijuana legislation in this, especially as it repeats other parts of the dissertation. If you are going to have a descriptive piece, it should set up the family and related arguments.  You may want to say simply that there were various votes on the legalization of marijuana and some passed and some didn?t, but that initiatives provide valuable data on support for the policy, which helps to go beyond findings in public opinion polling that one sees in other research.
}
\end{quotation}


I have removed the canned history section and, as you recommend, briefly state in the introduction that there were various votes on the issue (pp 67--68). Next, I discuss the ways in which understanding parents'/families' concerns helps us understate different (opposing) stakes in the legalization debate, and more than that, that the distribution of these concerns is important for aggregate support -- which is beyond what is expected by partisanship/public opinion models (pp 68--69)


\begin{quotation}{\color{red}\noindent \footnotesize
You need a separate theory type of section that indicates the determinants of support for ?liberal? policies, including gun control, abortion rights, civil rights, lgbt rights, etc.  These sorts of sections and this would probably should not include substantive discussions of the case at hand. These arguments should focus on age, race, education, religion, partisanship.  It might be worth showing results or findings on public opinion, etc. for some of these policies from other research projects.  It would be valuable to make claims about differences between different liberal types of policies in terms of how parents might react to them. 
}
\end{quotation}



The next section focuses on the various determinants of support for liberal or progressive policy change. As such, I have removed discussions of marijuana legalization, per se, and focused on progressive policies as a class of cases (pp. 69--71). In interest of not repeating other parts of the dissertation, and to not distract from the discussion of parents as taking conservative stances on policies, I have kept the discussion of support for other kinds of liberal policies brief. 


\begin{quotation}{\color{red}\noindent \footnotesize
It sounds as though you want to argue that some of these policies, including marijuana, are hindered more than others by the fact that parents do not want liberalization for specific reasons, which would be different from abortion, say, or gun control. Unless you are saying that parental segregation etc. matters more for all liberal policies. You are probably going to have to locate this in better citations, as David indicates, or indicate allied literatures from where you are drawing hypotheses.
}
\end{quotation}

I discuss the ways in which parents oppose liberalization on progressive/morality-based policies (pp 71--72). Beyond that, I argue that segregation of opposing views limits interactional opportunities for each group to discuss their position, and shape aggregate political behavior (pp 72--73). Moreover, this is particularly important for  progressive/morality-related policies, where parents' conservative views are strong, but their fears about community and child wellbeing are quieted when they are spatially separated from those most likely to take advantage of these policy changes.. This plays out for legalization, where parents may fear that nonparents in their communities might be smoking around their kids. 


\begin{quotation}{\color{red}\noindent \footnotesize
You should probably line up these arguments as alternatives.  Probably age, race, and education, etc. could be juxtaposed to religion, and that juxtaposed to partisanship.  There are different ways to group these, but you should decide and then group them.  You should end up with some hypotheses. This section should end with your hypotheses about parents and mobility, which are the value-added ones. \newline

\noindent Given this new set up, it might make more sense to present results not first with just the segregation measure, but with each grouping of measures as you appraise each set of hypotheses.  You could then play them out more than you do here, and make more of the work. \newline

\noindent You should then add in your measures regarding parental segregation and then the two interactions.   Just adding parental segregation adds only less than one percentage points to your explained variance, but your best model adds with the one interaction about 3.5 percentage points to the explained variance and you probably can make something of that.
}
\end{quotation}

 
I line up the predictions about race, age, education, religion, and partisanship, then add the parental segregation, nonparent percent, and my mobility measures of income inequality and occupational heterogeneity (pp 75--77). Moreover, I set up my models in this way as well (pp 83--89). 

It is true that the interaction does a lot better in terms of explaining variation in the outcome. As such, I spend more time discussing the interaction, its meaning, and even present predicted values for the interactions (pp 85--89).


\begin{quotation}{\color{red}\noindent \footnotesize
Show results with some counties that are matched on many key measures, but differ on parental segregation or maybe that and occupational segregation, as most of your results are based on that interaction, and then how they voted. 
}
\end{quotation}


For you and David, I have selected a few cases that are similar on various aspects, but differ on the parental segregation measure, and thus, support for legalization. We see two matched counties (pp 84--85), yet one county (Mahoning, OH) opposed legalization, while the other (Apline, CA) supported legalization. 


I end the chapter with trying to place the case in the larger set of cases and try to understand how processes important for legalization may inform policy change on other contentious political issues. \newline



\textbf{Chapter 5}: 

I have added a conclusion that discusses the take-aways from the previous three chapters, as well as appraises the various arguments discussed in Chapter 1.

%\setlength\parindent{24pt}
%\indent C. Ben Gibson \\ 
%\indent Department of Sociology \\ 
%\indent University of California, Irvine \\
%\indent 4100 Calit2 Building\\
%\indent UCI, California, 92617\\
%\indent Phone: (251) 510-0864 (cell)\\
%\indent Email: cbgibson@uci.edu\\

%----------------------------------------------------------------------------------------

\end{newlfm}
\end{document}