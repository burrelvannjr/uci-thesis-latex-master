\chapter{Introduction}
\begin{quotation}
\begin{singlespace}
\noindent {\footnotesize The sex fiend is a progressive criminal. He often begins with annoyances. He progresses to the sending of obscene letters, or exhibitionism. One finds him ``annoying'' children, or following women. For all these things, he often is merely fined, or given ``orders to leave town,'' or punished by a short jail sentences -- none of which deters him in the slightest degree from other and more serious offenses. And every sex criminal is a potential murderer.\newline

%\noindent Therefore, it would seem that the problem of the sex criminal is being mishandled... \newline

\noindent It should be determined, for instance, to what extent the recently widespread use of marijuana, or American hashish, has been responsible for the sex crime. \newline

\noindent Thus, a tremendous force may now be exerted toward the eradication of a drug which violently affects sex impulses.}\footnote{Hoover, J Edgar. September 26, 1937. ``WAR ON THE SEX CRIMINAL!'' \textit{Los Angeles Times}.}
\end{singlespace}
\end{quotation}
The above quote, appearing in the \textit{Los Angeles Times} in the fall of 1937, marked the beginning of the political fight, led by Harry J. Ansligner, to eradicate marijuana from America. Journalistic accounts like these, and the subsequent slew of restrictions on drugs, resulted in marijuana being conflated with fear, and ultimately prohibited in the United States through the Marihuana Tax Act of 1937. Terms like criminal, immoral, and deviant were often used, and it was assumed that marijuana was a danger to society. While early discussions of marijuana were usually negative, it took over half a century for the tone of coverage to shift towards attacking these perceptions:
\begin{quotation}
\begin{singlespace}
\noindent {\footnotesize In 1998, President Bill Clinton signed a provision that made people temporarily or permanently ineligible for federal financial aid depending on how many times they had been arrested and convicted of a drug offense... the effect was real and devastating: the people most in need of financial aid were also being the most targeted for marijuana arrests and were therefore the most at risk of being frozen out of higher education. \newline

\noindent Why would Democrats support a program that has such a deleterious effect on their most loyal constituencies? The fact that they are ruining the lives of hundreds of thousands of black and Hispanic men... barely seems to register. \newline

\noindent This is outrageous and immoral and the Democrat's complicity is unconscionable...

\noindent No one knows all the repercussions of legalizing marijuana, but it is clear that criminalizing it has made it a life-ruining racial weapon. When will politicians have the courage to stand up, acknowledge this fact and stop allowing young minority men to be collateral damage?}\footnote{Blow, Charles M. October 23, 2010. ``Smoke and Horrors'' \textit{New York Times}.}
\end{singlespace}
\end{quotation}

%Some of the most consequential legislation in this state's history has been enacted this way: most famously, Proposition 13, which put a cap on property tax increases and required a two-thirds vote by the Legislature to raise taxes.

%For anyone with money and political savvy, the ballot initiative has been an effective way to work around a balky Legislature. And the only way to change or repeal an initiative is to go back to voters.

%"Why try to pass something in the Legislature when you have the money to get something on the ballot and make it that much more difficult to change it later," said Joe Mathews, a longtime critic of the California governance system. "This serves people who have the money and have the power."

By the end of the twentieth century, newspapers began to change their tune. Editorials encouraging the passage of marijuana legalization were more frequent. Coverage began to be associated with medicine, social justice, and civil rights. Although marijuana remained illegal at the federal level, there was an uptick in state level activity to change marijuana laws statewide -- and the use of the ballot initiative proved relevant for enacting medicalization and legalization. Most noticeably, medical marijuana initiatives took off in the 1990s and 2000s and were successful in California's 1996 Proposition 215.

At the same time, there were changes in public opinion on marijuana -- by 2013, American public support for marijuana surpassed 50 percent \citep{gallup_2013}. This presents my puzzle: What accounts for the change in marijuana policies in the United States since 1990? 

Understanding marijuana policy change is important for substantive discussions about progressive policy change and the set of cases to which it belongs. Marijuana legalization bears some similarity to contentious and morality-related policies such as abortion rights, anti-prohibition, and lotteries \citep{gusfield_1963,andrews_and_seguin_2015,pierce_and_miller_1999,berry_and_berry_1990}. Marijuana legalization is similar to anti-prohibition and lotteries in that their perceived negative consequences and costs will only be incurred by those who make use of the policy \citep{mikesell_and_zorn_1986,berry_and_berry_1990}. Marijuana legalization differs, however, from anti-prohibition given alcohol's longstanding social and economic impact on American society \citep{andrews_and_seguin_2015}. Marijuana legalization also bears similarity to taxation policy \citep{amenta_and_halfmann_2000} in its generation of economic benefits. Finally, marijuana legalization has followed the reverse trend of tobacco policy. Tobacco use is tied to social and economic fabric of the United States. Yet, in recent years, has experienced some retrenchment --  due to increasingly negative connotations and beliefs attached to smoking. Marijuana, on the other hand, has become increasingly associated with medicinal and economic benefits. 

Legalization as progressive policy presents an intriguing case not wholly identical to others, yet one that can inform our understanding of the factors that contribute to progressive policy change. Moreover, given the recency of legalization, studying this policy can provide insight into the process of early adoption of policy in the long history of policy change. 

%This question is important because, as Ferree, Gamson, Gerhards, and Rucht (2002) observe, the media arena is the primary site for contests over meaning. How homosexuality is portrayed in the news media will have a powerful impact on on the larger public discussion of homosexuality. The news media is the main source of frame packages around an issue that get deployed in public conversations and when formulating public opinion (Gamson and Modigliani, 1989). Studying newspaper coverage of homosexuality can serve as a proxy for the larger public discussion over time, giving insight into how the general public understood homosexuality and how this changed over time. Additionally, changing the media portrayal of homosexuality is likely to lead to changes in the public discussion of homosexuality, so understanding how activists can change media portrayals of issues would contribute to understanding how movement can impact culture. 

%Social movements are an important part of the political landscape. They are vehicles by which people express grievances and work towards change. They have the potential to change both the political and cultural landscape. How they do this is still not well understood. Assessing how movements impact policy making and other political outcomes is complicated, making consensus on the mechanisms of influence elusive (Amenta and Caren, 2004). Even less well understood is how movements might have cultural impact. 

This project aims to fill some of these gaps by explaining how and why changes in the cultural, political, and structural landscape have contributed to changes marijuana policy in since 1990.


\section{Explanations for Policy Change}

What are possible explanations for changes in marijuana policy? I explore five perspectives to explain the legalization of marijuana. These five perspectives guide my analysis throughout this dissertation. In the conclusion, I evaluate how well each perspective helped in explaining my puzzle.


%\subsection{Public Opinion}

%Scholars have long studied the relationship between public opinion and policy change \citep{burstein_and_linton_2002}. The evidence suggests that the relationship is recursive. 

%Public opinion shifts when previously unused frames enter the media. These frames are in turn used in public discussions and are incorporated into personal formulations of opinion on an issue. Brewer (2003) explores how political knowledge mediates the impact of media frames on public opinion, and finds that undisputed media frames are more likely to influence public opinion than disputed frames.



\subsection{Political Institutional Contexts}

One possible explanation for the shifts in marijuana policy is related to political institutional contexts. Certain contexts provide more opportunities for policy change than others. Political context theories often focus on the role of elite allies in government that may contribute to the overall success of a policy change \citep{amenta_et_al_1994,amenta_2006}. Research in political sociology and political science has focused on the role of left- or reform-oriented parties in the passage or adoption of progressive policies \citep{amenta_and_elliott_2019,amenta_et_al_2005,korpi_1983}. Moreover, whether or not elite allies hold majorities in government is important for the likelihood for policy change\citep{ansolabehere_and_snyder2006,winters_1976,abramowitz_1983,campagna_and_grofman_1990}. And because political officials as allies want to make good on their promises, they will often support or sponsor policies that accord with the interests of their constituents \citep{page_and_shapiro_1983,mayhew_1974,downs_1957,stimson_et_al_1995}. 

Voting results are also an important part of the political institutional environment. Support for specific parties signal to both constituents and political officials that certain types of policy change are possible or not \citep{amenta_and_elliott_2019}. These characteristics pique the interests of politicians and constituents to support specific policy changes \citep{berry_and_berry_1990,boushey_2016}. 

Public opinion is also an important part of the political institutional environment \citep{burstein_1998,burstein_2003}. Similarly, public opinion serves as a signal about what sorts of policies are possible. Therefore, public opinion reveals the saliency of political issues for constituents as well as political officials \citep{pacheco_2012,nicholson-crotty_2009}.



%What exactly counts as political opportunity is not well defined (Goodwin, Jasper, and Khattra, 1999). It has been used broadly to include virtually all macro-level effects on movement mobilization and outcomes. In an international comparative context, political opportunity refers to the openness or closedness of a political system (Kriesi, 2006). In the single nation context, political opportunity refers to the configuration of allies, antagonists and bystanders. Scholars have tended to interpret this as how friendly political institutions are at any given time, as well as whether public opinion is on the movement's side. 

%The political opportunities that lead to mobilization and policy success may also lead to increased and better media coverage of movements. Just as political opportunities can signal to movements that the time is right for mobilization, political opportunities could send signals to media professionals to pay more attention to an issue, or to treat an issue differently. These signals could include the election or appointment of new politicians, or the passage of new reforms regarding an issue. Political opportunities also influence movement's tactics in the media arena (Rohlinger, 2006), which could, in turn, influence the coverage of the movement's issue. Previous research has found mixed results for political opportunity?s influence of news coverage of movement organizations, as it seems to influence coverage in partisan news outlets, but not in mainstream outlets (Amenta, Caren, and Stobaugh, 2012; Rohlinger, Kail, Taylor, and Conn, 2012).


\subsection{Policy Feedback}


An alternative explanation related to the political context involves the legacy of policy reforms or policy feedback around an issue. Drawing on ideas of increasing returns \citep{pierson_1996,pierson_2000}, policy reforms will have both immediate, as well as long lasting benefits for future policy change \citep{amenta_et_al_2012}. Policies often lead to institutionalized benefits to a specific group of people. These benefits can provide a boost in the efficacy of future advocacy on the part of beneficiaries \citep{amenta_and_caren_2004}. 


%If the group is represented by a movement, organizations affiliated with this movement can gain an additional benefit in the form of acceptance (Amenta and Caren, 2004). Policy reforms may cast organizations that represent the constituencies benefiting from the reforms in a more legitimate light. This could lead to the organizations having access to more and better resources, enabling them to be more effective in future advocacy. Furthermore, this increased legitimacy can lead to members of the movement being placed in positions of authority within the state. %As an example, AIDS activists were appointed to a Presidential task force on AIDS in the 1990s (Epstein, 2007). These members were then able to more directly influence policy on their issues, further improving its policy profile. 


%Applying this to the media arena, the political process is typically a highly newsworthy process, most newspapers assign one or more beats to covering politics. Movements and their actors are likely to be covered if a policy they are advocating for is making its way through the political process. If the policy is successful, the movement may be seen as more legitimate not only by state actors, but also the media, leading to further coverage, especially as the policy is implemented. This effect will be cumulative, with more policy successes leading to more legitimacy in the media arena. Additionally, if a movement actor is appointed to a state position, they will receive the benefit all state actors receive in the media arena, with much easier access and more frequent coverage by the media (Gans, 1979; Oliver and Maney, 2000; Sobieraj, 2010; Tuchman, 1978).


%\subsection{Crisis}


%Another possible explanation for the change in coverage of homosexuality concerns crises (Snow, Cress, Downey, and Jones, 1998). Crises are typically highly newsworthy, having several characteristics journalists look for in deciding what's news (Gans, 1979). Crises are unusual, outside of the routine of daily life. Crises typically impact a large number of people. Crises also usually involve the government trying to solve the crisis. Newspapers are likely, then, to devote a fair amount of coverage to crises. Movements may be able to benefit from this coverage if they are affiliated with an issue impacted by the crisis, and crises can spawn entirely new movements as those affected mobilize themselves (Snow et al., 1998). 

%The partial meltdown at the nuclear reactor at Three Mile Island is an example of this type of crisis. Local pre-existing anti-nuclear organizations experienced a tremendous surge in members after the accident, and new organizations were founded in communities closest to Three Mile Island to oppose the restart of the Unit 1 reactor and impose strict regulations on the cleanup of Unit 2 (Cable, Walsh, and Warland, 1988). The Three Mile Island accident also introduced new framing packages that significantly shifted public opinion against nuclear power (Gamson and Modigliani, 1989). 

%Crises may also undermine the authority of the status quo, bringing attention to deficiencies of the dominant groups (Bail, 2012). Especially if the government is seen as contributing to the disaster, or negligent in addressing the disaster, official authorities may lose legitimacy on the issue (Snow et al., 1998). The news may turn to movements in these cases in the search for new experts. Movements could potentially be considered alternative experts on the issue, or they may be used as ``authentic'' responses to the crisis Sobieraj (2010). Either way, movements can expect to receive increased coverage in the wake of a crisis that involves an issue they are mobilized around. This coverage offers movements the opportunity to inject new ways to think about the issue, and the crisis may have destabilized the dominant discourse in ways that make the movement especially influential in this area.


\subsection{Cultural Context/Discourse}

Mass media are central for making sense of relevant events \citep{gamson_and_modigliani_1989}. According to \citet{ferree_et_al_2002} are a master forum within which actors compete for coverage of their issues \citep{amenta_et_al_2012}, which serves to identify and redefine, which can shape public perceptions of issues. Importantly, news organizations operate by a set of ``news values'' procedures that helps to identify what \textit{counts} as news \citep{amenta_et_al_2012,galtung_and_ruge_1965}, which not only affects the selection of topics to be covered \citep{galtung_and_ruge_1965} but also the ways in which these topics are covered.

The news values process necessarily selects on official or institutional news coverage \citep{schudson_2002,gitlin_1980,gans_1979} that tends to center on institutional political action and actors, because they are seen as newsworthy \citep{amenta_et_al_2012}. Relatedly, the economics of news media puts various pressures on the news to run stories that are not too gruesome, not too critical, and not too controversial. 

These and similar pressures lead to stories that do not venture too far from mainstream ideas and beliefs, so media coverage of an issue is likely to be close to public opinion about that issue \citep{gamson_and_modigliani_1989}. News can also influence public opinion. For example, \citep{gamson_and_modigliani_1989} find that the appearance of new frames around nuclear energy in news coverage influenced public discussion of the nuclear energy issues, which ultimately shapes public opinion about the issue.

Yet, organizations and other actors in the environment also have the ability to shape these frames. Recent research on the cultural consequences of social movements \citep{earl_2004} finds that organizations can impact public conversation about issues \citep{bail_et_al_2017} and initiate discursive change by offering their own diagnoses of and solutions to problems \citep{bail_2012,snow_et_al_2007,benford_and_snow_2000}. By injecting new frames into the broader discursive environment, organizations can shape the evolution of discourse. Although frames that `fit' the broader discursive environment \citep{mccammon_et_al_2007} or those that articulate widespread beliefs usually win out \citep{mccammon_et_al_2001,snow_et_al_2007,gamson_and_modigliani_1989}, alternative or fringe frames have the ability to alter discourse on a topic \citep{bail_2012}.

These studies suggest, then, that news media is not likely to (but can) feature media frames that are from the fringe of public discourse, but the media frames they do feature have an impact on public opinion or support for an issue. 

\subsection{Advocacy Organizations}

Perhaps marijuana advocacy organizations influenced marijuana policy change by way of being covered about the marijuana issue. Scholarly attention has traditionally focused on how the news media covers advocacy organizations \citep{amenta_et_al_2009,andrews_and_caren_2010}. Understanding this process is an important first step, as gaining coverage may be necessary to influence the public discussion of an issue. Coverage gives organizations an opportunity to inject new framing packages into the media arena, potentially changing the conversation around an issue \citep{gamson_and_modigliani_1989}. Although research on social movements typically focuses on the impact of protest on coverage \citep{earl_et_al_2004,oliver_and_myers_1999}, marijuana advocacy organizations rarely engaged in the type of protest necessary to gain coverage. Coverage of organizations may provide better opportunities for influence the discussion of an issue. 


\subsection{Structural Contexts}

Perceptions of and support for political issues results from the contexts within which people are embedded \citep{blau_1977a,blau_1977b,mcveigh_and_diaz_2009}. Importantly, the distribution of people and their interests across space can shape aggregate support for progressive policy change. \citet{mcveigh_et_al_2014}, for example, find that the distribution of the highly educated had implications for support for conservative mobilization. Structural contexts consists of the presence or absence of certain attributes \citep{blau_1977a,blau_1977b,blau_and_duncan_1967}, as well as their spatial spread across a local environment. The impact of social structure on policy change can also involve the presence or absence of advocacy organizations \citep{vannjr_2018,soule_and_olzak_2004}, aspects of the lived environment \citep{olzak_and_soule_2009} as well as levels of segregation within local contexts \citep{andrews_and_seguin_2015,olzak_et_al_1994}. 







While any one of the previous perspectives may provide a better explanation for the changes in the passage of marijuana legalization than other perspectives, they do no operate in a vacuum. Each of the processes described by the perspectives are likely to influence and be influenced by the processes of other perspectives. For example, amenable political contexts may lead to additional coverage of marijuana, but the nature of this coverage is likely to be influenced by policy reform mechanisms. My analyses take into account these processes.


\section{Data}


To explain my puzzle, I collected a variety of data. Importantly, my dependent variables include data on marijuana initiative voting and on newspaper coverage of marijuana from 1990 to 2016. The battle for marijuana legalization only recently shifted to state governments. Therefore, I rely on voting data from ballot initiatives\footnote{rather than state legislative hearing date because these hearings were few and far between} and newspaper articles about marijuana. From 1990 to 2016, states with the provision of the ballot initiative became sites of policy change on the marijuana issue. Thus, I use data from citizen/voter initiatives as measures of support for legalization. These data come from the Secretary of State websites for each state. 

%%%%%%%%%
%%%%USE THIS
%I searched the ProQuest archives for mentions of ``marijuana.''  This resulted in 5,893 articles across 100 newspapers. I chose these sources because they are widely circulated, represent both geographic diversity, and can be mapped onto to counties and states, which allows me to geolocate discourse about marijuana over time. I also searched the ProQuest newspaper archives for mentions of one of the four major marijuana advocacy organizations. In order to be included, the organization should be politically oriented with the goal of progressive marijuana policy change (e.g. decriminalization, medicalization, and legalization of marijuana), which excludes organizations like the NAACP, which does not primarily engage in marijuana advocacy. Using a list of search terms generated for each organization, I identified 787 articles, across 62 newspapers, mentioning at least one of these organizations. 










%To collect data on the content of the coverage, I created a 0.5% sample per year from the coverage of homosexuality, with a minimum of five articles included in each year. I supplemented this with a 2% sample of the coverage of LGBT and AIDS organizations. This resulted in 720 articles to code. I coded information about the article including the author, the occasion for coverage, and the type of article. I coded information about any social movement organization mentioned in the article, both LGBT and AIDS, and nonLGBT organizations. I also coded every paragraph that mentioned homosexuality in some way, including information on the speaker, the valence of the paragraph, and what terms and claims appeared. I discuss these measures in more detail where relevant in the following chapters. 


\section{Summary of Dissertation}


The rest of the dissertation is laid out in the following way. In chapter 2, I explore the amount and type of coverage newspapers gave to marijuana, from 1990 to 2016. I do so by considering the various frames associated with marijuana. Specifically, I separate the analysis into coverage of marijuana alone and coverage that also included marijuana advocacy organizations. I find that during the early years, marijuana coverage was mostly negative, and centered on a variety of frames. In later years, coverage became less negative (more neutral and positive) and frames shifted towards discussions of politics and revenue creation associated with marijuana legalization. This suggests that when controversial political issues enter public discourse by being put on the political agenda, a number of frames will be juxtaposed with one another and compete for dominance. But, given the controversial nature of these issues, as coverage of these issues matures, to maintain coverage, they must be linked to or framed as non-controversial, ``traditionalist,'' or quotidian \citep{snow_et_al_1998} behavior. 


Next, in chapter 3 I analyze the rate of adoption of legalization. I use event history analysis to investigate what explains the rate of adoption of legalization across the American states, from 1990 to 2016. I find that positive coverage on marijuana increases the rate of passage, particularly in states with the ballot initiative. I also find that Democratic support and political competition increase the rate of legalization. Moreover, prior histories with marijuana in a state, by way of medicalization, increases the rate at which states adopt legalization over time. 

Chapter 4 takes a structural approach to understanding legalization. Using ordinary least squares analysis of county-level support for legalization initiatives between 2000 and 2016, I examine how local level factors that contribute to increased voter support for policy change. I find that parental segregation and (the absence of) inequality ultimately create contexts within which marijuana legalization is viewed as non-threatening to the community. 

Finally, Chapter 5 concludes the project, summarizing how each perspective contributed to my explanation of the outcomes investigated in chapters 2--4. I argue that discourse about marijuana became less negative as a result of framing shifts. This shift in framing ultimately helped to speed up the process of legalization across the U.S. Yet, at the local level, segregation and inequality play a role in aggregate support for legalization. I discuss limitations of this project, as well as directions for future study.



%%% Local Variables: ***
%%% mode: latex ***
%%% TeX-master: "thesis.tex" ***
%%% End: ***
