\chapter{Introduction}
\bibpunct[:]{(}{)}{;}{a}{}{,} % new punctuation for these, replace parentheses before and after citealt, and replace citealt with citep

\begin{quotation}
\begin{singlespace}
\noindent {\footnotesize The sex fiend is a progressive criminal. He often begins with annoyances. He progresses to the sending of obscene letters, or exhibitionism. One finds him ``annoying'' children, or following women. For all these things, he often is merely fined, or given ``orders to leave town,'' or punished by a short jail sentences -- none of which deters him in the slightest degree from other and more serious offenses. And every sex criminal is a potential murderer.\newline

%\noindent Therefore, it would seem that the problem of the sex criminal is being mishandled... \newline

\noindent It should be determined, for instance, to what extent the recently widespread use of marijuana, or American hashish, has been responsible for the sex crime. \newline

\noindent Thus, a tremendous force may now be exerted toward the eradication of a drug which violently affects sex impulses.}\footnote{Hoover, J Edgar. September 26, 1937. ``WAR ON THE SEX CRIMINAL!'' \textit{Los Angeles Times}.}
\end{singlespace}
\end{quotation}
The above quote, appearing in the \textit{Los Angeles Times} in the fall of 1937, marked the beginning of the political fight, led by Harry J. Anslinger, to eradicate marijuana from America. Journalistic accounts like these, resulted in marijuana being conflated with fear. Terms like criminal, immoral, and deviant were often used, and it was assumed that marijuana was a danger to society. These linkages of marijuana with fear and danger ultimately resulted in the Marihuana Tax Act of 1937, which outlawed the possession, sale, use use of marijuana in the United States. 

While early discussions of marijuana were usually negative, nearly half a century later, the tone of coverage to has begun to change:
\begin{quotation}
\begin{singlespace}
\noindent {\footnotesize In 1998, President Bill Clinton signed a provision that made people temporarily or permanently ineligible for federal financial aid depending on how many times they had been arrested and convicted of a drug offense... the effect was real and devastating: the people most in need of financial aid were also being the most targeted for marijuana arrests and were therefore the most at risk of being frozen out of higher education. \newline

\noindent Why would Democrats support a program that has such a deleterious effect on their most loyal constituencies? The fact that they are ruining the lives of hundreds of thousands of black and Hispanic men... barely seems to register. \newline

%\noindent This is outrageous and immoral and the Democrat's complicity is unconscionable...

\noindent It is clear that criminalizing it has made it a life-ruining racial weapon. When will politicians have the courage to stand up, acknowledge this fact and stop allowing young minority men to be collateral damage?}\footnote{Blow, Charles M. October 23, 2010. ``Smoke and Horrors'' \textit{New York Times}.}
\end{singlespace}
\end{quotation}

%Some of the most consequential legislation in this state's history has been enacted this way: most famously, Proposition 13, which put a cap on property tax increases and required a two-thirds vote by the Legislature to raise taxes.

%For anyone with money and political savvy, the ballot initiative has been an effective way to work around a balky Legislature. And the only way to change or repeal an initiative is to go back to voters.

%"Why try to pass something in the Legislature when you have the money to get something on the ballot and make it that much more difficult to change it later," said Joe Mathews, a longtime critic of the California governance system. "This serves people who have the money and have the power."

As the example above illustrates, by the end of the twentieth century, editorials criticizing the criminalization and prohibition of marijuana were more frequent, as were those that encouraged the legalization of marijuana  \citep{mosher_and_akin_2019}. %When covered, marijuana began to be associated with medicine, social justice, and civil rights. 
What is more, the politics of marijuana had begun to change. According to Gallup polls, public support in regards to marijuana for medical purposes had was around 75 percent of Americans, and around one-third supported marijuana legalization for recreation \citep{gallup_2001,gallup_2003}. By 2013, American public support for marijuana legalization surpassed 50 percent \citep{gallup_2013}. Relatedly, despite federal prohibition, the issue was taken up in several states by way of the ballot initiative. Most noticeably, while medical marijuana initiatives took off in the 1990s and 2000s, California's 1996 Proposition 215 was an early success story of the citizen initiative-derived changes in marijuana policy.

What is puzzling about the case of marijuana is the disparity between increasing efforts to legalize, purportedly to increase access to those who most need or desire to use it, in the face of such a small group of Americans who actually use the substance. According to Gallup data, as of 2013, only seven percent of Americans reported regular use of marijuana, and had risen to only 13 percent by 2016 \citep{gallup_2016}. Why, then, have efforts to legalize increased?

This dissertation focuses on how marijuana legalization as an issue has evolved between 1990 and 2016. Understanding marijuana policy change is important for substantive reasons described above. Yet, there are also theoretical reasons to understand how and why this type of policy has continued to gain traction in the States. Marijuana legalization, for one, cannot be categorized easily into traditional policy typologies. As I explain below, we may need to think of the ways in which policies can be fluid, and may be more or less likely to gain support as they become linked with traditional policy types. 


\section{What Type of Policy is Marijuana Legalization}

A great deal of the public policy literature focuses on categorizing policies. Most of this work stems from \citet{lowi_1964,lowi_1969}, which argues that policy issues have their own political structures. In this oft-cited work, \citet{lowi_1964} provides three ideal-types of policy -- distributive, redistributive, and regulatory. Distributive policies are those that relate to state spending for public works, such as buildings, bridges, highways, and the like. These policies often operate in the form of taxes, where costs are bared by many people, and these funds are allocated towards projects that also benefit many people. Redistributive policies, on the other hand, often encounter fierce resistance by organized interests like business \citep{amenta_2000,amenta_and_elliott_2019}. Redistributive policies operate in the form of reallocation from one group to another, or from most groups and individuals to those most disadvantaged. Examples include social security, welfare, unemployment insurance, and policies that promote civil rights and equality \citep{hicks_1998,korpi_1978}. Finally, \citet{lowi_1964} includes the regulatory policy. From an economic perspective, regulatory policies impose restrictions and rules on the economy, such as setting minimum wages, work-related constraints, and water and air pollution standards. 

Beyond these classifications developed by \citet{lowi_1964}, scholars have identified similarly regulatory policies that define restrictions on personal conduct and behavior, known as morality policies. This outgrowth was the result of the \citet{lowi_1964} classifications centering on economic issues, and other political issues not being able to be classified within the traditional framework \citep{mooney_and_schuldt_2008}. Morality policies are those which involve conflicts over what is ``right'' and what is ``wrong'' along a ``particular set of values'' \citep[675]{mooney_1999}, and necessarily include advocates on each side of the conflict. Most notable among morality policies are those relating to abortion and homosexuality \citep{camobreco_and_barnello_2018,haider_markel_1999,haider_markel_and_meier_1996}, as well policies pornography \citep{brisbin_2001,smith_2001}, and gambling and state lotteries \citep{pierce_and_miller_2004,von_herrmann_2002,pierce_and_miller_1999,berry_and_berry_1990}. This new classification has enabled to scholars to categorize issues that failed to fully fit within the \citet{lowi_1964} framework. In fact, some scholars have attempted to create subcategories of morality policies, including those issues that relate to sexual behavior (e.g. gay marriage, pornography), addictive behavior (e.g. alcohol, drugs and tobacco, gambling, lotteries), life and death (e.g. abortion, death penalty), freedoms (e.g. gun control) \citep{mooney_2001}. 


However, in recent years, scholars have called into question the ability to categorize any policy as one of these ideal types \citep{greenberg_et_al_1977,roh_and_berry_2008,spitzer_1987,steinberger_1980}. And this problem persists with subcategorizing policies \textit{within} the morality framework -- for example, the issue of abortion, which is understood as a ``life and death'' issue, can also be viewed as a ``sexual behavior'' issue. Similarly, pornography can be simultaneously understood as both a ``sexual behavior'' and ``addictive behavior'' issue. In fact, as \citet{meier_2001} argues, many morality policies are multidimensional across larger policy typologies, and cannot be simply classified as solely a morality type of policy \citep{spitzer_1995}. \citet{roh_and_berry_2008}, for example, find that the issue of abortion is mainly a morality policy, but over time, increasingly included language that  centered on state \textit{funding} for abortions. This shift has led scholars to reclassify the abortion issue as both a morality as well as a redistributive policy -- strong restrictions on the ability of individuals to obtain an abortion, but money from one group of the population is used to fund abortions for another group \citep{roh_and_berry_2008,meier_and_mcfarlane_1993}. These findings suggest that morality policies often contain dimensions of one or more of the other three types of policies, and that this multidimensionality may be fluid over time.

Characteristic of morality policy is the role of organized interests. Indeed, passage of or changes in morality issues are often the result of public opinion \citep{geer_1996}, religious groups \citep{morgan_and_meier_1980,fairbanks_1977}, and advocacy organizations on both sides of the issue \citep{haider_markel_and_meier_1996,mooney_1999,roh_and_berry_2008}. To varying degrees, changes in morality policies can result from a combination of public opinion, religious groups, and advocacy organizations. For example, research on same-sex marriage has found that passage occurred in places with higher numbers pro gay social movement organizations, supportive public opinion, and places with fewer evangelicals \citep{baunach_2012,soule_2004}. While instructive for understanding changes in marijuana policy, it is important to identify some similarities and distinctions between marijuana as a morality issue, and other kinds of morality issues. 


Research on alcohol policy provides insight into conditions that may make particular morality policies more or less likely to ``succeed'' or ``fail.'' From the formation of the Union, through the turn of the \nth{20} Century, alcohol constituted an enormous portion of social and economic life in the American States \citep{aaron_and_musto_1981}. Yet, by the 1820s, temperance organizations had mobilized a large national membership base, often centered on religious values \citep{gusfield_1963,skocpol_et_al_2000,andrews_and_seguin_2015,young_2002,beisel_1997}, that attempted to prohibit alcohol consumption. Temperance organizations aimed their efforts at converting Church congregations towards their cause, and worked with state and county political apparatuses to pursue national prohibition \citep{blocker_1989}. This example illustrates the tremendous pressure required to gain support for prohibition -- advocates had to overcome both the widespread social value of drinking, as well as convince policymakers  and the public of the negative consequences of drinking, irrespective of all of the economic benefits alcohol provided. In the end, this effort was a key factor in the proposal, ratification, and passage of the Eighteenth Amendment to the U.S. Constitution. 


With these frameworks in mind, we get a sense of how policy change occurs. Based on the aforementioned literature, public opinion is critical for policy change. This may explain why a great deal, when public opinion supported the consumption of alcohol, the vast mobilization of temperance groups was required to overcome the widespread use and support of drink to push for prohibition. Many scholars of same-sex marriage and abortion rights find that policy change resulted from the combination of mobilization and rising public opinion \citep{rohlinger_2002,zucker_1999,carmines_and_woods_2002,halfmann_2011,schnabel_and_sevell_2017}, opinion which may have also resulted from mobilization. Some morality issues, however, have small beneficiary groups and, therefore, often lack mobilization around the issue. Issues like gambling and lotteries have experienced increasing support, yet some locales still oppose their expansion. Given that few people engage in lotteries or gambling, what explains the adoption of these kinds of policies? Recent observers have found that, for instance, state lotteries are adopted more rapidly when they are framed as a means to generate revenue for various programs and services \citep{mikesell_and_zorn_1986,berry_and_berry_1990}. What these works imply is that, when a morality policy issue lacks mobilization, reframing the issue to align more closely with ``distributive'' policy goals -- a policy that imposes taxes and fees on goods and services for the purpose of widespread use \citep{lowi_1964} -- may yield increasing success in terms of its adoption.




Given these frameworks, I make three main claims about the adoption and support of legalization. First, given that classifications of morality issues are fluid, I argue that the marijuana issue similarly shifted in its framing over time. Second, I argue that morality issues will have greater success when, over time, their characteristics become more closely aligned with traditionalist policy typologies. That is, when morality policies can signal their potential to generate (rather than require) revenue for other programs and services, it may be more likely to be adopted. For the adoption of marijuana legalization, I argue that reframing the issue away from ``morality'' and as a ``distributive'' policy, will increase the policy's rate of passage. Finally, the aforementioned work on morality policy highlights the role of antagonistic groups that view the issue as either good or bad. Therefore, third, I argue that, when these oppositional groups are separated from one another, we are likely to see increased support for legalization.
This project aims to fill some of these gaps by explaining how and why changes in the cultural, political, and structural landscape have contributed to changes marijuana policy in since 1990.







%Importantly, understanding marijuana policy change is important for substantive discussions about progressive policy change and the set of cases to which it belongs. Marijuana legalization is an interesting case of progressive policy that bears some similarity to  contentious and morality-related policies such as abortion liberalization or, to a greater extent, tobacco restrictions, anti-prohibition \citep{gusfield_1963,andrews_and_seguin_2015} and lotteries \citep{pierce_and_miller_1999,berry_and_berry_1990}, which are often opposed for their perceived impacts on youth \citep{beisel_1997}. Marijuana legalization is similar to alcohol prohibition/anti-prohibition and lottery battles in that their perceived negative consequences and costs will only be incurred by those who make use of the policy \citep{mikesell_and_zorn_1986,berry_and_berry_1990}. Marijuana legalization differs, however, from anti-prohibition policy change because of alcohol's longstanding social and economic impact on American society \citep{andrews_and_seguin_2015}. Only now are we beginning to understand the potential economic benefits that could result from marijuana legalization, and public opinion polls indicate increasing marijuana use over time \citep{gallup_2013}. Marijuana legalization thus bears some similarity to taxation policy \citep{amenta_and_halfmann_2000} in that there are perceived economic benefits to policy change, yet longstanding opposition from citizens and powerful interests \citep{amenta_and_elliott_2019}. Finally, marijuana legalization can also be thought of as similar to, but a reversed trend of, tobacco policy. Tobacco policy, much like alcohol policy has been built into the social and economic fabric of the United States. However, in recent years, there have been increasing attempts to restrict tobacco use through increased taxation, removal from retail locations, and elimination in public use spaces -- which may be due to increasingly negative connotations and beliefs attached to smoking. On the contrary, over time, marijuana has become increasingly associated with medicinal uses for patients as well as economic benefits for locales. 


 %progressive policy change and the set of cases to which it belongs. Marijuana legalization bears some similarity to contentious and morality-related policies such as abortion rights, anti-prohibition, and lotteries \citep{gusfield_1963,andrews_and_seguin_2015,pierce_and_miller_1999,berry_and_berry_1990}. Marijuana legalization is similar to anti-prohibition and lotteries in that their perceived negative consequences and costs will only be incurred by those who make use of the policy \citep{mikesell_and_zorn_1986,berry_and_berry_1990}. Marijuana legalization differs, however, from anti-prohibition given alcohol's longstanding social and economic impact on American society \citep{andrews_and_seguin_2015}. Marijuana legalization also bears similarity to taxation policy \citep{amenta_and_halfmann_2000} in its generation of economic benefits. Finally, marijuana legalization has followed the reverse trend of tobacco policy. Tobacco use is tied to social and economic fabric of the United States. Yet, in recent years, has experienced some retrenchment --  due to increasingly negative connotations and beliefs attached to smoking. Marijuana, on the other hand, has become increasingly associated with medicinal and economic benefits. 



%This question is important because, as Ferree, Gamson, Gerhards, and Rucht (2002) observe, the media arena is the primary site for contests over meaning. How homosexuality is portrayed in the news media will have a powerful impact on on the larger public discussion of homosexuality. The news media is the main source of frame packages around an issue that get deployed in public conversations and when formulating public opinion (Gamson and Modigliani, 1989). Studying newspaper coverage of homosexuality can serve as a proxy for the larger public discussion over time, giving insight into how the general public understood homosexuality and how this changed over time. Additionally, changing the media portrayal of homosexuality is likely to lead to changes in the public discussion of homosexuality, so understanding how activists can change media portrayals of issues would contribute to understanding how movement can impact culture. 

%Social movements are an important part of the political landscape. They are vehicles by which people express grievances and work towards change. They have the potential to change both the political and cultural landscape. How they do this is still not well understood. Assessing how movements impact policy making and other political outcomes is complicated, making consensus on the mechanisms of influence elusive (Amenta and Caren, 2004). Even less well understood is how movements might have cultural impact. 




\section{A General History of Marijuana}

\citet{bonnie_and_whitebread_1970} provide an abbreviated legal and social history of marijuana in the United States, which I draw from here. The history of marijuana can be divided into four phases. First, between 1915 and 1930, was the period that grew out of the ascendancy of alcohol prohibition. The second phase ranges from about 1932 to 1937, when the drug was suppressed both nationally and subnationally. In the third phase, in the 1950s, there was an escalation of penalties. The final phase, from around 1965 and beyond, has been characterized by public debate about marijuana and legalization. 

During the first phase, public opinion was inoperative given that the group of people using was so small and inaccessible. Yet, in the absence of public opinion, lawmakers still worked to make marijuana illegal. There was no indication that legislators consulted scientific data on the drug, and instead relied on sensationalistic journalist and police accounts, which associated marijuana with crime. What is more, since marijuana was used mainly by immigrant Mexicans in the South and West, and by Blacks in the East, legislators moves to prohibit marijuana may have reflected public hostility to minority groups -- perhaps a primary force for prohibition -- without regard to pharmacological evidence. Marijuana legislation may have reflected public antipathy to any deviant tendency of new immigrant minorities, and legislators' desire to suppress or assimilate them. During this same time, marijuana prohibition occurred simultaneously with alcohol prohibition, which signals legislators preoccupation with intoxicants of any kind. 

By 1931, many states had enacted marijuana prohibition of their own, without regard to federal legislation. By 1932, however, the Uniform Narcotic Drug Act, which aimed to make all state legislation on narcotics similar, included prohibitions of marijuana, alongside prohibitions for opium, cocaine, and heroin. During this time, the Federal Bureau of Narcotics began an educational campaign against narcotic drugs, which included marijuana -- once passed, the Bureau put the full weight its propaganda machine on criminalizing marijuana -- generating a feeling amongst members of Congress that federal prohibition was necessary. Yet still, during this time, because so few people used the drug, public opinion on marijuana remained dormant while Congress shepherded in the Marijuana Tax Act of 1937. 

During the 1940s, public opinion on marijuana remained dormant, but as soldiers began to return from the War, they too, like those who had returned nearly 30 years prior, many were in need of euphoriants to deal with their pain and ailments. Yet, during the 1950s, there was a growth in the study of psychology of fear, which led to the repression of political and cultural deviation (e.g. McCarthyism). During this time, there was public interest in narcotics. The increase in drug abuse, especially amongst Veterans, increased the public's knowledge about marijuana. In the fifties, the Bureau continued to release propaganda on marijuana, and called for harsher penalties for those presumed morally and socially deviant. Congress then responded with the Boggs Act, which established mandatory minimum sentences, and fines. At the same time, the Bureau replaced the older narrative of ``marijuana as a producer of criminal activity'' with a new ``gateway drug'' narrative, that implied that marijuana use was simply a stepping stone towards use of harsher, more dangerous drugs. This propaganda, and work with congress to eliminate marijuana peaked in 1956 with the Narcotic Control Act. Therefore, by 1956, marijuana possession was a federal felony. 

Then the 1960s came along. As more and more middle class adults on college campuses experimented with the drug and experienced no deleterious effects, they encouraged others to do so. And by 1970, over 10 percent of the American middle class had experimented with marijuana -- in violation of federal law. Thus, by this time, public opinion on marijuana began to take shape. 

Thus, between 1915 and 1937, there was virtually no public opinion on marijuana. Few people used or even knew about the drug... and those who did where from minority groups. Therefore, the Bureau took advantage of racialized sentiments to criminalize the drug through prohibitive legislation as a way to preserve cultural homogeneity in America. In this way, the dominant group was unfamiliar with the activity and behavior of Mexican and Black minorities in the States. Yet, the Bureau was able to use legislation to prohibit their conduct (e.g. using marijuana) which created the presumption of immorality of minority groups because they were violating new laws. Moreover, many of these laws worked under the guise of ``protecting new immigrants and minority groups from inhibiting their own success in America.'' Yet, this idea of assimilation into the American economic and social system by way of restricted behavior was not only developed to stimulate success, but more importantly, protect the supposed superior White American way from what many would have called contamination. Succeeding generations were viewed as needing to be assimilated as quickly as possible for fear of posing a threat to the dominant order.The 1950s brought a tremendous expansion of penalties associated with use. Yet, in the 1960s and 1970s, we see a reversal -- the identities of these ``deviants'' changed. Now marijuana was no longer confined to immigrant Mexicans or working-class Blacks, but was mainly used on college campuses by middle class White students. At this same time, there was a crosscurrent in American culture that centered on growing values of individuality and privacy, which may have influenced a cultural shift in perceptions of marijuana -- that use is a private activity that should not be infringed upon. This is why, during this time, we may have also seen expansion of laws protecting individual freedoms in various aspects of private life, including those related to homosexuality, abortion, contraception, etc. 




According to some scholars, initial legislation around marijuana occurred in a vacuum \citep{bonnie_and_whitebread_1970}. In fact, prohibition occurred in a vacuum, such that prohibition was the result of a new public image of marijuana, followed by legislative action on marijuana prohibition, then followed by public opposition to marijuana. 

The first users of the drug were medical addicts: in the nineteenth century, Civil War hospitals used opium and morphine to deal with injuries, inciting a widespread epidemic of users addicted to the drugs. As more people began to use opium and heroin for pain, the use set grew, to what some have called ``street'' use. This set off a process wherein physicians, concerned with growing use, began to argue for stricter regulation of drugs. 

This all converged in the Harrison Act. In 1914, the Act, which was a taxation measure, required the registration and payment by persons importing, producing, or selling opium, cocaine, or their derivatives. For those not complying, penalties included fines of up to \$2,000 and/or five years of imprisonment. The passage of the Harrison Act fostered an image of the degenerate dope fiend with immoral proclivities, as all addicts, whether accidental or pleasure-seeking, were shut off from their supply and had to turn underground to purchase drugs. The price-inflated aspect of underground prices sometimes led users towards criminal activity, and this activity inevitably invoked in the public a hysteria around the user.


Until the inclusion of marijuana in the Uniform Narcotic Act of 1932 and the Marihuana Tax Act of 1937, there was no official policy regarding marijuana. From 1914 to 1931, most states had enacted narcotics laws that included restrictions on cannabis \citep{bonnie_and_whitebread_1970}. With the Harrison Act of 1914 came a shift in public perception of the narcotics addict -- which was now called a criminal ``dope fiend.'' This hysteria, along with the actual growth in drug-related criminal activity as the result of closing clinics led to many states including cannabis alongside other narcotics in their increasing number of narcotic statues aimed to deal with the growing narcotics problem \citep{terry_and_pellens_1928}. However, according to \citet{bonnie_and_whitebread_1970}, there was little evidence of public concern about, interest in, or opinion on marijuana. Yet, why did some states work to prohibit marijuana when it went relatively unnoticed by the public or legislators?

These scholars argue that the prominent reason for banning marijuana was racial prejudice, given that marijuana prohibition was largely concentrated in western and southern states, where increasing numbers of Mexican-Americans immigrants were inhabiting -- immigrants who were the primary users of the drug. Secondly, there was a widely expressed perception that marijuana was addictive, and could be used as a substitute for narcotics and alcohol. 


First, the rationale of the west was one that saw increases in Mexican immigration, and sizeable Mexican-American minorities. It is no wonder, then, that much like the Klan of the 1920s, whose support for policy change, namely focusing on liquor and beer, resulted from a desire curb the growth of Irish and Catholic immigrant populations \citep{mcveigh_2009,andrews_and_seguin_2015}, marijuana prohibition in various western states resulted from a desire to curb the growth of Mexican migration. 

In 1915, Utah became the first state to enact marijuana prohibition, yet there was little public attention. In fact, the combination of Mormons' opposition to euphoriants, and increasing Mexican immigration, led marijuana to be included in the omnibus narcotics bill. Texas and New Mexico also passed similar legislation in 1923, yet newspaper references to marijuana remained minimal, with the exception of coverage of these legislative actions. 

The same can be said for coverage in Montana and Colorado. However, beyond this, marijuana's racial undertones were made explicit in news coverage, where these messages were also linked to criminality. The public perception of marijuana's ethnic origins and crime producing tendencies went hand in hand through media coverage. These articles attempted to link marijuana with criminality by blaming marijuana as the reason for crimes committed by individuals.


For example, on April 16, 1929, printed a story of a girl who was murdered by her Mexican step-father. The story, reported in the \textit{Denver Post}, stated the following questioning of and narration about the step-father:

\begin{quotation}
\noindent ``You smoke marihuana?'' 
\noindent ``Yes'' 
\noindent  The Mexican said he had been without the weed for two days before the killing of his step-daughter.
\end{quotation}

Many of those who committed violent crimes attempted to blame their actions on the effects of marijuana, which ultimately resulted in the associated moniker of ``killer weed'' \citep{rusby_et_al_1930}. These ideas were rampant. In fact, \citet{hayes_and_bowery_1933} called for stricter penalties on marijuana, stating that during the exhilaration phase of the drug, the user is likely to have increased sexual desires and will commit acts of violence and murder.

In 1914, New York passed the Boylan Bill, which regulated the sale and use of habit-forming drugs, which initially excluded marijuana, but which was amended to include cannabis indica \citep{nyt_1914}. Within New York, the argument for prohibiting cannabis was preventive: given the limited access to alcohol after Prohibition, or to cocaine and opium after the passage of the Harrison Act, marijuana had to be prohibited to keep prior addicts from switching to it as a substitute \citep{bonnie_and_whitebread_1970}. The concern over marijuana was primarily related to a fear that its use would spread, especially amongst whites, as a substitutes for opiates and alcohol, which were now much harder to get. This concern was most noticeable in Western states, where there was tremendous growth in the Mexican American population. And the fear developed that marijuana use would increase amongst white youth. 

According to \citet{bonnie_and_whitebread_1970}, during the 1930s, public opinion against marijuana had not yet crystallized, although public policy was on that trajectory. During this time, prior sympathy towards those who had fallen victim to opiate addiction (most of whom were war veterans) had now turned into moral judgment. As a result of the Harrison Act, and thus, the closure of many hospitals and rehabilitation facilities, many people who were addicted to opiates were driven to criminal activity to sustain their habits. 


During this period, there was sufficient public support for drinking, especially amongst the middle class, which effectively led to the reversal of prohibition, yet, because users of marijuana were mainly economic and racial minorities, no such support was either felt or acted upon by political officials. As such, laws were quickly enacted during the 1930s that would prohibit marijuana, without much regard to public opinion. Thus, it is puzzling that marijuana would have even been prohibited.

Given the lack of uniformity in the states' laws on narcotics, there was interest in developing a ``Uniform Narcotic Drug Act'' that would implement these laws similarly. Importantly, these laws would establish rules for controlling interstate crime. In late 1927, the National Commissioners on Uniform State Laws (consisting of two representatives from each state, appointed by the state Governor), worked together to develop these anti-narcotic regulations. The first draft of which included cannabis as a habit forming drug, without explanation for its inclusion \citep{bonnie_and_whitebread_1970}. And by 1932, the National Conference of Commissioners was instructed that any state wishing to restrict the sale and possession of marijuana simply had to add it to the list of ``narcotic drugs'' \citep{terry_and_pellens_1928}. 

Accordingly, \citet{bonnie_and_whitebread_1970} report the following:

\begin{quotation}
\begin{singlespace}
\noindent {\footnotesize Use of the drug was still slight and confined to underprivileged or fringe groups who had no access either to public opinion or to the legislators. The middle class had little knowledge and even less interest in the drug and the legislation. Passage of the [Uniform Narcotics] Act in each state was attended by little publicity, no scientific study and even more blatant ethnic aspersions than the earlier laws. In short, the laws went unnoticed by legal commentators, the press and the public at large, despite the propagandizing efforts of the Bureau of Narcotics.}
\end{singlespace}
\end{quotation}

Again, marijuana use was often confined to the West and near the Mexican border, however, by the 1920s, use of the drug began to spread to many of the larger U.S. cities. The drug had become popular among jazz musicians and dancers. Yet, the the number of users was small and concentrated in the Southwest and West amongst lower class Mexican and Black communities. 

Prior to 1935, few of the population knew little, if anything, about marijuana. What little information was filtered to the middle class was generated by sporadic and sensationalist campaigns by local newspapers that played on ethnic prejudices, often associated with Mexicans, and being directly the result of unrestricted Mexican immigration\citep{bonnie_and_whitebread_1970}. As such, when testifying in favor of the Marihuana Tax Act of 1937, Commissioner Anslinger used these sensationalistic examples of attacks by minorities (supposedly) under the influence of marijuana, as evidence in favor of prohibition. In fact, many believe that while Anslinger began a moral crusade against the drug, using newspapers and law journals to boost support for the Act \citep{anslinger_1932}, others also assert that the Bureau may back believed its own propaganda on the link between criminality and dope fiends \citep{king_1953}. Although the Bureau is not the sole reason for prohibition against marijuana, it did quicken the pace of federal prohibition. 

According to their analysis of media coverage surrounding the state-level passage of the Act, \citet{bonnie_and_whitebread_1970} find that there was little-to-no media attention given to the possession, sale, and distribution of marijuana -- and therefore, no public outcry for prohibition. In addition, none of the states undertook independent investigation of the effects of marijuana, and instead relied on the reports of the Federal Bureau of Narcotics. 

Still, during this time, little was known about the effects of marijuana. Although the \citet{hemp_1894} had commissioned a report, the minimal effects of the drug were either ignored or not widely disseminated to the public. Many of the studies that \textit{were} conducted on marijuana during the 1930s were either (1) in obscure journals or (2) found marijuana relatively harmless, especially when compared to alcohol. Yet, none of these works were considered during the passage of the Act in the states. 

But why the Marihuana Tax Act of 1937? Given the piecemeal legislation enacted by various states under the Uniform Narcotic Act, there was little uniformity in the federal enforcement of those Acts. As such, enforcement difficulty and public hysteria (via news accounts of marijuana's harms) were reasons for federal action.  Many states where marijuana use had become more common were dealing with the problem effectively \citep{bonnie_and_whitebread_1970}. And a federal law would not have helped enforcement at the state and local level, that was nevertheless the justification presented by Anslinger for the Marihuana Tax Act. Thus, whatever publicity the marijuana ``problem'' received during the 1930s was attributable to Commissioner Anslinger and the Bureau's campaign that disseminating propaganda. 

In addition, during the Tax Act Hearings \citep{tax_1937} in front of the \nth{75} Congress, Anslinger relied on horror story accounts of criminal activity by those under the influence of marijuana, studies linking the drug to the population of inmates in Louisiana jails, and experimentation on dogs -- as justification for the Tax Act. In the end, after five days of hearings, the H.R. 6385 was redrafted as 6906, and was confirmed on May 11, 1937. This act was signed into law on August 2, 1937 by President Franklin D. Roosevelt. Under the Act, possession of marijuana, plus failure to produce the required tax stamp documentation was evidence of criminal activity. The only way in which someone could possess marijuana was to purchase tax stamps that enabled the possession and sale. 

Boggs Act:

Mandatory Minimums:



There was a dramatic increase in marijuana use during the 1960s. There was a subsequent growth in the prosecution of individuals using the drug, which therefore increased visibility of the drug ``problem.'' The latter part of the 1960s saw a growth in dissent against political and legal systems, as more people began to defy the law. Over time, more people arrived at the courts to question the longstanding prohibitions against individuals' private decisions such as abortion, contraception, drugs, and homosexuality. 


\section{Motivating Questions for Substantive Chapters}





\section{Explanations for Policy Change}

What are possible explanations for changes in marijuana policy? I explore five perspectives to explain the legalization of marijuana. These five perspectives guide my analysis throughout this dissertation. In the conclusion, I evaluate how well each perspective helped in explaining my puzzle.


%\subsection{Public Opinion}

%Scholars have long studied the relationship between public opinion and policy change \citep{burstein_and_linton_2002}. The evidence suggests that the relationship is recursive. 

%Public opinion shifts when previously unused frames enter the media. These frames are in turn used in public discussions and are incorporated into personal formulations of opinion on an issue. Brewer (2003) explores how political knowledge mediates the impact of media frames on public opinion, and finds that undisputed media frames are more likely to influence public opinion than disputed frames.



\subsection{Political Institutional Contexts}

One possible explanation for the shifts in marijuana policy is related to political institutional contexts. Certain contexts provide more opportunities for policy change than others. Political context theories often focus on the role of elite allies in government that may contribute to the overall success of a policy change \citep{amenta_et_al_1994,amenta_2006}. Research in political sociology and political science has focused on the role of left- or reform-oriented parties in the passage or adoption of progressive policies \citep{amenta_and_elliott_2019,amenta_et_al_2005,korpi_1983}. Moreover, whether or not elite allies hold majorities in government is important for the likelihood for policy change\citep{ansolabehere_and_snyder2006,winters_1976,abramowitz_1983,campagna_and_grofman_1990}. And because political officials as allies want to make good on their promises, they will often support or sponsor policies that accord with the interests of their constituents \citep{page_and_shapiro_1983,mayhew_1974,downs_1957,stimson_et_al_1995}. 

Voting results are also an important part of the political institutional environment. Support for specific parties signal to both constituents and political officials that certain types of policy change are possible or not \citep{amenta_and_elliott_2019}. These characteristics pique the interests of politicians and constituents to support specific policy changes \citep{berry_and_berry_1990,boushey_2016}. 

Public opinion is also an important part of the political institutional environment \citep{burstein_1998,burstein_2003}. Similarly, public opinion serves as a signal about what sorts of policies are possible. Therefore, public opinion reveals the saliency of political issues for constituents as well as political officials \citep{pacheco_2012,nicholson-crotty_2009}.



%What exactly counts as political opportunity is not well defined (Goodwin, Jasper, and Khattra, 1999). It has been used broadly to include virtually all macro-level effects on movement mobilization and outcomes. In an international comparative context, political opportunity refers to the openness or closedness of a political system (Kriesi, 2006). In the single nation context, political opportunity refers to the configuration of allies, antagonists and bystanders. Scholars have tended to interpret this as how friendly political institutions are at any given time, as well as whether public opinion is on the movement's side. 

%The political opportunities that lead to mobilization and policy success may also lead to increased and better media coverage of movements. Just as political opportunities can signal to movements that the time is right for mobilization, political opportunities could send signals to media professionals to pay more attention to an issue, or to treat an issue differently. These signals could include the election or appointment of new politicians, or the passage of new reforms regarding an issue. Political opportunities also influence movement's tactics in the media arena (Rohlinger, 2006), which could, in turn, influence the coverage of the movement's issue. Previous research has found mixed results for political opportunity?s influence of news coverage of movement organizations, as it seems to influence coverage in partisan news outlets, but not in mainstream outlets (Amenta, Caren, and Stobaugh, 2012; Rohlinger, Kail, Taylor, and Conn, 2012).


\subsection{Policy Feedback}


An alternative explanation related to the political context involves the legacy of policy reforms or policy feedback around an issue. Drawing on ideas of increasing returns \citep{pierson_1996,pierson_2000}, policy reforms will have both immediate, as well as long lasting benefits for future policy change \citep{amenta_et_al_2012}. Policies often lead to institutionalized benefits to a specific group of people. These benefits can provide a boost in the efficacy of future advocacy on the part of beneficiaries \citep{amenta_and_caren_2004}. 


%If the group is represented by a movement, organizations affiliated with this movement can gain an additional benefit in the form of acceptance (Amenta and Caren, 2004). Policy reforms may cast organizations that represent the constituencies benefiting from the reforms in a more legitimate light. This could lead to the organizations having access to more and better resources, enabling them to be more effective in future advocacy. Furthermore, this increased legitimacy can lead to members of the movement being placed in positions of authority within the state. %As an example, AIDS activists were appointed to a Presidential task force on AIDS in the 1990s (Epstein, 2007). These members were then able to more directly influence policy on their issues, further improving its policy profile. 


%Applying this to the media arena, the political process is typically a highly newsworthy process, most newspapers assign one or more beats to covering politics. Movements and their actors are likely to be covered if a policy they are advocating for is making its way through the political process. If the policy is successful, the movement may be seen as more legitimate not only by state actors, but also the media, leading to further coverage, especially as the policy is implemented. This effect will be cumulative, with more policy successes leading to more legitimacy in the media arena. Additionally, if a movement actor is appointed to a state position, they will receive the benefit all state actors receive in the media arena, with much easier access and more frequent coverage by the media (Gans, 1979; Oliver and Maney, 2000; Sobieraj, 2010; Tuchman, 1978).


%\subsection{Crisis}


%Another possible explanation for the change in coverage of homosexuality concerns crises (Snow, Cress, Downey, and Jones, 1998). Crises are typically highly newsworthy, having several characteristics journalists look for in deciding what's news (Gans, 1979). Crises are unusual, outside of the routine of daily life. Crises typically impact a large number of people. Crises also usually involve the government trying to solve the crisis. Newspapers are likely, then, to devote a fair amount of coverage to crises. Movements may be able to benefit from this coverage if they are affiliated with an issue impacted by the crisis, and crises can spawn entirely new movements as those affected mobilize themselves (Snow et al., 1998). 

%The partial meltdown at the nuclear reactor at Three Mile Island is an example of this type of crisis. Local pre-existing anti-nuclear organizations experienced a tremendous surge in members after the accident, and new organizations were founded in communities closest to Three Mile Island to oppose the restart of the Unit 1 reactor and impose strict regulations on the cleanup of Unit 2 (Cable, Walsh, and Warland, 1988). The Three Mile Island accident also introduced new framing packages that significantly shifted public opinion against nuclear power (Gamson and Modigliani, 1989). 

%Crises may also undermine the authority of the status quo, bringing attention to deficiencies of the dominant groups (Bail, 2012). Especially if the government is seen as contributing to the disaster, or negligent in addressing the disaster, official authorities may lose legitimacy on the issue (Snow et al., 1998). The news may turn to movements in these cases in the search for new experts. Movements could potentially be considered alternative experts on the issue, or they may be used as ``authentic'' responses to the crisis Sobieraj (2010). Either way, movements can expect to receive increased coverage in the wake of a crisis that involves an issue they are mobilized around. This coverage offers movements the opportunity to inject new ways to think about the issue, and the crisis may have destabilized the dominant discourse in ways that make the movement especially influential in this area.


\subsection{Cultural Context/Discourse}

Mass media are central for making sense of relevant events \citep{gamson_and_modigliani_1989}. According to \citet{ferree_et_al_2002} are a master forum within which actors compete for coverage of their issues \citep{amenta_et_al_2012}, which serves to identify and redefine, which can shape public perceptions of issues. Importantly, news organizations operate by a set of ``news values'' procedures that helps to identify what \textit{counts} as news \citep{amenta_et_al_2012,galtung_and_ruge_1965}, which not only affects the selection of topics to be covered \citep{galtung_and_ruge_1965} but also the ways in which these topics are covered.

The news values process necessarily selects on official or institutional news coverage \citep{schudson_2002,gitlin_1980,gans_1979} that tends to center on institutional political action and actors, because they are seen as newsworthy \citep{amenta_et_al_2012}. Relatedly, the economics of news media puts various pressures on the news to run stories that are not too gruesome, not too critical, and not too controversial. 

These and similar pressures lead to stories that do not venture too far from mainstream ideas and beliefs, so media coverage of an issue is likely to be close to public opinion about that issue \citep{gamson_and_modigliani_1989}. News can also influence public opinion. For example, \citep{gamson_and_modigliani_1989} find that the appearance of new frames around nuclear energy in news coverage influenced public discussion of the nuclear energy issues, which ultimately shapes public opinion about the issue.

Yet, organizations and other actors in the environment also have the ability to shape these frames. Recent research on the cultural consequences of social movements \citep{earl_2004} finds that organizations can impact public conversation about issues \citep{bail_et_al_2017} and initiate discursive change by offering their own diagnoses of and solutions to problems \citep{bail_2012,snow_et_al_2007,benford_and_snow_2000}. By injecting new frames into the broader discursive environment, organizations can shape the evolution of discourse. Although frames that `fit' the broader discursive environment \citep{mccammon_et_al_2007} or those that articulate widespread beliefs usually win out \citep{mccammon_et_al_2001,snow_et_al_2007,gamson_and_modigliani_1989}, alternative or fringe frames have the ability to alter discourse on a topic \citep{bail_2012}.

These studies suggest, then, that news media is not likely to (but can) feature media frames that are from the fringe of public discourse, but the media frames they do feature have an impact on public opinion or support for an issue. 

\subsection{Advocacy Organizations}

Perhaps marijuana advocacy organizations influenced marijuana policy change by way of being covered about the marijuana issue. Scholarly attention has traditionally focused on how the news media covers advocacy organizations \citep{amenta_et_al_2009,andrews_and_caren_2010}. Understanding this process is an important first step, as gaining coverage may be necessary to influence the public discussion of an issue. Coverage gives organizations an opportunity to inject new framing packages into the media arena, potentially changing the conversation around an issue \citep{gamson_and_modigliani_1989}. Although research on social movements typically focuses on the impact of protest on coverage \citep{earl_et_al_2004,oliver_and_myers_1999}, marijuana advocacy organizations rarely engaged in the type of protest necessary to gain coverage. Coverage of organizations may provide better opportunities for influence the discussion of an issue. 


\subsection{Structural Contexts}

Perceptions of and support for political issues results from the contexts within which people are embedded \citep{blau_1977a,blau_1977b,mcveigh_and_diaz_2009}. Importantly, the distribution of people and their interests across space can shape aggregate support for progressive policy change. \citet{mcveigh_et_al_2014a}, for example, find that the distribution of the highly educated had implications for support for conservative mobilization. Structural contexts consists of the presence or absence of certain attributes \citep{blau_1977a,blau_1977b,blau_and_duncan_1967}, as well as their spatial spread across a local environment. The impact of social structure on policy change can also involve the presence or absence of advocacy organizations \citep{vann_jr_2018,soule_and_olzak_2004}, aspects of the lived environment \citep{olzak_and_soule_2009} as well as levels of segregation within local contexts \citep{andrews_and_seguin_2015,olzak_et_al_1994}. 







While any one of the previous perspectives may provide a better explanation for the changes in the passage of marijuana legalization than other perspectives, they do no operate in a vacuum. Each of the processes described by the perspectives are likely to influence and be influenced by the processes of other perspectives. For example, amenable political contexts may lead to additional coverage of marijuana, but the nature of this coverage is likely to be influenced by policy reform mechanisms. My analyses take into account these processes.


\section{Data}


To explain my puzzle, I collected a variety of data. Importantly, my dependent variables include data on marijuana initiative voting and on newspaper coverage of marijuana from 1990 to 2016. The battle for marijuana legalization only recently shifted to state governments. Therefore, I rely on voting data from ballot initiatives\footnote{rather than state legislative hearing date because these hearings were few and far between} and newspaper articles about marijuana. From 1990 to 2016, states with the provision of the ballot initiative became sites of policy change on the marijuana issue. Thus, I use data from citizen/voter initiatives as measures of support for legalization. These data come from the Secretary of State websites for each state. 

%%%%%%%%%
%%%%USE THIS
%I searched the ProQuest archives for mentions of ``marijuana.''  This resulted in 5,893 articles across 100 newspapers. I chose these sources because they are widely circulated, represent both geographic diversity, and can be mapped onto to counties and states, which allows me to geolocate discourse about marijuana over time. I also searched the ProQuest newspaper archives for mentions of one of the four major marijuana advocacy organizations. In order to be included, the organization should be politically oriented with the goal of progressive marijuana policy change (e.g. decriminalization, medicalization, and legalization of marijuana), which excludes organizations like the NAACP, which does not primarily engage in marijuana advocacy. Using a list of search terms generated for each organization, I identified 787 articles, across 62 newspapers, mentioning at least one of these organizations. 










%To collect data on the content of the coverage, I created a 0.5% sample per year from the coverage of homosexuality, with a minimum of five articles included in each year. I supplemented this with a 2% sample of the coverage of LGBT and AIDS organizations. This resulted in 720 articles to code. I coded information about the article including the author, the occasion for coverage, and the type of article. I coded information about any social movement organization mentioned in the article, both LGBT and AIDS, and nonLGBT organizations. I also coded every paragraph that mentioned homosexuality in some way, including information on the speaker, the valence of the paragraph, and what terms and claims appeared. I discuss these measures in more detail where relevant in the following chapters. 


\section{Summary of Dissertation}


The rest of the dissertation is laid out in the following way. In chapter 2, I explore the amount and type of coverage newspapers gave to marijuana, from 1990 to 2016. I do so by considering the various frames associated with marijuana. Specifically, I separate the analysis into coverage of marijuana alone and coverage that also included marijuana advocacy organizations. I find that during the early years, marijuana coverage was mostly negative, and centered on a variety of frames. In later years, coverage became less negative (more neutral and positive) and frames shifted towards discussions of politics and revenue creation associated with marijuana legalization. This suggests that when controversial political issues enter public discourse by being put on the political agenda, a number of frames will be juxtaposed with one another and compete for dominance. But, given the controversial nature of these issues, as coverage of these issues matures, to maintain coverage, they must be linked to or framed as non-controversial, ``traditionalist,'' or quotidian \citep{snow_et_al_1998} behavior. 


Next, in chapter 3 I analyze the rate of adoption of legalization. I use event history analysis to investigate what explains the rate of adoption of legalization across the American states, from 1990 to 2016. I find that positive coverage on marijuana increases the rate of passage, particularly in states with the ballot initiative. I also find that Democratic support and political competition increase the rate of legalization. Moreover, prior histories with marijuana in a state, by way of medicalization, increases the rate at which states adopt legalization over time. 

Chapter 4 takes a structural approach to understanding legalization. Using ordinary least squares analysis of county-level support for legalization initiatives between 2000 and 2016, I examine how local level factors that contribute to increased voter support for policy change. I find that parental segregation and (the absence of) inequality ultimately create contexts within which marijuana legalization is viewed as non-threatening to the community. 

Finally, Chapter 5 concludes the project, summarizing how each perspective contributed to my explanation of the outcomes investigated in chapters 2--4. I argue that discourse about marijuana became less negative as a result of framing shifts. This shift in framing ultimately helped to speed up the process of legalization across the U.S. Yet, at the local level, segregation and inequality play a role in aggregate support for legalization. I discuss limitations of this project, as well as directions for future study.



%%% Local Variables: ***
%%% mode: latex ***
%%% TeX-master: "thesis.tex" ***
%%% End: ***
