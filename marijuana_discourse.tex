
\chapter{The Discursive Shift on Marijuana Legalization}



%\begin{abstract}
%\begin{singlespace}
%What drives discourse about drugs? This study examines the dramatic decrease in negative coverage about marijuana in U.S. states from 1990 to 2016. I argue that, when focusing on liberties/freedoms, patients, and linking marijuana legalization with voting, the dominant discourse on contested issues can shift away. Results from logistic regression analyses in newspaper articles demonstrate that the shift away from negative discourse on marijuana was significantly influenced by shifting the discussion towards these topics. 
%\end{singlespace}
%\end{abstract}
%\newpage

\bibpunct[:]{(}{)}{;}{a}{}{,} % new punctuation for these, replace parentheses before and after citealt, and replace citealt with citep



%--------------------------------------------------------------------------------------------------------------------------------------
\section{Introduction}


Examination of frames, narratives, and discourse has been central to sociological understanding of cultural and political change \citep{ghaziani_and_baldassarri_2011,vasi_et_al_2015,bail_2012,mccammon_et_al_2007}. I contribute to this line of inquiry by investigating whether and how framing about marijuana changed over time. Coverage of marijuana presents an ideal case with which to examine discursive shifts, given that the ways in which marijuana was covered has moved away from negative discussions that centered on crime and danger \citep{newhart_and_dolphin_2018,bonnie_and_whitebread_1970,rosenthal_and_kubby_1996}, towards positive and neutral discussions regarding the benefits of marijuana and marijuana legalization \citep{mosher_and_akins_2019}. To what extent has there been a general shift away from negative coverage of marijuana? And to what extent have new frames come to dominate the discussion of marijuana? 


%I advance this body of work by examining the outcome of discursive shifts about marijuana. Given that, until recently, many politicians did not want to tackle the marijuana issue for fear of appearing less-than-tough on crime \citep{rosenthal_and_kubby_1996}, shaping media discourse about marijuana proved critical for advancing legislation on marijuana. Why did coverage of marijuana change, or become less negative, over time? Most importantly, to what extent did the narratives about marijuana contribute to this shift?

%I argue that discursive change is initiated by a variety of mechanisms, and that discourse changes by altering the conversation attached to an issue. Reframing of an issue in new and  culturally relevant ways \citep{gamson_and_modigliani_1989} that provide alternative understandings to longstanding debates \citep{bail_2012,snow_2004} can shift public perceptions and dominant discourse on an issue over time \citep{ghaziani_and_baldassarri_2011,baldassarri_and_bearman_2007}.

With recent advances in the field of computational social science, it has become easier than ever to gather, code, and analyze large corpuses of text for the purpose of tracking discursive trends \citep{bateman_et_al_2019,bail_2012,vasi_et_al_2015}. As such, I rely on webscraping techniques to gather newspaper articles about marijuana over time, breaking these articles as either covering marijuana with or without coverage of a marijuana advocacy organization. I, next, rely on automated text analysis to code the valence of each news article. Finally, using a combination of inductive and deductive approaches, I create a list of eight possible frames (and keywords associated with these frames) that may have appeared in marijuana coverage. I use automated coding to categorize each article has having the presence or absence of each frame. 


In this article, I shed light on the ways in which frames about marijuana changed in the United States between 1990 (when marijuana initiatives began to take off) and 2016. I demonstrate various characteristics about marijuana coverage and discourse over time. First, I show that, after 1996, when marijuana medicalization was on the ballot in California, coverage of marijuana increased dramatically. Secondly, I show that coverage of marijuana became decreasingly negative and increasingly neutral over time. Thirdly, I show that frames centering on revenue creation and politics came to dominate the discussion of marijuana, followed by framings related to rights and patients. 


% various framings or narratives about contested issues impact overall discourse. I do so by examining the impact narratives about marijuana on general media coverage of marijuana in the U.S. from 1990 to 2016. In particular, I examine whether the shift towards less negative discourse about marijuana resulted from various framings about marijuana. 


%American public opinion on marijuana legalization has undergone rapid change over the past two decades (Gallup 2013; Pew Research Center 2013). In 2006, a Colorado ballot initiative that would have legalized marijuana was handedly defeated; fewer than one quarter of Colorado counties supported the initiative. The issue was revisited in 2012, and with 55 percent of the vote, ushered in a new era of marijuana reform in the United States. The rapid growth in support for legalization set off a wave of state-level activity (as shown in Table 1), as organizations, voters, and politicians attempted to reform marijuana policy. Despite growing support for legalization, opponents generally perceive legalization as harmful to society; supporters, on the other hand, view legalization as an opportunity to remedy social problems (Caulkins et al. 2012; Rosenthal and Kubby 1996). Supporters of legalization argue that regulation and taxation of marijuana would help reduce crime, addiction, and drug abuse while also providing necessary revenue for educational and rehabilitative programs. 

%Perceptions of opportunity are the product of social construction processes, developed through social interaction, and are dependent upon the contexts in which people are embedded. In this article, I examine how residential segregation in local contexts that is based on the presence of households with (or without) children ? what I term, parental segregation ? influences voting behavior in support of policy change. This investigation is rooted in my understanding of how segregation has implications for how opportunity is perceived and acted on. Class-, race-, and education-based segregation operate by concentrating opportunities for some portion of the population while restricting access for others. Class segregation, for example, can restrict poor residents? access to quality education and employment (Massey and Denton 1993; Wilson 1987). Moreover, segregation concentrates individuals who have the knowledge and resources necessary to recognize and act on those opportunities. Importantly, segregation can facilitate reactive political action when it is vulnerable to penetration, or threatened by the increasing presence or political power of minority groups (McVeigh et al. 2014; Beisel 1997, 1990; Andrews and Seguin 2015). Limited contact with minority groups can lead majority group members to view racial minorities and the poor as responsible for their disadvantaged condition (Beisel 1990, 1997; Higham 1955). However, these beliefs may not resonate where actual threats do not exist. In fact, segregation concentrates residents of similar groups and decreases intergroup exposure, which can mobilize progressive political action (Galbraith and Hale 2008).

%The recent rise in ballot initiatives devoted to marijuana legalization provides a unique opportunity to examine how structural features of local contexts shape political behavior related to opportunity. The fight over marijuana?s legality is important in its own right because it involves battles over meanings of civil liberty, drugs, medicine, abuse, and morality. The growth in support for legalization is in part driven by backlash against prohibition and claims that legalization could contribute to safer communities by helping to shrink the size and influence of the illegal drug market (Marijuana Policy Project 2015; Rosenthal and Kubby 1996; Caulkins et al. 2012; Blocker 1989). While opposition exists, it remains highest amongst groups arguing that legalization would negatively impact communities, families, and children ? namely parents.

%In this article, I aim to shed additional light on marijuana legalization in the United States. I do so by considering whether the residential segregation of households with children from households without children (or, put another way, parents from nonparents) may facilitate support for policy change, such as marijuana legalization, that provides an opportunity to remedy social ills. Moreover, because perceptions of opportunity resulting from potential policy change is reflected in structural relations, changes in local communities may have a diminishing effect on support for marijuana reform. As such, I expect the effects segregation to be strong in unchanging communities.




%--------------------------------------------------------------------------------------------------------------------------------------



\section{Marijuana Discourse Across Time}


According to historians studying marijuana, initial depictions of marijuana were positive, centering on the medicinal and material benefits of cannabis \citep{bonnie_and_whitebread_1970,rosenthal_and_kubby_1996}. While there was relatively little coverage of marijuana during this time \citep{bonnie_and_whitebread_1970,mosher_and_akins_2019}, what little coverage did exist was positive. However, during the 1930s, the valence of this coverage shifted as a result of political and bureaucratic changes. 

In 1930, President Herbert Hoover established the Federal Bureau of Narcotics (FBN) and appointed Harry J. Anslinger as commissioner. According to scholars, Anslinger was in charge of repurposing prohibition funds, and, in noticing that Americans were enjoying Mexican and Native-American cannabis, Ansligner chose to direct his Bureau's resources toward cannabis \citep{newhart_and_dolphin_2018}. In fact, Anslinger used media to paint cannabis in a negative light as a way to ramp up public opposition to the substance. Anslinger used stories and advertisements in newspapers (including William Randolph Hearst's newspapers) to portray cannabis as dangerous to women, children, and society \citep{mosher_and_akins_2019}. Hearst, for his part, stood to lose economically if American cannabis use expanded -- he invested in wood pulp, which he used for his papers, and the expansion of hemp (which could also be used for cheaper newspaper manufacturing) put him in danger of losing his fortune. Through a campaign of ``yellow'' journalism, which enabled Anslinger to rebrand the drug with the more Native-sounding name marihuana (or marijuana) instead of cannabis, Anslinger and Hearst were able to associate the drug with a source or group of people responsible for the drug problem: immigrants, Mexicans, and indigenous ``others.'' Through newspapers, Anslinger and Hearst were able to ``sell'' marijuana as dangerous -- relying on a fear narrative that argued that only through cannabis prohibition could America's children, women, and society be protected \citep{mosher_and_akins_2019,newhart_and_dolphin_2018,rosenthal_and_kubby_1996}. 

Within a few years, these narratives took hold. In fact, over this time, marijuana became increasingly associated with criminality -- and in particular, minority criminality. During this time, marijuana was thought to be used mainly by minorities (freed Black slaves and Mexican immigrants) and had psychological properties that made them more prone to violence \citep{caulkins_et_al_2012,slaughter_1987}. These dominant narratives came to be used in arguments to Congress in favor of a full ban on marijuana -- resulting in the creation of the Marihuana Tax Act of 1937 \citep{newhart_and_dolphin_2018}. 



Between the 1930s and 1960s, the Act faced tough criticism in state and federal courts, as the judicial system worked to clarify the parameters of the law as well as what could and could not be enforced \citep{bonnie_and_whitebread_1970,mosher_and_akins_2019}. Around this time, advocacy organizations emerged to fight for access to marijuana \citep{newhart_and_dolphin_2018}. In 1970, the first marijuana movement organization, the National Organization for the Reform of Marijuana Laws, was created to fight against marijuana prohibition and to move public opinion on marijuana so as to enable full legalization of marijuana for all people. It wasn't until the mid-1990s and early 2000s (during NORML's fight for medicinal marijuana use on the California ballot) that other organizations such as the Marijuana Policy Project (1995), Students for Sensible Drug Policy (1998), and the Drug Policy Alliance (2000) joined the fight -- each with a specific purpose for legalization. For example, MPP would work on marijuana policy specifically while DPA would focus on both marijuana and similar narcotic policies, and SSDP would work to change the minds of youth, particularly on college campuses. 

According to prior research, marijuana discourse in the shifted in the 1960s. During this time, coverage of marijuana centered on the freedom to use the drug, but also included negative coverage about users -- framing users as addicts, hippies, and burnouts \citep{mosher_and_akins_2019}. Given that many scholars have given attention to coverage of marijuana during this period, and that of coverage during the beginning of President Nixon's ``War on Drugs,'' \citep{bonnie_and_whitebread_1970,mosher_and_akins_2019,alexander_2010,caulkins_et_al_2012}, I focus on coverage that took place immediately following this time, when change via statewide initiatives became a plausible alternative for policy change. 

In the mid-1990s, private individuals began to fight against marijuana prohibition in the United States by sponsoring marijuana medicalization initiatives in states with direct democratic processes. During this same time, newspapers 1) give increasing attention to marijuana, and 2) link discussions of marijuana with additional topics. Some scholars have indicated that during this time, narratives about marijuana began to become more positive, as coverage tended to focus on the political aspects of initiatives or voters, the benefits to patients, and the rights of cannabis users \citep{newhart_and_dolphin_2018,mosher_and_akins_2019,bonnie_and_whitebread_1970}. Over time, the trajectory of marijuana legalization is such that states (under federal prohibition of marijuana, and those with the initiative/referendum process) would first propose medical legislation, followed by recreational legalization legislation. During this time, marijuana coverage began to link with larger ``American'' values of liberties and freedom, in addition to shifting to the benefits of legalization for revenue, creating resources for rehabilitation, decreasing crime, and altering policing practices -- especially in communities of color \citep{mosher_and_akins_2019,newhart_and_dolphin_2018}.

Given this shift, I expect to find a shift in narratives about marijuana over time, coupled with a shift in the valence of attention to marijuana over time. Before turning to my data and methods, I outline some of the current thinking on the ways in which discourse changes. 

%\subsection{\textit{The Shift: Reframing and Positive Discourse about Marijuana}}



%\input{/Users/burrelvannjr/Dropbox/Professional/Research/Projects/dissertation/chapters/ch2.movements-and-discourse/paper/figures/figure2.tex}

%\input{/Users/burrelvannjr/Dropbox/Professional/Research/Projects/dissertation/chapters/ch2.movements-and-discourse/paper/figures/figure3.tex}





%----------------------------------------------------------------------------------------------------------------------------------------------------


\section{Discursive Change Processes}

The study of discourse has been central to work in the sociology of culture \citep{dimaggio_1997,swidler_1986,lamont_1992}. In recent years, the study of discourse has also come to the forefront of work in political sociology, with increased focus on the influence of advocacy organizations on public discourse \citep{bail_2012,earl_2004,mccammon_et_al_2007,ghaziani_and_baldassarri_2011,vasi_et_al_2015,gamson_and_modigliani_1989,andrews_and_caren_2010}. Importantly, organizations must first be covered by news media, which provides the opportunity to gain ``standing'' or serve as legitimate representatives for their issues and have their preferred messages conveyed \citep{amenta_et_al_2012,ferree_et_al_2002,elliott_et_al_2016}. Drawing on previous work on framing \citep{goffman_1974,benford_and_snow_2000}, organizations can impact public conversation about issues \citep{bail_et_al_2017} and initiate discursive change by offering their own diagnoses of and solutions to problems \citep{bail_2012,snow_et_al_2007,benford_and_snow_2000}. Frames that `fit' the broader discursive environment \citep{mccammon_et_al_2007} or those that articulate widespread beliefs and values usually survive over alternative frames \citep{mccammon_et_al_2001,snow_et_al_2007,gamson_and_modigliani_1989}, because their frames ``are more easily integrated into broader media narratives'' on an issue \citep[858]{bail_2012} and appear familiar, realistic, or legitimate. 



While organizations can shape the direction of discourse on a topic, the social movements literature sometimes gives less attention to the role of mass media and news organizations in the process of cultural or discursive change. Because, especially for social movements, coverage of issues is critical for affecting discourse, news media are equally important for cultural change. Importantly, mass media are a master forum \citep{ferree_et_al_2002} within which actors compete for coverage of their issues \citep{amenta_et_al_2012}.  Mass media are central for making sense of relevant events \citep{gamson_and_modigliani_1989}, and serve to identify and redefine, which can shape public perceptions of those issues. Yet, media organizations themselves operate by a set of procedures, known as news values \citep{galtung_and_ruge_1965}, that can also have effects on discourse. 


News organizations must make decisions about what counts as ``news'' \citep{galtung_and_ruge_1965}. Often times, what counts is based on timeliness/currency, the impact of the events being covered, and the proximity of those events to potential readers -- with local news angles being important \citep{amenta_et_al_2012,galtung_and_ruge_1965}. In particular, politics receives the most coverage because political decisions have high impact and include prominent people. In addition, reporters often have increased access to political actors. In sum, much of what counts as news centers on institutional political activity, such as stories about politicians, bills being discussed, or laws being passed \citep{amenta_et_al_2012}. Therefore, it is reasonable to assume that, even when discourse on issues changes, the attention to those issues may include discussions of institutional political actors or action. Yet, this is not to suggest that political actors are driving the discussion, but instead that the larger discursive about an issue becomes increasingly ``newsworthy'' when included alongside coverage of institutional political action or actors. This may be the case for coverage of marijuana. Yet as discussed in the previous section, for many contentious issues like marijuana, political actors may not want to be the drivers of discursive change. In particular, when dominant discourse on a topic is negative, political actors may be unwilling to discuss the issue in new, positive ways, for fear of reprisals from their constituents in the form of lost votes. Therefore, driving discursive change may provide little political advantage. 



Given the tendency of journalists and the norms of news-gathering organizations to seek out official sources \citep{schudson_2002,gitlin_1980,gans_1979}, especially during times of high attention to the issue in the news cycle \citep{baumgartner_and_jones_1993} the general lack of discussion by political actors creates opportunities for other actors (including journalists themselves) to, not only, provide alternative narratives on an issue, but shift the character of discourse by linking the issue with new topics or frames. 

%that are different from dominant discourse on that topic can be a political disadvantage


I argue that, by linking discourse about contested issues to additional, more institutionalized topics or narratives, journalists can facilitate a stark transformation in public understandings on an issue, which can call into question the legitimacy of previously dominant representations of the issue. Initial marijuana discourse centered on criminality and the negative educational, occupational, and mental effects of marijuana use. In recent years, however, this discourse has given way to increasingly positive (or fewer negative) discussions of marijuana's medical, community, and economic benefits. This coverage has shifted the arenas within which marijuana was discussed by linking marijuana with narrative topics related to medicine, rights, freedom, economics, crime, and policing. Yet it remains unclear whether and how these new narrative topics contributed to the discursive shift away from a dominant negative discourse about marijuana. 

%In addition, the movement has devoted much of it's resources not on protest as a means of changing public definitions of marijuana, but on local and statewide initiatives, gaining media coverage of their issue, and on holding local community-based meetings. 


%--------------------------------------------------------------------------------------------------------------------------------------

%--------------------------------------------------------------------------------------------------------------------------------------
\section{Data \& Method}

Given my interest in \textit{whether} and \textit{how} marijuana discourse shifted over time, I analyze text from print news media across the United States. To do so, I rely on the ProQuest newspaper database. I constrain the analysis to 1990 and on because coverage on marijuana was relatively low prior to 1990, and because this time frame immediately followed Reagan's intensified ``War on Drugs'' and ``Just Say No'' campaign. 

To track discursive change, I rely on articles about marijuana in the Proquest database from 1990 to 2016. Because marijuana advocacy organizations may have had an impact on coverage, I separately searched for articles about marijuana in the absence of advocacy organizations, and articles about marijuana that included advocacy organizations. To accomplish this,  I wrote a Python script to identify and download all local articles from Proquest that mention ``marijuana'' between 1990 and 2016.\footnote{This does not include variants of the word marijuana, or the word cannabis} Because national newspapers may be more likely to cover national issues over local issues, I exclude national newspapers, including the \textit{New York Times}, the \textit{Los Angeles Times}, the \textit{Washington Post}, and the \textit{Wall Street Journal}. In addition, I exclude articles that mention at least one of the four main marijuana advocacy organizations. Therefore, I also exclude articles that mention National Organization for the Reform of Marijuana Laws (NORML), Marijuana Policy Project (MPP), Drug Policy Alliance (DPA), and Students for Sensible Drug Policy (SSDP), and their variants. In total, there were 14,163 articles mentioning marijuana. After removing duplicate articles, articles outside of the U.S. or located in the U.S. capitol\footnote{ProQuest sometimes mistakenly identifies non-U.S. articles when only-U.S. articles are specified.}, short articles (e.g. articles with fewer than 100 words), and articles that are not fully searchable,\footnote{Articles with fewer than about 900 words.}, I am left with 10,096 locally-based articles, across 185 newspapers, that mention marijuana in some fashion. In addition, I removed articles that come from ``alternative'' or sensationalized newspapers. To figure out whether or not the newspaper was an ``alternative newspaper,'' I searched the websites for each newspaper, removing any newspaper that claimed that it was an alternative newspaper. As can be seen in Table \ref{tab:local_coverage}, I am left with 5,893 articles, across 100 newspapers, about marijuana which do not include mention of marijuana advocacy organizations. %498 are positive, 74 have plagiarism


\input{/Users/burrelvannjr/Dropbox/Professional/Research/Projects/dissertation/chapters/ch2.movements-and-discourse/paper/tables/table_local.tex}


Because marijuana advocacy organizations' discussion of marijuana may be important for discursive change on marijuana, I also include coverage of ``marijuana'' alongside coverage of marijuana advocacy organizations. As such, I wrote a separate Python script to identify and download all articles from Proquest that mention ``marijuana'' and any one of the four largest marijuana advocacy organizations (and the variants of their names) between 1990 and 2016. Therefore, the script was able to capture all coverage of ``marijuana'' coupled with coverage of marijuana advocacy organizations, including the National Organization for the Reform of Marijuana Laws (NORML), Marijuana Policy Project (MPP), Drug Policy Alliance (DPA), and Students for Sensible Drug Policy (SSDP).\footnote{Importantly, I separate these sets of coverage for future empirical work on the impact of organizations on the discursive shift.} In total, there were 1,616 articles mentioning a marijuana movement organization. After cleaning the data set of articles by removing duplicate articles, I am left with 1,150 articles mentioning marijuana advocacy organizations. In addition, after removing and articles coming from alternative news sources, I am left with 787 marijuana organization-related articles from 62 newspapers (see Table \ref{tab:mm_coverage}). For these articles, I include a dummy code to represent that they include mentions of organizations. 


\input{/Users/burrelvannjr/Dropbox/Professional/Research/Projects/dissertation/chapters/ch2.movements-and-discourse/paper/tables/table_mm.tex}


As can be seen in Table \ref{tab:all_coverage}, there are a total of 6,680 articles used across these analyses. 


\input{/Users/burrelvannjr/Dropbox/Professional/Research/Projects/dissertation/chapters/ch2.movements-and-discourse/paper/tables/table_all.tex}

 


%The unit of analysis is at the article level.%I focus on local rather than national level discourse in print media given recent criticism against relying on national media sources \citep{earl_et_al_2004}\footnote{I therefore exclude the \textit{New York Times}, \textit{Los Angeles Times}, \textit{Wall Street Journal}, and \textit{Washington Post}.}, and because marijuana movement organizations focus mainly on local level activism which may garner media attention (e.g. chapter meetings and campaigns for ballot initiatives) instead of distributing press releases.\footnote{Unlike women's jury rights studied by \citet{mccammon_et_al_2007} or the civil society organizations targeting Islam and Muslims studied by \citet{bail_2012}, many of the marijuana movement organizations show little evidence of information distribution, therefore, locating items such as press releases, speeches, pamphlets or flyers, transcripts of television or public speeches proves difficult. In fact, much of the organizations activity centered on local chapter meetings (which rarely include meeting minutes or are transcribed) and public action during ballot initiative campaigns or events such as HempFest. Much of the movement's contemporary activism is geared towards maintaining a social media presence for local chapters. This activity, again, centers on promoting local initiatives. In the future, I hope to diversify the explanatory sources to include any movement-generated documents that have been found to influence media (see \citealt{sobieraj_2011}) including mission statements, meeting minutes, and newsletters/flyers promoted.}

%However, there is considerable variability in the extent to which local articles report on marijuana. For example, some newspapers may report on marijuana only once in a given year, not at all the subsequent years, and heavily many years later. Such a pattern of reporting across local news sources creates analytical problems. Ultimately, because discourse is often constrained by the environments in which they take place \citep{mccammon_et_al_2007}, I examine the incidence of positive marijuana discourse in each U.S. state between 1990 and 2016, using state-years as the unit of analysis. State-years provide comparative leverage because I can compare 1,350 cases (27 years across 50 states). A state level analysis, using state-years allows me to account for state-level heterogeneity, and differences across years.

%Because I am interested in how discourse changes over time, it is necessary to track changes in marijuana discourse over time. %As such, the main dependent variable is binary -- whether or not the article about marijuana is predominantly negative. I therefore use logistic regression to estimate the models. To reduce the risk of biased estimates, I use fixed effects models. The fixed effects design explicitly models the change that occurs within states over time, therefore, the results are identical to those that would be obtained if I manually inserted a dichotomous variable for every state. One important advantage of the fixed-effects model is that it controls for all constant, but unobserved and unmeasured, differences across our cases (Allison 1994). Because I estimate change within states over time, omitted variables are problematic only if they are time-variant.



%\subsection{\it{Dependent Variable}}

Because I am interested in the shift away from negative discourse about marijuana, I categorize each article based on it's polarity or valence.\footnote{To prepare all documents for textual analysis, following the procedure used by \citet{bail_2012}, I use software in R to transform each article into fully-searchable sets of words, and clean the textual data by eliminating excessive words (e.g. stop-words such as numbers, conjunctions, and determiners), and transforming each word into it's stem variant.} I code each article with the assistance of a na\"{i}ve Bayes classifier in \textsf{R}'s \texttt{sentiment} package \citep{jurka_2012}. The na\"{i}ve Bayes algorithm uses a stock of trained text that has been associated with three types of polarity (positive, neutral, or negative), and attempts to classify each document as one of the three polarities.\footnote{The stock of trained text comes from Janyce Wiebe's subjectivity lexicon \citep{wilson_et_al_2005}, which can be found at: \url{https://mpqa.cs.pitt.edu/lexicons/subj_lexicon/}.} In doing so, the algorithm simply compares the word stems in each article to word stems in each of the three dictionaries and classifies each word in the article as negative, neutral, or negative. Next, the algorithm calculates the log likelihood that a given article contains positive or negative sentiment, followed by a fit score, which is a ratio of the log likelihoods between positive and negative sentiment, where a score of `1' indicates neutral polarity in the article, a score below `1' indicates negative polarity, and a score above `1' indicates positive polarity. I simplified version of this process is below. In an article in the \textit{Tennessee Tribune} in 2013, Congressman Steve Cohen (D-TN) praised the Administration's shift on marijuana: 

%1439216197
\begin{quotation}
\begin{singlespace}
\noindent President Obama's Administration is making incremental progress to address the basic unfairness of our federal drug policy and law enforcement policy, and I appreciate that it has started to work on these important issues.\footnote{``Cohen Statement on Medicinal Marijuana.'' \textit{Tennessee Tribune}. September 5, 2013.}
\end{singlespace}
\end{quotation}

In this section of the article, the words ``progress,'' ``important,'' and ``appreciate'' would be categorized as positive, and the word ``unfairness'' would be categorized as negative, given the linguistic dictionary -- a stock of trained keywords signifying negative, neutral, and positive coverage -- provided by the \texttt{sentiment} package in \textsf{R} \citep{jurka_2012}.\footnote{The dictionaries can be viewed through the package information provided by \citet{jurka_2012}.} In simplified terms, this section of the article would be categorized as positive, given that it would be given more positive codes than negative codes. I therefore use this package (with the algorithm based on the subjectivity lexicon dictionary) to code the valence or polarity of each article, for all 6,680 articles. 






%. A key principal in computational social science is ``supervised'' learning, wherein the researcher does a small set of qualitative coding, the algorithm learns from those codes to generate most similar codes for the rest of the data. The na\"{i}ve Bayes classifier is best suited for training on a previously-coded, small subset of data, learning from those codes, and classifying those data based on the researcher-generated codes but also provides the option of ``NULL'' where it cannot appropriately classify. Alternatively, the researcher could conduct qualitative coding on an already coded subset of data to see if his/her codes align with the algorithm-generated codes. For polarity and emotional content, I will engage in the latter. However, to understand the exact \textit{frames} the movement put forth, I will engage in the former. Both of these methods of textual analysis, using semi-automated or supervised learning techniques will provide an interrater reliability score by which I can demonstrate consistency in coding. Moreover, both will be the bulk of my remaining work.} 

%\subsection{\it{Independent Variable}}


Given my interest in the frames included in coverage of marijuana, I give each article a code for the presence or absence of various frames. To select frames, I rely on a process of induction -- my selection of frames is based on the various frames that have been associated with marijuana over time. In particular, I focus on seven frames about marijuana, based on prior research. 

In their work describing reasons why marijuana should be legal, \citet{rosenthal_and_kubby_1996}'s arguments center on the rights, as given by the Constitution, as well as Liberterian ideals like individual freedom and personal liberty to use marijuana without fear of retribution or governmental intrusion into private affairs. As such, I include two frames that capture ``rights,'' and ``liberties.''  Additionally, in recent years, there has been a shift in public framing of marijuana, alongside the shift toward statewide ballot initiatives as a venue for legalization \citep{mosher_and_akins_2019}. These frames have centered on the benefits of legalization for creating streams of revenue through taxation on regulated marijuana sales \citep{caulkins_et_al_2012,miron_2010}. I therefore include a frame related to ``revenue creation.'' \citet{newhart_and_dolphin_2018} have recently demonstrated that part of the broader fight for marijuana legalization necessarily depends on preceding fights for and successful medicalization. As such, discussions of legalization are often intertwined with discussions about the medicinal and public health benefits of marijuana, particularly for patients who might require alternative treatments for their medical issues. As such, I include a frame that centers on ``patients.'' Scholars have also focused on the restorative justice effects of marijuana legalization. This line of inquiry usually focuses on how legalization could redirect police resources away from hyper-policing in communities of color \citep{alexander_2010,davis_2003}, toward rehabilitation \citep{alexander_2010,mosher_and_akins_2019,davis_2003} and how legalization would restrict the flow of marijuana from underground drug markets, thereby reducing crime \citep{mosher_and_akins_2019,caulkins_et_al_2012}. 
As such, include a frame that encompasses ``policing.''\footnote{Upon closer investigation of the data, I noticed that frames of ``crime reduction'' and ``rehabilitation'' often covered people arrested, their histories of marijuana use, their placement in rehabilitation facilities, and variable mentions of the term ``reduce.'' As such, they were removed from the analysis.} %,'' and ``crime reduction.'' 
Finally, given the nature of coverage of marijuana as a political issue, and given literature that identifies institutional politics as newsworthy \citep{amenta_et_al_2012,galtung_and_ruge_1965}, I also include a ``politics'' frame.

To identify the frames, I rely on keywords to select whether frames are absent or present in coverage of marijuana. In the Table below, I outline the search terms used for identifying these frames. In the Table, ``+'' represents the logical operator ``OR'' and ``*'' represents the logical operator ``AND.''

\input{/Users/burrelvannjr/Dropbox/Professional/Research/Projects/dissertation/chapters/ch2.movements-and-discourse/paper/tables/table1new.tex}

In what follows below, I show examples of each frame by using article excerpts. For example, the following excerpt comes from an article in the \textit{Colorado Springs Independent} in 2013 that was coded as containing a ``revenue'' frame:

\begin{quotation}
\begin{singlespace}
\noindent New city sales tax revenue from the venue would exceed \$275,000. (By comparison, medical marijuana last year brought in \$1.1 million in sales tax.) The stadium would cost about \$60 million to build.\footnote{Zubeck, Pam. ``Talking a Good Game.'' \textit{Colorado Springs Independent}. July 3, 2013.}
\end{singlespace}
\end{quotation}

As seen, although this article covers revenue from a stadium, it also compares that revenue to revenue generated by medical marijuana. In total, there were 2,316 articles that included the ``revenue'' frame. Next, from an article in \textit{The Louisiana Weekly} in 2013 that was coded as containing a ``liberties'' frame:

\begin{quotation}
\begin{singlespace}
\noindent On the eve of a new year, a libertarian strain pulses through America -- a get-government-out-of-my-personal-life sensibility that cuts across ideologies and is driven by a younger generation's cultural attitudes.\newline

\noindent We've seen it in gay-marriage legalization and marijuana decriminalization.\footnote{Sidoti, Liz. ``Gun Debate Revives Enduring American Fight.'' \textit{The Louisiana Weekly}. January 14, 2013.}
\end{singlespace}
\end{quotation}

Shown in the above excerpt, the author is drawing connections between the growth of Libertarian ideals in America and arguments in favor of not just marijuana, but also other issues like gay marriage (and beyond this, gun rights). In total, there were only 84 articles that included the ``liberties'' frame. Next, I include an excerpt from an article in \textit{The Boston Banner} in 2013 that was coded as containing a ``rights'' frame:

\begin{quotation}
\begin{singlespace}
\noindent The Barack Obama administration has announced a set of proposals aimed at stemming the growth of the U.S. prison population and racial disparities in the criminal justice system - chief among them, the elimination of mandatory minimum sentencing for low-level drug offenses.\newline

\noindent African Americans are nearly four times as likely to be arrested for marijuana possession than whites, despite similar usage rates between the two groups.\newline

\noindent Mandatory minimum sentences are not only unfair in stature and consequence, they represent a serious threat to the civil rights gains and progress of the 1960s and '70s.\footnote{Kandil, Caitlin Yoshiko. ``Obama's Criminal Justice Reform Lauded for Historic Proposals.'' \textit{The Boston Banner}. August 15, 2013.}
\end{singlespace}
\end{quotation}

In the above excerpt, we see how marijuana legalization and the removal of mandatory minimum sentences associated with marijuana infractions as a civil rights issue. In total, there were 1,255 articles that included the ``rights'' frame. I next give an example of the ``patient'' frame, from an article in the \textit{Atlanta Inquirer} in 2005:


\begin{quotation}
\begin{singlespace}
\noindent When the U.S. Supreme Court ruled against medical-marijuana users, many critics of the decision thought the six-justice majority failed to show compassion for severely ill people. \newline

\noindent Under California's Compassionate Use Act, doctors may prescribe marijuana to patients with severe medical problems. Those patients are then permitted to grow marijuana for their own use. The state closely regulates the prescription, cultivation, and use of the product to prevent others from obtaining it. (At least nine other states have similar laws.)\newline

\noindent Many well-intentioned people say yes: of course, severely ill people should be able to grow and use marijuana by prescription without fear that federal agents will barge into their homes, destroy their plants, and charge them with unlawful possession.\newline

\noindent Sick people need freedom, not permission, however compassionate the motive.\footnote{``Muddle At The Supreme Court Over Medical Marijuana.'' \textit{Atlanta Inquirer}. July 9, 2005.}
\end{singlespace}
\end{quotation}

Here, we see that marijuana is being framed in terms of the patients, or groups of people with illnesses that could be treated through marijuana use. In total, there were 806 articles that included the ``patient'' frame. Next, I show the ``policing'' frame, which focuses not just on policing, but policing's effects on communities of color. This example comes from an article in the \textit{Sun Reporter} in 2015:


\begin{quotation}
\begin{singlespace}
\noindent Blacks and Latinos are incarcerated at disproportionately higher rates in part because police target them for minor crimes. \newline

\noindent Racial disparities that exist at every step in the criminal justice system, the report noted. That helpsexplain why Blacks and Latinos account for about 30 percent of the United States population, but 56 percent of the incarcerated population. \newline


\noindent These trends are driven by race-neutral laws that still have a significant have racial impact, criminal justice professionals influenced by racial bias, an underfunded criminal justice system, and policies that impose strict "collateral consequences" that make it harder for ex-offenders to return their home after prison.\footnote{``Police Killings Underscore Need For Reform.'' \textit{Sun Reporter}. February 12, 2015.}
\end{singlespace}
\end{quotation}

Here, the author draws attention to criminal justice practice (specifically policing) in communities of color, particularly regarding marijuana, as a reason for the spike in the prison population. In total, there were 126 articles that included the ``policing'' frame. Finally, I have an example of the ``politics'' frame, coming from the \textit{Milwaukee Courier} in 2012:


\begin{quotation}
\begin{singlespace}
\noindent Election night ushered in some other political surprises as well. Last Tuesday's election was a watershed moment for the gay marriage movement. Voters in three states voted to legalize it - something no state had done before - and a fourth state voted against a proposed ban.\newline

\noindent Tuesday's progressive electorate also weighed in on the issue of marijuana usage. Colorado and Washington became the first U.S. states to legalize the possession and sale of marijuana for recreational use -- in defiance of federal law.\footnote{``Capitol Report - The Dawning of Liberal America?'' \textit{Milwaukee Courier}. November 17, 2012.}
\end{singlespace}
\end{quotation}

As can be seen, in this article, there is traditional electoral political coverage, of which marijuana is a topic covered. Below, I describe variations in coverage of marijuana in the presence and absence of organizations. 










%Given my argument about the influence of the movement's discourse on marijuana discourse generally, I use novel plagiarism detection software, developed by \citet{welbers_and_van_atteveldt_2016}, to link articles with movement-initiated marijuana discourse to subsequent articles about marijuana. The software\footnote{\textit{RNewsflow}, in the R statistical environment, is designed to analyze homogeneity in text between two news articles (or corpora of text) while also tracing the diffusion of similar subsets of text across time.} compares strings of text in articles in the explanatory set (movement discourse about marijuana) to strings of text in subsequent articles in the outcome set (general discourse about marijuana). For example, an article from May 1, 1970 in which NORML discusses marijuana or is mentioned alongside a discussion of marijuana will be compared to all articles in the outcome set (non-movement mentions of marijuana) that occur on or after May 2, 1970. Each movement article then receives a similarity score to represent the proportion of text that is reported verbatim in every ``non-movement'' article (the outcome set). The creators of this software recommend a minimum similarity score cutoff of .40, which has been a reliable threshold for identifying sets of text that address similar events. It is therefore possible for an article to have multiple scores, representing varying similarities with numerous documents. Each article in the outcome/dependent variable set is then dummy coded to represent whether or not what it's text contains verbatim text (with a score of .40 or above) from an article with coverage of a marijuana movement organization \textit{alongside} a discussion of marijuana. This means that the marijuana could be projecting it's own frames about marijuana or be given standing on the marijuana issue. This approach is similar to those used by other scholars studying media coverage of social movements \citep{amenta_et_al_2009,andrews_and_caren_2010}.

%It is important to note that, unlike many social movement organizations, the nature of the marijuana issue in political discourse has pushed marijuana activists away from the disruptive types of of action that woul 

%\subsection{\it{Control Variables}}

%I include variables to that may also account for less negative coverage on marijuana. Research on media coverage shows that advocacy organizations can also shape the direction of discourse \citep{seguin_2016,bail_2012}. I therefore include a measure for whether or not the article covers a marijuana advocacy organization. Research suggests that discursive opportunities are related to structural conditions \citep{mccammon_et_al_2007}. I therefore include for various state-level factors that may account for decreases in negative coverage of marijuana. First, from the Secretary of State websites for each state in each year, I include 1) a measure of whether or not marijuana legalization was on the ballot in that state and in that year, and 2) whether or not marijuana had been or was already medicalized in that year. 

%In addition, I include various other state-level controls, many of which come directly from the Census and the American Community Survey. I match each decennial Census with the year it was taken and the following years not covered by the subsequent Census, matching the 1990 Census with years 1990 through 1999 and the 2000 Census with years 2000 through 2008. For years 2009 through 2016, however, I use the 2005-2009 American Community Survey (ACS).\footnote{I exclude data from Alaska due to availability.} Because each of these data are measured only during Census years, I use linearly interpolate values for interim years. Firstly, the number of newspaper articles about marijuana may be a function of population. As such, I include a measure for the natural log of the total population in a state. Second, recent research has demonstrated that there is higher support for marijuana in locales with higher percentages of college graduates and liberal voters \citep{caulkins_et_al_2012,rosenthal_and_kubby_1996}. As such, I include a measure for the percent of the total population aged 25 or older with a four-year college degree. It is reasonable to assume that decreasingly negative coverage of marijuana could have resulted from severe economic conditions in locales (e.g. marijuana may be perceived as beneficial for strengthening local economies)  \citep{caulkins_et_al_2012,caulkins_2010,miron_2010}. For this reason, I include a measure for the percent of the population aged 16 or older that is employed. 





%Finally, given that Democratic political officials and voters tend to exhibit higher support for legalization, I use data from Congressional Quarterly's {\it{America Votes}} to measure the percentage of voters who voted for the Democratic candidate in the presidential elections that coincide with, or immediately precede, yearly data. This means that for each presidential election year during the period from 1990 to 2016, values are calculated directly from voter percentages. All years between presidential election years are linearly interpolated. For example, articles written in 1990, I use the percent of the vote for Michael Dukakis in 1988, linearly interpolated from 1988 to 1992, using the values for 1990. For articles written in 1992 and 1996, I use the percent vote for Bill Clinton. For articles written in 2000, I use the percent of the vote for Al Gore. Articles written in 2004 are associated with the percent of the vote for John Kerry, while articles written in 2008 and 2012 are associated with the percent of the vote for Barack Obama. Finally, for articles written in 2016, I use the percent of the vote for Hillary Clinton. To be clear, values for years between presidential elections are calculated using linear interpolation. 



%--------------------------------------------------------------------------------------------------------------------------------------
\section{Results}

\subsection{Frames in Overall Coverage}
%\textit{Frames in Overall Coverage}
As seen in the figures below, there was a substantial change in marijuana coverage over time. In particular, in Figure \ref{fig:coverage}, it is important to recognize that there was a consistent increase in overall coverage of marijuana over time. The amount of attention to marijuana reached its peak in 2014, with just over 600 articles. We can see that there are two dramatic peaks in coverage -- in 2010 and 2014, respectively. These peaks coincide with ballot initiatives California, Alaska, and Oregon. 

\input{/Users/burrelvannjr/Dropbox/Professional/Research/Projects/dissertation/chapters/ch2.movements-and-discourse/paper/figures/figure1_a.tex}



Coding the valence/polarity for each articles, reveals an interesting pattern: there was both a substantial decrease in negative attention to marijuana and an increase in neutral coverage of marijuana. As seen in the Figure \ref{fig:both_pct} below, the percent of negative coverage about marijuana only stayed consistently below fifty percent only after 2010. %in 1996 (when medical marijuana was on the ballot in California) and again in 2008. Only after 2010 did negative attention to marijuana consistently drop below fifty percent. 
During this same time, we see that coverage of marijuana became increasingly neutral over time. In fact, a majority of attention to marijuana was neutral in later years, between 2013 and 2016. Given these data, there is a sense that marijuana discourse changed over time, becoming less negative -- a shift away from the dominant negative discourse on marijuana identified in prior research \citep{mosher_and_akins_2019,bonnie_and_whitebread_1970}. 

\input{/Users/burrelvannjr/Dropbox/Professional/Research/Projects/dissertation/chapters/ch2.movements-and-discourse/paper/figures/figure1new.tex}


%%pct negative
%\input{/Users/burrelvannjr/Dropbox/Professional/Research/Projects/dissertation/chapters/ch2.movements-and-discourse/paper/figures/figure1.tex}

%%pct neutral
%\input{/Users/burrelvannjr/Dropbox/Professional/Research/Projects/dissertation/chapters/ch2.movements-and-discourse/paper/figures/figure1_1.tex}

I now turn to the frames presented in articles about marijuana. To be sure, all frames exhibit some degree of variation over time. As shown in the figures below, I have broken the frames into high, moderate, and low frequency frames for ease of interpretation. As seen in Figure \ref{fig:narratives_high}, there is a general increase in frames about politics and revenue creation.  Importantly, throughout nearly the entire period under investigation, discussions that linked marijuana to revenue, taxes, or many were the most prevalent. While the dominance of this frame is clear, narratives linking marijuana to politics came second, and oftentimes showed a similar, albeit lower, trend. In addition to the general growth in these two frames over time, there was also a large drop for both in 2012. Part of this drop may be due to higher levels of election coverage in 2012 that did not include mentions of marijuana. 


%there are periodic drops in frames that coincide with national elections (Presidential re-elections and midterms), which is contrary to literature suggesting that elections increasing attention to political issues. Here we see large drops in most frames in 2004 and 2012, indicating that other frames may have come to dominate marijuana coverage when national issues reach the public agenda. To be sure, frames varied in quantity over time. Importantly, between 1990 and 2016, frames about revenue creation dominated nearly all yearly coverage. This indicates that often times, when discussing marijuana, these narratives were linked with discussions of revenue, money, or taxation. While the dominance of this frame is clear, narratives linking marijuana to political behavior came second. All frames exhibit some degree of variation over time, with 



\input{/Users/burrelvannjr/Dropbox/Professional/Research/Projects/dissertation/chapters/ch2.movements-and-discourse/paper/figures/figure2_1.tex}


I next turn to moderately covered frames in overall coverage in Figure \ref{fig:narratives_mid}. These include coverage of patients and rights. These two trends are relatively similar across time. We see brief spikes in coverage of marijuana associated with patients (e.g. medicinal uses) and rights in 2010 and 2014. These two spikes may be related to the added coverage of marijuana at when marijuana-related initiatives were put on the statewide ballots in California (2010), Alaska (2014), and Oregon (2014), which provided increased opportunities for marijuana to be linked with additional frames.

\input{/Users/burrelvannjr/Dropbox/Professional/Research/Projects/dissertation/chapters/ch2.movements-and-discourse/paper/figures/figure2_2.tex}

Finally, Figure \ref{fig:narratives_low} depicts low-frequency frames in news coverage of marijuana. From this figure, we can see that frames centering on liberties and policing remained relatively low (near zero) during the period of investigation. Only in later years, around 2013, did we see a slight increase in the policing frame, which may be related to increased coverage of police shootings of unarmed Black people and coverage of the Black Lives Matter organization. 

\input{/Users/burrelvannjr/Dropbox/Professional/Research/Projects/dissertation/chapters/ch2.movements-and-discourse/paper/figures/figure2_3.tex}



\subsection{Frames by Local versus Organization-Related Coverage}

I next separate the above frames to compare differences in frames used between local coverage about marijuana only, and coverage about marijuana alongside mentions of marijuana advocacy organizations. Again, I present three figures that align with the high-, moderate-, and low-frequency frames presented above. These figures show the percentage of coverage in a given year that includes the various frames, broken out by whether or not the coverage also mentioned at least one of the four marijuana advocacy organizations. 

Figure \ref{fig:both_narratives_high} depicts the percentage of all coverage in that year that includes politics or revenue creation frames. As we can see, general marijuana coverage tends to center on revenue as the topic of conversation, even if only around 30 percent of the time. In general (local) coverage of marijuana, politics was mentioned in connection with marijuana even less, often between 15 and 20 percent of the time. Turning to organizational coverage, we see that coverage of politics and revenue creation frames show considerable variability, with frames of revenue creation sometimes dominating coverage and frames of politics sometime dominating. 


%For politics and revenue frames, there was a substantial increase in their coverage over time. This trend was less dramatic for articles that mentioned marijuana advocacy organizations, given that these occurred less frequently overall than coverage of marijuana only. 

\input{/Users/burrelvannjr/Dropbox/Professional/Research/Projects/dissertation/chapters/ch2.movements-and-discourse/paper/figures/figure2_4.tex}


Next, \ref{fig:both_narratives_mid} shows that coverage of organizations was more likely than general marijuana coverage to include frames that centered on patients and those that centered on rights. In addition, when compared to Figure \ref{fig:both_narratives_high} above, and Figure \ref{fig:both_narratives_low} below, we see that organizational coverage was most likely to include frames that focused on rights. When compared to all coverage, we can see that organizational coverage may be driving the amount of articles that include rights frames. 

\input{/Users/burrelvannjr/Dropbox/Professional/Research/Projects/dissertation/chapters/ch2.movements-and-discourse/paper/figures/figure2_5.tex}


Finally, Figure \ref{fig:both_narratives_low} shows that both general marijuana coverage as well as organization-related coverage were least likely to include frames about policing or liberties. Here we see that for both, there was near-zero percent of coverage that included these frames. There were brief spikes in liberties frames in organization-related coverage over time. In addition, both coverage sets show a minor spike in policing frames later in time. 

\input{/Users/burrelvannjr/Dropbox/Professional/Research/Projects/dissertation/chapters/ch2.movements-and-discourse/paper/figures/figure2_6.tex}


\subsection{Frames and Polarity}

Do different frames impact the polarity of coverage? To investigate this further, in Figure \ref{fig:all_narratives_pct_neg} below, we can see how each frame, when used, has become less associated with negative polarity over time. The graphs are broken out by whether or not the frame appeared in general marijuana coverage or organization-related coverage. In addition, I order these graphs in terms of their total frequency over time (with the revenue appearing most frequently, and the policing frame appearing least frequently). 

Beyond this, in Figure \ref{fig:all_narratives_pct_neu} below, we can see how each frame became more neutral over time. Again, graphs are broken out by local versus organization-related coverage, and are ordered by their total frequency over time (with the revenue appearing most frequently, and the policing frame appearing least frequently). As we see, for many frames, neutral coverage either hung around 50 percent of each frame's coverage, or increased over time. 

\input{/Users/burrelvannjr/Dropbox/Professional/Research/Projects/dissertation/chapters/ch2.movements-and-discourse/paper/figures/figure2_7.tex}

\input{/Users/burrelvannjr/Dropbox/Professional/Research/Projects/dissertation/chapters/ch2.movements-and-discourse/paper/figures/figure2_8.tex}



To follow this up, I conducted a simple logistic regression of negative coverage on the various frames. In Table \ref{tab:logit}, we see that various frames have distinct impacts on negative polarity of an article. As seen below, more  frames have significant impacts on the polarity of coverage in general marijuana coverage than in organization-related coverage. In particularly, for general marijuana coverage, frames centering on liberties, patients, and politics significantly decrease the likelihood that an article is classified as negative. For organization-related coverage, however, none of the frames decrease the likelihood that an article is negative. Conversely, frames of revenue creation and policing increase the likelihood an article is classified as negative -- only policing frames impact negative polarity for organization-related articles. In sum, we see not only that frames were used variably over time, and that both general and organization-related coverage included these frames variably, but also that some frames (more than others) contributed to decreases in negative coverage over time. 





%Table 2 presents logistic regression results for the likelihood of negative articles about marijuana in each state from 1990 to 2016, including state level fixed effects. Column 1 includes only the variables of interest, various narratives about marijuana. As shown, the coefficients for liberties, patients, and politics narratives about marijuana all have significant negative effects on the likelihood of negative coverage about marijuana in the U.S. This provides support for my claim that linking marijuana with American values narratives, in addition to beneficiary groups and traditional political processes is relevant for the overall decrease in negative discourse about marijuana. On the other hand, articles that include narratives about policing or revenue were more likely to be negative. Coefficients in logistic regression can be  interpreted by exponentiation, or $(e^{b} - 1)*100$, which gives the expected change in the dependent variable that coincides with a one-unit increase in the independent variable. In sum, articles with narratives about liberties, patients, or politics were had decreasing likelihoods of being negative -- they were associated with a 31, 22, and 29 percent decrease in the likelihood of negative coverage about marijuana.

%\input{/Users/burrelvannjr/Dropbox/Professional/Research/Projects/dissertation/chapters/ch2.movements-and-discourse/paper/tables/table2new.tex}


%\input{/Users/burrelvannjr/Dropbox/Professional/Research/Projects/dissertation/chapters/ch2.movements-and-discourse/paper/tables/table5_3-21c.tex}


%\input{/Users/burrelvannjr/Dropbox/Professional/Research/Projects/dissertation/chapters/ch2.movements-and-discourse/paper/tables/table6_08_29.tex}


%\input{/Users/burrelvannjr/Dropbox/Professional/Research/Projects/dissertation/chapters/ch2.movements-and-discourse/paper/tables/table1_11_18.tex}

%The second column of Table 1 removes the narrative variables of interest and instead includes control variables that may be associated with decreasing negative coverage. As can be seen, marijuana articles written in years when legalization is on the ballot in a state are less likely to be negative. In addition, articles written later in time and those written in states with higher percentages of Democratic voters were less likely to be negative. Conversely, articles written in states with higher percentages of college graduates and in states with larger populations are more likely to be negative. Finally, all other control measures are non-significant. 


%The third and final model incorporates all measures. Overall, with the exception of the measure for whether or not legalization was on the ballot in that year, all previously significant variables maintain their significance with the outcome. In addition, the lagged variables of prior movement coverage and prior positive marijuana coverage maintain their non-significant relationships with the outcome. 

\input{/Users/burrelvannjr/Dropbox/Professional/Research/Projects/dissertation/chapters/ch2.movements-and-discourse/paper/tables/table_pol.tex}

%--------------------------------------------------------------------------------------------------------------------------------------
\section{Conclusions} 

%Scholars of cultural change often focus on the role of movements in the discursive change process. 
In this article, I account for variation in the discourse about marijuana by considering how narratives have changed over time. As I have demonstrated, marijuana discourse did, in fact, change over time. Firstly, I show that coverage of marijuana has increased during the period of study. We see peaks and valleys in coverage, but also a general upward trend for attention to marijuana. Yet, not only did coverage discourse about marijuana become less negative, and more neutral, over time, but the ways in which marijuana was discussed also changed over time. There was a general broadening, over time, of the frames used to discuss marijuana. While most coverage of marijuana in 1990 included relatively equal numbers of various frames, frames became more varied over time, with frames about revenue creation coming to dominate a large number of articles. In most years, linking marijuana coverage with frames of revenue creation and politics was most common, whereas frames about liberty, rehabilitation, and policing were least likely. %Conversely, the shift on marijuana discourse was the result of journalists linking marijuana discussions to larger narratives on beneficiary groups, American values, and traditional politics. Each narrative has impacts the likelihood of negative marijuana discourse, and this likelihood varies substantially across states. For example, fewer negative articles were written in places like California and Washington, whereas the likelihood of articles about marijuana being negative was higher in places like New Mexico and Arkansas. Part of the reason for negative coverage was due to direct democratic processes afforded in each state, while this coverage was also largely a function of what was said about marijuana. 

%After controlling for numerous other attributes of U.S. states, I still find a strong, statistically significant relationship between narratives and whether or not discourse about marijuana is negative. As I have argued, this relationship can be explained in terms of linking the contested topic of marijuana with more traditional beliefs, behaviors, and groups (including liberties, politics, and patients). 

The current study investigates the characteristics of discursive change on contested issues. I find that, over time, marijuana became linked with frames that centered on traditional citizen behavior (political behavior) and benefits to society (revenue creation). Additionally, this work contributes to a growing chorus of scholarship on discursive change \citep{bail_2012,bateman_et_al_2019}. In particular, this descriptive chapter broadens the scope of scholarly study by investigating the changing character of narratives about contentious political issues. It is my hope that this work will stimulate research on discursive and political factors that influence discursive change.


%In this article, I focused on how patterns of discussions shape dominant discourse on a controversial issue. It is my hope that this work will stimulate research on discursive and political factors that influence discursive change.
%\newpage

%\newpage
%--------------------------------------------------------------------------------------------------------------------------------------
%\section{References}

%\bibliographystyle{/Users/burrelvannjr/Dropbox/Professional/Research/References/asa_new}
%\renewcommand{\section}[2]{}%
%\setlength{\bibhang}{40pt}%matches the indentation above for references
%\bibliography{/Users/burrelvannjr/Dropbox/Professional/Research/References/library,/Users/burrelvannjr/Dropbox/Professional/Research/References/ext_library}
%\newpage



%--------------------------------------------------------------------------------------------------------------------------------------
%\section{Appendix}

%\input{/Users/burrelvannjr/Dropbox/Professional/Research/Projects/dissertation/chapters/ch2.movements-and-discourse/paper/tables/table_loc1.tex}



%\input{/Users/burrelvannjr/Dropbox/Professional/Research/Projects/dissertation/chapters/ch2.movements-and-discourse/paper/tables/table_smo1.tex}


