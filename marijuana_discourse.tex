
\chapter{The Discursive Shift on Marijuana Legalization}



%\begin{abstract}
%\begin{singlespace}
%What drives discourse about drugs? This study examines the dramatic decrease in negative coverage about marijuana in U.S. states from 1990 to 2016. I argue that, when focusing on liberties/freedoms, patients, and linking marijuana legalization with voting, the dominant discourse on contested issues can shift away. Results from logistic regression analyses in newspaper articles demonstrate that the shift away from negative discourse on marijuana was significantly influenced by shifting the discussion towards these topics. 
%\end{singlespace}
%\end{abstract}
%\newpage

\bibpunct[:]{(}{)}{;}{a}{}{,} % new punctuation for these, replace parentheses before and after citealt, and replace citealt with citep



%--------------------------------------------------------------------------------------------------------------------------------------
\section{Introduction}


Examination of frames, narratives, and discourse has been central to sociological understanding of cultural and political change \citep{ghaziani_and_baldassarri_2011,vasi_et_al_2015,bail_2012,mccammon_et_al_2007}. I contribute to this line of inquiry by investigating whether and how framing about marijuana changed over time. Coverage of marijuana presents an ideal case with which to examine discursive shifts, given that the ways in which marijuana was covered has moved away from negative discussions that centered on crime and danger \citep{newhart_and_dolphin_2018,bonnie_and_whitebread_1970,rosenthal_and_kubby_1996}, towards positive and neutral discussions regarding the benefits of marijuana and marijuana legalization \citep{mosher_and_akins_2019}. To what extent has there been a general shift away from negative coverage of marijuana? And to what extent have new frames come to dominate the discussion of marijuana? 


%I advance this body of work by examining the outcome of discursive shifts about marijuana. Given that, until recently, many politicians did not want to tackle the marijuana issue for fear of appearing less-than-tough on crime \citep{rosenthal_and_kubby_1996}, shaping media discourse about marijuana proved critical for advancing legislation on marijuana. Why did coverage of marijuana change, or become less negative, over time? Most importantly, to what extent did the narratives about marijuana contribute to this shift?

%I argue that discursive change is initiated by a variety of mechanisms, and that discourse changes by altering the conversation attached to an issue. Reframing of an issue in new and  culturally relevant ways \citep{gamson_and_modigliani_1989} that provide alternative understandings to longstanding debates \citep{bail_2012,snow_2004} can shift public perceptions and dominant discourse on an issue over time \citep{ghaziani_and_baldassarri_2011,baldassarri_and_bearman_2007}.

With recent advances in the field of computational social science, it has become easier than ever to gather, code, and analyze large corpuses of text for the purpose of tracking discursive trends \citep{bateman_et_al_2019,bail_2012,vasi_et_al_2015}. As such, I rely on webscraping techniques to gather newspaper articles about marijuana over time, breaking these articles as either covering marijuana with or without coverage of a marijuana advocacy organization. I, next, rely on automated text analysis to code the valence of each news article. Finally, using a combination of inductive and deductive approaches, I create a list of eight possible frames (and keywords associated with these frames) that may have appeared in marijuana coverage. I use automated coding to categorize each article has having the presence or absence of each frame. 


In this article, I shed light on the ways in which frames about marijuana changed in the United States between 1990 (when marijuana initiatives began to take off) and 2016. I demonstrate various characteristics about marijuana coverage and discourse over time. First, I show that, after 1996, when marijuana medicalization was on the ballot in California, coverage of marijuana increased dramatically. Secondly, I show that coverage of marijuana became decreasingly negative and increasingly neutral over time. Thirdly, I show that frames centering on revenue creation and politics came to dominate the discussion of marijuana, followed by framings related to rights and patients. 


% various framings or narratives about contested issues impact overall discourse. I do so by examining the impact narratives about marijuana on general media coverage of marijuana in the U.S. from 1990 to 2016. In particular, I examine whether the shift towards less negative discourse about marijuana resulted from various framings about marijuana. 


%American public opinion on marijuana legalization has undergone rapid change over the past two decades (Gallup 2013; Pew Research Center 2013). In 2006, a Colorado ballot initiative that would have legalized marijuana was handedly defeated; fewer than one quarter of Colorado counties supported the initiative. The issue was revisited in 2012, and with 55 percent of the vote, ushered in a new era of marijuana reform in the United States. The rapid growth in support for legalization set off a wave of state-level activity (as shown in Table 1), as organizations, voters, and politicians attempted to reform marijuana policy. Despite growing support for legalization, opponents generally perceive legalization as harmful to society; supporters, on the other hand, view legalization as an opportunity to remedy social problems (Caulkins et al. 2012; Rosenthal and Kubby 1996). Supporters of legalization argue that regulation and taxation of marijuana would help reduce crime, addiction, and drug abuse while also providing necessary revenue for educational and rehabilitative programs. 

%Perceptions of opportunity are the product of social construction processes, developed through social interaction, and are dependent upon the contexts in which people are embedded. In this article, I examine how residential segregation in local contexts that is based on the presence of households with (or without) children ? what I term, parental segregation ? influences voting behavior in support of policy change. This investigation is rooted in my understanding of how segregation has implications for how opportunity is perceived and acted on. Class-, race-, and education-based segregation operate by concentrating opportunities for some portion of the population while restricting access for others. Class segregation, for example, can restrict poor residents? access to quality education and employment (Massey and Denton 1993; Wilson 1987). Moreover, segregation concentrates individuals who have the knowledge and resources necessary to recognize and act on those opportunities. Importantly, segregation can facilitate reactive political action when it is vulnerable to penetration, or threatened by the increasing presence or political power of minority groups (McVeigh et al. 2014; Beisel 1997, 1990; Andrews and Seguin 2015). Limited contact with minority groups can lead majority group members to view racial minorities and the poor as responsible for their disadvantaged condition (Beisel 1990, 1997; Higham 1955). However, these beliefs may not resonate where actual threats do not exist. In fact, segregation concentrates residents of similar groups and decreases intergroup exposure, which can mobilize progressive political action (Galbraith and Hale 2008).

%The recent rise in ballot initiatives devoted to marijuana legalization provides a unique opportunity to examine how structural features of local contexts shape political behavior related to opportunity. The fight over marijuana?s legality is important in its own right because it involves battles over meanings of civil liberty, drugs, medicine, abuse, and morality. The growth in support for legalization is in part driven by backlash against prohibition and claims that legalization could contribute to safer communities by helping to shrink the size and influence of the illegal drug market (Marijuana Policy Project 2015; Rosenthal and Kubby 1996; Caulkins et al. 2012; Blocker 1989). While opposition exists, it remains highest amongst groups arguing that legalization would negatively impact communities, families, and children ? namely parents.

%In this article, I aim to shed additional light on marijuana legalization in the United States. I do so by considering whether the residential segregation of households with children from households without children (or, put another way, parents from nonparents) may facilitate support for policy change, such as marijuana legalization, that provides an opportunity to remedy social ills. Moreover, because perceptions of opportunity resulting from potential policy change is reflected in structural relations, changes in local communities may have a diminishing effect on support for marijuana reform. As such, I expect the effects segregation to be strong in unchanging communities.




%--------------------------------------------------------------------------------------------------------------------------------------
\section{Discursive Change Processes}

Discourse has been the subject of recent academic work, with most focusing on the role of social movements or advocacy organizations in initiating discursive change \citep{bail_2012,earl_2004,mccammon_et_al_2007,ghaziani_and_baldassarri_2011,vasi_et_al_2015,gamson_and_modigliani_1989}. In particular, organizations offer their own diagnoses of and solutions to problems and can impact public discourse on an issue injecting novel, highly resonant frames into public discussion \citep{benford_and_snow_2000}. Frames that `fit' the broader discursive environment \citep{mccammon_et_al_2007} or those that articulate widespread beliefs and values usually survive over alternative frames \citep{mccammon_et_al_2001,snow_et_al_2007,gamson_and_modigliani_1989}, because their frames ``are more easily integrated into broader media narratives'' on an issue because these frames seem more familiar \citep{bail_2012} and appear realistic or legitimate. 

Coverage of issues is critical to this process \citep{amenta_et_al_2009,ferree_et_al_2002}. Therefore, mass media are important to cultural change. Mass media are central for making sense of relevant events \citep{gamson_and_modigliani_1989}, and serve to identify and redefine issues for the public. These discussions can shape audience perceptions. While media are a master forum within which actors compete for coverage of their issues \citep{amenta_et_al_2012}, media organizations themselves operate by a set of procedures that can also have effects on discourse on issues. 


News organizations make decisions about what counts as ``news'' \citep{galtung_and_ruge_1965}. News is selected based on its timeliness/currency, the impact of the events to be covered, and the proximity of those events to potential readers (with local news angles being considered important, particularly in national stories). In particular, politics receives the most coverage because political decisions have high impact and include prominent people. In addition, reporters often have increased access to political actors. In sum, much of what counts as news revolves around institutional political activity, including stories about politicians, bills being discussed or laws being passed \citep{amenta_et_al_2012}. Therefore, it is reasonable to assume that discursive change on marijuana may have occurred by way of institutional political actors. However, as is the case for many hotly contested issues like marijuana, engaging in discussions on these topics provides little political advantage, rendering many political actors unwilling to even discuss the issue. In particular, when dominant discourse on a topic is negative, political actors will be unwilling to discuss the issues in new, more positive ways, for fear of reprisals from their constituents in the form of lost votes. 

Given the tendency of journalists and the norms of news-gathering organizations to seek out official sources \citep{schudson_2002,gitlin_1980,gans_1979}, especially during times of high attention to the issue in the news cycle \citep{baumgartner_and_jones_1993} the general lack of discussion by political actors creates opportunities for other actors (including journalists themselves) to, not only, provide alternative narratives on an issue, but shift the character of discourse by linking the issue with alternative topics. 

%that are different from dominant discourse on that topic can be a political disadvantage


I argue that, by linking discourse about contested issues to additional, more institutionalized topics or narratives, journalists can facilitate a stark transformation in public understandings on an issue, and call into question the legitimacy of dominant representations. Initial marijuana discourse centered on criminality and the negative educational, occupational, and mental effects of marijuana use. In recent years, however, this discourse has given way to increasingly positive (or fewer negative) discussions of marijuana's medical, community, and economic benefits. This coverage has shifted the arenas within which marijuana was discussed by linking marijuana with narrative topics related to medicine, rights, freedom, economics, crime, and policing. Yet it remains unclear whether and how these new narrative topics contributed to the discursive shift away from a dominant negative discourse about marijuana. 

%In addition, the movement has devoted much of it's resources not on protest as a means of changing public definitions of marijuana, but on local and statewide initiatives, gaining media coverage of their issue, and on holding local community-based meetings. 


%--------------------------------------------------------------------------------------------------------------------------------------
\section{Marijuana Discourse Across Time}


According to historians studying marijuana, initial depictions of marijuana were positive, centering on the medicinal and material benefits of cannabis \citep{bonnie_and_whitebread_1970,rosenthal_and_kubby_1996}. While there was relatively little coverage of marijuana during this time \citep{bonnie_and_whitebread_1970,mosher_and_akins_2019}, what little coverage did exist was positive. However, during the 1930s, the valence of this coverage shifted as a result of political and bureaucratic changes. 

In 1930, President Herbert Hoover established the Federal Bureau of Narcotics (FBN) and appointed Harry J. Anslinger as commissioner. According to scholars, Anslinger was in charge of repurposing prohibition funds, and, in noticing that Americans were enjoying Mexican and Native-American cannabis, Ansligner chose to direct his Bureau's resources toward cannabis \citep{newhart_and_dolphin_2018}. In fact, Anslinger used media to paint cannabis in a negative light as a way to ramp up public opposition to the substance. Anslinger used stories and advertisements in newspapers (including William Randolph Hearst's newspapers) to portray cannabis as dangerous to women, children, and society \citep{mosher_and_akins_2019}. Hearst, for his part, stood to lose economically if American cannabis use expanded -- he invested in wood pulp, which he used for his papers, and the expansion of hemp (which could also be used for cheaper newspaper manufacturing) put him in danger of losing his fortune. Through a campaign of ``yellow'' journalism, which enabled Anslinger to rebrand the drug with the more Native-sounding name marihuana (or marijuana) instead of cannabis, Anslinger and Hearst were able to associate the drug with a source or group of people responsible for the drug problem: immigrants, Mexicans, and indigenous ``others.'' Through newspapers, Anslinger and Hearst were able to ``sell'' marijuana as dangerous -- relying on a fear narrative that argued that only through cannabis prohibition could America's children, women, and society be protected \citep{mosher_and_akins_2019,newhart_and_dolphin_2018,rosenthal_and_kubby_1996}. 

Within a few years, these narratives took hold. In fact, over this time, marijuana became increasingly associated with criminality -- and in particular, minority criminality. During this time, marijuana was thought to be used mainly by minorities (freed Black slaves and Mexican immigrants) and had psychological properties that made them more prone to violence \citep{caulkins_et_al_2012,slaughter_1987}. These dominant narratives came to be used in arguments to Congress in favor of a full ban on marijuana -- resulting in the creation of the Marihuana Tax Act of 1937 \citep{newhart_and_dolphin_2018}. 



Between the 1930s and 1960s, the Act faced tough criticism in state and federal courts, as the judicial system worked to clarify the parameters of the law as well as what could and could not be enforced \citep{bonnie_and_whitebread_1970,mosher_and_akins_2019}. Around this time, advocacy organizations emerged to fight for access to marijuana \citep{newhart_and_dolphin_2018}. In 1970, the first marijuana movement organization, the National Organization for the Reform of Marijuana Laws, was created to fight against marijuana prohibition and to move public opinion on marijuana so as to enable full legalization of marijuana for all people. It wasn't until the mid-1990s and early 2000s (during NORML's fight for medicinal marijuana use on the California ballot) that other organizations such as the Marijuana Policy Project (1995), Students for Sensible Drug Policy (1998), and the Drug Policy Alliance (2000) joined the fight -- each with a specific purpose for legalization. For example, MPP would work on marijuana policy specifically while DPA would focus on both marijuana and similar narcotic policies, and SSDP would work to change the minds of youth, particularly on college campuses. 

According to prior research, marijuana discourse in the shifted in the 1960s. During this time, coverage of marijuana centered on the freedom to use the drug \citep{mosher_and_akin_2019}. 

In the mid-1990s, private individuals began to fight against marijuana prohibition in the United States by sponsoring marijuana medicalization initiatives in states with direct democratic processes. During this same time, journalists began to 1) give increasing attention to marijuana, and 2) link discussions of marijuana with additional topics. Some scholars have indicated that during this time, narratives about marijuana began to become more positive, as coverage tended to focus on the political aspects of initiatives or voters, the benefits to patients, and the rights of cannabis users \citep{newhart_and_dolphin_2018,mosher_and_akins_2019,bonnie_and_whitebread_1970}. Over time, the trajectory of marijuana legalization is such that states (under federal prohibition of marijuana, and those with the initiative/referendum process) would first propose medical legislation, followed by recreational legalization legislation. During this time, marijuana coverage began to link with larger ``American'' values of liberties and freedom, in addition to shifting to the benefits of legalization for revenue, creating resources for rehabilitation, decreasing crime, and altering policing practices -- especially in communities of color \citep{mosher_and_akins_2019,newhart_and_dolphin_2018}.

Given this shift, I expect that these various narratives should have distinct impact in the decrease in negative attention to marijuana. Below, I outline the ways in which I study these effects. 

%\subsection{\textit{The Shift: Reframing and Positive Discourse about Marijuana}}



%\input{/Users/burrelvannjr/Dropbox/Professional/Research/Projects/dissertation/chapters/ch2.movements-and-discourse/paper/figures/figure2.tex}

%\input{/Users/burrelvannjr/Dropbox/Professional/Research/Projects/dissertation/chapters/ch2.movements-and-discourse/paper/figures/figure3.tex}


%--------------------------------------------------------------------------------------------------------------------------------------
\section{Data \& Method}

Given my interest in \textit{whether} and \textit{how} marijuana discourse shifted over time, I analyze text from print news media across the United States. To do so, I rely on the ProQuest newspaper database. I constrain the analysis to 1990 and on because coverage on marijuana was relatively low prior to 1990, and because this time frame immediately followed Reagan's intensified ``War on Drugs'' and ``Just Say No'' campaign. 

To track discursive change, I rely on articles about marijuana in the Proquest database from 1990 to 2016. Because marijuana advocacy organizations may have had an impact on coverage, I separately searched for articles about marijuana in the absence of advocacy organizations, and articles about marijuana that included advocacy organizations. To accomplish this,  I wrote a Python script to identify and download all local articles from Proquest that mention ``marijuana'' between 1990 and 2016.\footnote{This does not include variants of the word marijuana, or the word cannabis} Because national newspapers may be more likely to cover national issues over local issues, I exclude national newspapers, including the \textit{New York Times}, the \textit{Los Angeles Times}, the \textit{Washington Post}, and the \textit{Wall Street Journal}. In addition, I exclude articles that mention at least one of the four main marijuana advocacy organizations. Therefore, I also exclude articles that mention National Organization for the Reform of Marijuana Laws (NORML), Marijuana Policy Project (MPP), Drug Policy Alliance (DPA), and Students for Sensible Drug Policy (SSDP), and their variants. In total, there were 14,163 articles mentioning marijuana. After removing duplicate articles, articles outside of the U.S. or located in the U.S. capitol\footnote{ProQuest sometimes mistakenly identifies non-U.S. articles when only-U.S. articles are specified.}, short articles (e.g. articles with fewer than 100 words), and articles that are not fully searchable,\footnote{Articles with fewer than about 900 words.}, I am left with 10,096 locally-based articles that mention marijuana in some fashion. In addition, I removed articles that come from ``alternative'' or sensationalized newspapers. To figure out whether or not the newspaper was an ``alternative newspaper,'' I searched the websites for each newspaper, removing any newspaper that claimed that it was an alternative newspaper. In sum, I am left with  5,893 articles about marijuana which do not include mention of marijuana advocacy organizations. %498 are positive, 74 have plagiarism


Because marijuana advocacy organizations' discussion of marijuana may be important for discursive change on marijuana, I also include coverage of ``marijuana'' alongside coverage of marijuana advocacy organizations. As such, I wrote a separate Python script to identify and download all articles from Proquest that mention ``marijuana'' and any one of the four largest marijuana advocacy organizations (and the variants of their names) between 1990 and 2016. Therefore, the script was able to capture all coverage of ``marijuana'' coupled with coverage of marijuana advocacy organizations, including the National Organization for the Reform of Marijuana Laws (NORML), Marijuana Policy Project (MPP), Drug Policy Alliance (DPA), and Students for Sensible Drug Policy (SSDP).\footnote{Importantly, I separate these sets of coverage for future empirical work on the impact of organizations on the discursive shift.} In total, there were 1,616 articles mentioning a marijuana movement organization. After cleaning the data set of articles by removing duplicate articles, I am left with 1,150 articles mentioning marijuana advocacy organizations. In addition, after removing and articles coming from alternative news sources, I am left with 787 marijuana organization-related articles.




%The unit of analysis is at the article level.%I focus on local rather than national level discourse in print media given recent criticism against relying on national media sources \citep{earl_et_al_2004}\footnote{I therefore exclude the \textit{New York Times}, \textit{Los Angeles Times}, \textit{Wall Street Journal}, and \textit{Washington Post}.}, and because marijuana movement organizations focus mainly on local level activism which may garner media attention (e.g. chapter meetings and campaigns for ballot initiatives) instead of distributing press releases.\footnote{Unlike women's jury rights studied by \citet{mccammon_et_al_2007} or the civil society organizations targeting Islam and Muslims studied by \citet{bail_2012}, many of the marijuana movement organizations show little evidence of information distribution, therefore, locating items such as press releases, speeches, pamphlets or flyers, transcripts of television or public speeches proves difficult. In fact, much of the organizations activity centered on local chapter meetings (which rarely include meeting minutes or are transcribed) and public action during ballot initiative campaigns or events such as HempFest. Much of the movement's contemporary activism is geared towards maintaining a social media presence for local chapters. This activity, again, centers on promoting local initiatives. In the future, I hope to diversify the explanatory sources to include any movement-generated documents that have been found to influence media (see \citealt{sobieraj_2011}) including mission statements, meeting minutes, and newsletters/flyers promoted.}

%However, there is considerable variability in the extent to which local articles report on marijuana. For example, some newspapers may report on marijuana only once in a given year, not at all the subsequent years, and heavily many years later. Such a pattern of reporting across local news sources creates analytical problems. Ultimately, because discourse is often constrained by the environments in which they take place \citep{mccammon_et_al_2007}, I examine the incidence of positive marijuana discourse in each U.S. state between 1990 and 2016, using state-years as the unit of analysis. State-years provide comparative leverage because I can compare 1,350 cases (27 years across 50 states). A state level analysis, using state-years allows me to account for state-level heterogeneity, and differences across years.

%Because I am interested in how discourse changes over time, it is necessary to track changes in marijuana discourse over time. %As such, the main dependent variable is binary -- whether or not the article about marijuana is predominantly negative. I therefore use logistic regression to estimate the models. To reduce the risk of biased estimates, I use fixed effects models. The fixed effects design explicitly models the change that occurs within states over time, therefore, the results are identical to those that would be obtained if I manually inserted a dichotomous variable for every state. One important advantage of the fixed-effects model is that it controls for all constant, but unobserved and unmeasured, differences across our cases (Allison 1994). Because I estimate change within states over time, omitted variables are problematic only if they are time-variant.




\input{/Users/burrelvannjr/Dropbox/Professional/Research/Projects/dissertation/chapters/ch2.movements-and-discourse/paper/figures/figure1_a.tex}

\input{/Users/burrelvannjr/Dropbox/Professional/Research/Projects/dissertation/chapters/ch2.movements-and-discourse/paper/figures/figure1.tex}

\input{/Users/burrelvannjr/Dropbox/Professional/Research/Projects/dissertation/chapters/ch2.movements-and-discourse/paper/figures/figure1_1.tex}

\subsection{\it{Dependent Variable}}

Because I am interested in the shift away from negative discourse about marijuana, I categorize each article based on it's polarity or valence.\footnote{To prepare all documents for textual analysis, following the procedure used by \citet{bail_2012}, I use software in R to transform each article into fully-searchable sets of words, and clean the textual data by eliminating excessive words (e.g. stop-words such as numbers, conjunctions, and determiners), and transforming each word into it's stem variant.} I code each article with the assistance of a na\"{i}ve Bayes classifier in R's \texttt{sentiment} package \citep{jurka_2012}. The na\"{i}ve Bayes algorithm uses a stock of trained text that has been associated with three types of polarity (positive, neutral, or negative), and tries to classify each document as one of the three polarities. I then create dummy codes for each article based on its polarity. 



%. A key principal in computational social science is ``supervised'' learning, wherein the researcher does a small set of qualitative coding, the algorithm learns from those codes to generate most similar codes for the rest of the data. The na\"{i}ve Bayes classifier is best suited for training on a previously-coded, small subset of data, learning from those codes, and classifying those data based on the researcher-generated codes but also provides the option of ``NULL'' where it cannot appropriately classify. Alternatively, the researcher could conduct qualitative coding on an already coded subset of data to see if his/her codes align with the algorithm-generated codes. For polarity and emotional content, I will engage in the latter. However, to understand the exact \textit{frames} the movement put forth, I will engage in the former. Both of these methods of textual analysis, using semi-automated or supervised learning techniques will provide an interrater reliability score by which I can demonstrate consistency in coding. Moreover, both will be the bulk of my remaining work.} 

%\subsection{\it{Independent Variable}}


Given my interest in the impact of narrative frames on decreasing negative coverage of marijuana, I code each sample of articles for the frames that exist. Using inductive processes based on prior research on arguments in favor of legalization \citep{newhart_and_dolphin_2018}, I identify eight narrative frames that may be present in each article. These include topics covering ``rights,'' ``liberties,'' ``revenue,'' ``patients,'' ``policing,'' ``crime reduction,'' ``politics,'' and ``rehabilitation.'' In Table 1 below, I outline the search terms used for identifying these frames. 

\input{/Users/burrelvannjr/Dropbox/Professional/Research/Projects/dissertation/chapters/ch2.movements-and-discourse/paper/tables/table1new.tex}


\input{/Users/burrelvannjr/Dropbox/Professional/Research/Projects/dissertation/chapters/ch2.movements-and-discourse/paper/figures/figure2new.tex}



%Given my argument about the influence of the movement's discourse on marijuana discourse generally, I use novel plagiarism detection software, developed by \citet{welbers_and_van_atteveldt_2016}, to link articles with movement-initiated marijuana discourse to subsequent articles about marijuana. The software\footnote{\textit{RNewsflow}, in the R statistical environment, is designed to analyze homogeneity in text between two news articles (or corpora of text) while also tracing the diffusion of similar subsets of text across time.} compares strings of text in articles in the explanatory set (movement discourse about marijuana) to strings of text in subsequent articles in the outcome set (general discourse about marijuana). For example, an article from May 1, 1970 in which NORML discusses marijuana or is mentioned alongside a discussion of marijuana will be compared to all articles in the outcome set (non-movement mentions of marijuana) that occur on or after May 2, 1970. Each movement article then receives a similarity score to represent the proportion of text that is reported verbatim in every ``non-movement'' article (the outcome set). The creators of this software recommend a minimum similarity score cutoff of .40, which has been a reliable threshold for identifying sets of text that address similar events. It is therefore possible for an article to have multiple scores, representing varying similarities with numerous documents. Each article in the outcome/dependent variable set is then dummy coded to represent whether or not what it's text contains verbatim text (with a score of .40 or above) from an article with coverage of a marijuana movement organization \textit{alongside} a discussion of marijuana. This means that the marijuana could be projecting it's own frames about marijuana or be given standing on the marijuana issue. This approach is similar to those used by other scholars studying media coverage of social movements \citep{amenta_et_al_2009,andrews_and_caren_2010}.

%It is important to note that, unlike many social movement organizations, the nature of the marijuana issue in political discourse has pushed marijuana activists away from the disruptive types of of action that woul 

\subsection{\it{Control Variables}}

I include variables to that may also account for less negative coverage on marijuana. Research on media coverage shows that advocacy organizations can also shape the direction of discourse \citep{seguin_2016,bail_2012}. I therefore include a measure for whether or not the article covers a marijuana advocacy organization. Research suggests that discursive opportunities are related to structural conditions \citep{mccammon_et_al_2007}. I therefore include for various state-level factors that may account for decreases in negative coverage of marijuana. First, from the Secretary of State websites for each state in each year, I include 1) a measure of whether or not marijuana legalization was on the ballot in that state and in that year, and 2) whether or not marijuana had been or was already medicalized in that year. 

In addition, I include various other state-level controls, many of which come directly from the Census and the American Community Survey. I match each decennial Census with the year it was taken and the following years not covered by the subsequent Census, matching the 1990 Census with years 1990 through 1999 and the 2000 Census with years 2000 through 2008. For years 2009 through 2016, however, I use the 2005-2009 American Community Survey (ACS).\footnote{I exclude data from Alaska due to availability.} Because each of these data are measured only during Census years, I use linearly interpolate values for interim years. Firstly, the number of newspaper articles about marijuana may be a function of population. As such, I include a measure for the natural log of the total population in a state. Second, recent research has demonstrated that there is higher support for marijuana in locales with higher percentages of college graduates and liberal voters \citep{caulkins_et_al_2012,rosenthal_and_kubby_1996}. As such, I include a measure for the percent of the total population aged 25 or older with a four-year college degree. It is reasonable to assume that decreasingly negative coverage of marijuana could have resulted from severe economic conditions in locales (e.g. marijuana may be perceived as beneficial for strengthening local economies)  \citep{caulkins_et_al_2012,caulkins_2010,miron_2010}. For this reason, I include a measure for the percent of the population aged 16 or older that is employed. 





Finally, given that Democratic political officials and voters tend to exhibit higher support for legalization, I use data from Congressional Quarterly's {\it{America Votes}} to measure the percentage of voters who voted for the Democratic candidate in the presidential elections that coincide with, or immediately precede, yearly data. This means that for each presidential election year during the period from 1990 to 2016, values are calculated directly from voter percentages. All years between presidential election years are linearly interpolated. For example, articles written in 1990, I use the percent of the vote for Michael Dukakis in 1988, linearly interpolated from 1988 to 1992, using the values for 1990. For articles written in 1992 and 1996, I use the percent vote for Bill Clinton. For articles written in 2000, I use the percent of the vote for Al Gore. Articles written in 2004 are associated with the percent of the vote for John Kerry, while articles written in 2008 and 2012 are associated with the percent of the vote for Barack Obama. Finally, for articles written in 2016, I use the percent of the vote for Hillary Clinton. To be clear, values for years between presidential elections are calculated using linear interpolation. 



%--------------------------------------------------------------------------------------------------------------------------------------
\section{Results}

Table 2 presents logistic regression results for the likelihood of negative articles about marijuana in each state from 1990 to 2016, including state level fixed effects. Column 1 includes only the variables of interest, various narratives about marijuana. As shown, the coefficients for liberties, patients, and politics narratives about marijuana all have significant negative effects on the likelihood of negative coverage about marijuana in the U.S. This provides support for my claim that linking marijuana with American values narratives, in addition to beneficiary groups and traditional political processes is relevant for the overall decrease in negative discourse about marijuana. On the other hand, articles that include narratives about policing or revenue were more likely to be negative. Coefficients in logistic regression can be  interpreted by exponentiation, or $(e^{b} - 1)*100$, which gives the expected change in the dependent variable that coincides with a one-unit increase in the independent variable. In sum, articles with narratives about liberties, patients, or politics were had decreasing likelihoods of being negative -- they were associated with a 31, 22, and 29 percent decrease in the likelihood of negative coverage about marijuana.

\input{/Users/burrelvannjr/Dropbox/Professional/Research/Projects/dissertation/chapters/ch2.movements-and-discourse/paper/tables/table2new.tex}


%\input{/Users/burrelvannjr/Dropbox/Professional/Research/Projects/dissertation/chapters/ch2.movements-and-discourse/paper/tables/table5_3-21c.tex}


%\input{/Users/burrelvannjr/Dropbox/Professional/Research/Projects/dissertation/chapters/ch2.movements-and-discourse/paper/tables/table6_08_29.tex}


%\input{/Users/burrelvannjr/Dropbox/Professional/Research/Projects/dissertation/chapters/ch2.movements-and-discourse/paper/tables/table1_11_18.tex}

The second column of Table 1 removes the narrative variables of interest and instead includes control variables that may be associated with decreasing negative coverage. As can be seen, marijuana articles written in years when legalization is on the ballot in a state are less likely to be negative. In addition, articles written later in time and those written in states with higher percentages of Democratic voters were less likely to be negative. Conversely, articles written in states with higher percentages of college graduates and in states with larger populations are more likely to be negative. Finally, all other control measures are non-significant. 


The third and final model incorporates all measures. Overall, with the exception of the measure for whether or not legalization was on the ballot in that year, all previously significant variables maintain their significance with the outcome. In addition, the lagged variables of prior movement coverage and prior positive marijuana coverage maintain their non-significant relationships with the outcome. 

%--------------------------------------------------------------------------------------------------------------------------------------
\section{Conclusions} 

Scholars of cultural change often focus on the role of movements in the discursive change process. 
Yet, as I have demonstrated, advocacy organizations have little impact on the shifting discourse about marijuana. Conversely, the shift on marijuana discourse was the result of journalists linking marijuana discussions to larger narratives on beneficiary groups, American values, and traditional politics. Each narrative has impacts the likelihood of negative marijuana discourse, and this likelihood varies substantially across states. For example, fewer negative articles were written in places like California and Washington, whereas the likelihood of articles about marijuana being negative was higher in places like New Mexico and Arkansas. Part of the reason for negative coverage was due to direct democratic processes afforded in each state, while this coverage was also largely a function of what was said about marijuana. 

In this article, I account for variation in the likelihood of negative discourse about marijuana by considering how various aspects of narratives, advocacy organizations, and opportunity structures. After controlling for numerous other attributes of U.S. states, I still find a strong, statistically significant relationship between narratives and whether or not discourse about marijuana is negative. As I have argued, this relationship can be explained in terms of linking the contested topic of marijuana with more traditional beliefs, behaviors, and groups (including liberties, politics, and patients). 

The current study addresses gaps political sociology and communication studies by investigating structural and story-centered effects on popular discourse. Additionally, this work contributes to a growing chorus of scholarship on the consequences of narratives on discourse \citep{bail_2012,bateman_et_al_2019}. In particular, this research broadens the scope of scholarly study by empirically investigating the impacts of narratives on discursive change. 

In this article, I focused on how patterns of discussions shape dominant discourse on a controversial issue. It is my hope that this work will stimulate research on discursive and political factors that influence discursive change.
%\newpage

%\newpage
%--------------------------------------------------------------------------------------------------------------------------------------
%\section{References}

%\bibliographystyle{/Users/burrelvannjr/Dropbox/Professional/Research/References/asa_new}
%\renewcommand{\section}[2]{}%
%\setlength{\bibhang}{40pt}%matches the indentation above for references
%\bibliography{/Users/burrelvannjr/Dropbox/Professional/Research/References/library,/Users/burrelvannjr/Dropbox/Professional/Research/References/ext_library}
\newpage



%--------------------------------------------------------------------------------------------------------------------------------------
\section{Appendix}

%\input{/Users/burrelvannjr/Dropbox/Professional/Research/Projects/dissertation/chapters/ch2.movements-and-discourse/paper/tables/table_loc1.tex}

\input{/Users/burrelvannjr/Dropbox/Professional/Research/Projects/dissertation/chapters/ch2.movements-and-discourse/paper/tables/table_all1.tex}

%\input{/Users/burrelvannjr/Dropbox/Professional/Research/Projects/dissertation/chapters/ch2.movements-and-discourse/paper/tables/table_smo1.tex}


